% !Mode:: "TeX:UTF-8"

\chapter{模板介绍与注意事项}

\section{模板说明}

本模板是在天津大学、及山东大学软件工程学院的模板格式的基础之上进行修改之后适合海南大学本科毕业生撰写本科毕业论文而编写的~\LaTeX~论文模板。在使用本模板之前建议对~\LaTeX~进行简要的了解,熟悉其文档的结构。
由于个人水平有限,虽然现在的这个版本基本上满足了学校的要求,但难免存在不足之处,欢迎大家积极反馈,更希望山海南大学~\LaTeX~爱好者能一同完善此模板,让更多同学受益。

如有模板的疑问或有意向加入模板的维护和编写队伍中来,请给作者: 孙浩然(sunhr620@foxmail.com)写信。

\section{目录内容}
本~\LaTeX{}~模板的源文件即为本科毕业设计中使用的模板,用户可以通过修改这些文件来编辑自己的毕业设计论文。
\begin{itemize}
\item{hnumain.tex}:主文件,包含封面部分和其他章节的引用信息。
\item{preface}: 包含本科毕业设计的封面和中英文摘要。
\item{body}: 包含本文正文中的所有章节。
\begin{itemize}
\item{intros.tex}: 包括本~\LaTeX{}~模板的介绍,编译方法和使用方法。
\item{figures.tex}: 包含论文中图片的插入和引用方法。
\item{tables.tex}: 包含论文中表格的插入和引用方法。
\item{equations.tex}: 包含论文中数学符号、公式的书写和排版方法。
\item{others.tex}: 包含论文中使用的罗列环境,定理环境等其他环境的排版方法。
\item{conclusion.tex}: 包含本文的总结。
\end{itemize}
\item{setup}:存放论文所使用的宏包和全文格式的定义。
\item{appendix}:存放论文的外文资料,中文译文和致谢部分。
\item{references/reference.bib}:存放论文所引用的全部参考文献信息。
\item{xiong.bat}:双击此批处理文件进行文档的编译生成。

\item{HNUThesis.bst}:默认的~BibTeX~样式文件,如果你想修改样式,如~IEEE~要求的样式, 只需将~IEEEtran.bst~的文件名~改为~HNUThesis.bst~即可。
\end{itemize}
需要说明的是,以上文件名并不是固定的,各位同学可以新建一个~tex~文件,例如~algorithm.tex,放在~body~目录下,并且在~hnumain.tex~中调用:
\begin{VerbWithBreak}
    \include{body/algorithm.tex}
\end{VerbWithBreak}
来引用之。当然你也可以重命名这些文件,只要~include~中的文件名是存在且合法,~\LaTeX~总能找到这些文件的。

在你写作某一章节的时候,你可能需要随时预览排版效果并~Debug,这时你可以在其他章节的\verb|\include|命令前加上一个\%,这代表注释掉本行,例如:
\begin{VerbWithBreak}
%%%%%%%%%%%%%%%%%%%%%%%%%%%%%%%%
           正文部分
%%%%%%%%%%%%%%%%%%%%%%%%%%%%%%%%
\mainmatter
% !Mode:: "TeX:UTF-8"

\chapter{模板介绍与注意事项}

\section{模板说明}

本模板是在天津大学、及山东大学软件工程学院的模板格式的基础之上进行修改之后适合海南大学本科毕业生撰写本科毕业论文而编写的~\LaTeX~论文模板。在使用本模板之前建议对~\LaTeX~进行简要的了解,熟悉其文档的结构。
由于个人水平有限,虽然现在的这个版本基本上满足了学校的要求,但难免存在不足之处,欢迎大家积极反馈,更希望山海南大学~\LaTeX~爱好者能一同完善此模板,让更多同学受益。

如有模板的疑问或有意向加入模板的维护和编写队伍中来,请给作者: 孙浩然(sunhr620@foxmail.com)写信。

\section{目录内容}
本~\LaTeX{}~模板的源文件即为本科毕业设计中使用的模板,用户可以通过修改这些文件来编辑自己的毕业设计论文。
\begin{itemize}
\item{hnumain.tex}:主文件,包含封面部分和其他章节的引用信息。
\item{preface}: 包含本科毕业设计的封面和中英文摘要。
\item{body}: 包含本文正文中的所有章节。
\begin{itemize}
\item{intros.tex}: 包括本~\LaTeX{}~模板的介绍,编译方法和使用方法。
\item{figures.tex}: 包含论文中图片的插入和引用方法。
\item{tables.tex}: 包含论文中表格的插入和引用方法。
\item{equations.tex}: 包含论文中数学符号、公式的书写和排版方法。
\item{others.tex}: 包含论文中使用的罗列环境,定理环境等其他环境的排版方法。
\item{conclusion.tex}: 包含本文的总结。
\end{itemize}
\item{setup}:存放论文所使用的宏包和全文格式的定义。
\item{appendix}:存放论文的外文资料,中文译文和致谢部分。
\item{references/reference.bib}:存放论文所引用的全部参考文献信息。
\item{xiong.bat}:双击此批处理文件进行文档的编译生成。

\item{HNUThesis.bst}:默认的~BibTeX~样式文件,如果你想修改样式,如~IEEE~要求的样式, 只需将~IEEEtran.bst~的文件名~改为~HNUThesis.bst~即可。
\end{itemize}
需要说明的是,以上文件名并不是固定的,各位同学可以新建一个~tex~文件,例如~algorithm.tex,放在~body~目录下,并且在~hnumain.tex~中调用:
\begin{Verbatim}[breaklines=true, breaksymbol=, breakanywheresymbolpre=]
    \include{body/algorithm.tex}
\end{Verbatim}
来引用之。当然你也可以重命名这些文件,只要~include~中的文件名是存在且合法,~\LaTeX~总能找到这些文件的。

在你写作某一章节的时候,你可能需要随时预览排版效果并~Debug,这时你可以在其他章节的\verb|\include|命令前加上一个\%,这代表注释掉本行,例如:
\begin{Verbatim}[breaklines=true, breaksymbol=, breakanywheresymbolpre=]
%%%%%%%%%%%%%%%%%%%%%%%%%%%%%%%%
           正文部分
%%%%%%%%%%%%%%%%%%%%%%%%%%%%%%%%
\mainmatter
% !Mode:: "TeX:UTF-8"

\chapter{模板介绍与注意事项}

\section{模板说明}

本模板是在天津大学、及山东大学软件工程学院的模板格式的基础之上进行修改之后适合海南大学本科毕业生撰写本科毕业论文而编写的~\LaTeX~论文模板。在使用本模板之前建议对~\LaTeX~进行简要的了解,熟悉其文档的结构。
由于个人水平有限,虽然现在的这个版本基本上满足了学校的要求,但难免存在不足之处,欢迎大家积极反馈,更希望山海南大学~\LaTeX~爱好者能一同完善此模板,让更多同学受益。

如有模板的疑问或有意向加入模板的维护和编写队伍中来,请给作者: 孙浩然(sunhr620@foxmail.com)写信。

\section{目录内容}
本~\LaTeX{}~模板的源文件即为本科毕业设计中使用的模板,用户可以通过修改这些文件来编辑自己的毕业设计论文。
\begin{itemize}
\item{hnumain.tex}:主文件,包含封面部分和其他章节的引用信息。
\item{preface}: 包含本科毕业设计的封面和中英文摘要。
\item{body}: 包含本文正文中的所有章节。
\begin{itemize}
\item{intros.tex}: 包括本~\LaTeX{}~模板的介绍,编译方法和使用方法。
\item{figures.tex}: 包含论文中图片的插入和引用方法。
\item{tables.tex}: 包含论文中表格的插入和引用方法。
\item{equations.tex}: 包含论文中数学符号、公式的书写和排版方法。
\item{others.tex}: 包含论文中使用的罗列环境,定理环境等其他环境的排版方法。
\item{conclusion.tex}: 包含本文的总结。
\end{itemize}
\item{setup}:存放论文所使用的宏包和全文格式的定义。
\item{appendix}:存放论文的外文资料,中文译文和致谢部分。
\item{references/reference.bib}:存放论文所引用的全部参考文献信息。
\item{xiong.bat}:双击此批处理文件进行文档的编译生成。

\item{HNUThesis.bst}:默认的~BibTeX~样式文件,如果你想修改样式,如~IEEE~要求的样式, 只需将~IEEEtran.bst~的文件名~改为~HNUThesis.bst~即可。
\end{itemize}
需要说明的是,以上文件名并不是固定的,各位同学可以新建一个~tex~文件,例如~algorithm.tex,放在~body~目录下,并且在~hnumain.tex~中调用:
\begin{Verbatim}[breaklines=true, breaksymbol=, breakanywheresymbolpre=]
    \include{body/algorithm.tex}
\end{Verbatim}
来引用之。当然你也可以重命名这些文件,只要~include~中的文件名是存在且合法,~\LaTeX~总能找到这些文件的。

在你写作某一章节的时候,你可能需要随时预览排版效果并~Debug,这时你可以在其他章节的\verb|\include|命令前加上一个\%,这代表注释掉本行,例如:
\begin{Verbatim}[breaklines=true, breaksymbol=, breakanywheresymbolpre=]
%%%%%%%%%%%%%%%%%%%%%%%%%%%%%%%%
           正文部分
%%%%%%%%%%%%%%%%%%%%%%%%%%%%%%%%
\mainmatter
% !Mode:: "TeX:UTF-8"

\chapter{模板介绍与注意事项}

\section{模板说明}

本模板是在天津大学、及山东大学软件工程学院的模板格式的基础之上进行修改之后适合海南大学本科毕业生撰写本科毕业论文而编写的~\LaTeX~论文模板。在使用本模板之前建议对~\LaTeX~进行简要的了解,熟悉其文档的结构。
由于个人水平有限,虽然现在的这个版本基本上满足了学校的要求,但难免存在不足之处,欢迎大家积极反馈,更希望山海南大学~\LaTeX~爱好者能一同完善此模板,让更多同学受益。

如有模板的疑问或有意向加入模板的维护和编写队伍中来,请给作者: 孙浩然(sunhr620@foxmail.com)写信。

\section{目录内容}
本~\LaTeX{}~模板的源文件即为本科毕业设计中使用的模板,用户可以通过修改这些文件来编辑自己的毕业设计论文。
\begin{itemize}
\item{hnumain.tex}:主文件,包含封面部分和其他章节的引用信息。
\item{preface}: 包含本科毕业设计的封面和中英文摘要。
\item{body}: 包含本文正文中的所有章节。
\begin{itemize}
\item{intros.tex}: 包括本~\LaTeX{}~模板的介绍,编译方法和使用方法。
\item{figures.tex}: 包含论文中图片的插入和引用方法。
\item{tables.tex}: 包含论文中表格的插入和引用方法。
\item{equations.tex}: 包含论文中数学符号、公式的书写和排版方法。
\item{others.tex}: 包含论文中使用的罗列环境,定理环境等其他环境的排版方法。
\item{conclusion.tex}: 包含本文的总结。
\end{itemize}
\item{setup}:存放论文所使用的宏包和全文格式的定义。
\item{appendix}:存放论文的外文资料,中文译文和致谢部分。
\item{references/reference.bib}:存放论文所引用的全部参考文献信息。
\item{xiong.bat}:双击此批处理文件进行文档的编译生成。

\item{HNUThesis.bst}:默认的~BibTeX~样式文件,如果你想修改样式,如~IEEE~要求的样式, 只需将~IEEEtran.bst~的文件名~改为~HNUThesis.bst~即可。
\end{itemize}
需要说明的是,以上文件名并不是固定的,各位同学可以新建一个~tex~文件,例如~algorithm.tex,放在~body~目录下,并且在~hnumain.tex~中调用:
\begin{Verbatim}[breaklines=true, breaksymbol=, breakanywheresymbolpre=]
    \include{body/algorithm.tex}
\end{Verbatim}
来引用之。当然你也可以重命名这些文件,只要~include~中的文件名是存在且合法,~\LaTeX~总能找到这些文件的。

在你写作某一章节的时候,你可能需要随时预览排版效果并~Debug,这时你可以在其他章节的\verb|\include|命令前加上一个\%,这代表注释掉本行,例如:
\begin{Verbatim}[breaklines=true, breaksymbol=, breakanywheresymbolpre=]
%%%%%%%%%%%%%%%%%%%%%%%%%%%%%%%%
           正文部分
%%%%%%%%%%%%%%%%%%%%%%%%%%%%%%%%
\mainmatter
\include{body/intros}
%\include{body/figures}
%\include{body/tables}
%\include{body/equations}
%\include{body/others}
%\include{body/conclusion}
\end{Verbatim}
那么,编译的时候就只编译未加~\%~的一章,在这个例子中,即本章~intros。

理论上,并不一定要把每章放在不同的文件中。但是这种自顶向下,分章节写作、编译的方法有利于提高效率,大大减少~Debug~过程中的编译时间,同时减小风险。

\section{参考文献生成和标注方法}

\LaTeX~具有插入参考文献的能力。Google Scholar~网站上存在兼容~BibTeX~的参考文献信息,通过以下几个步骤,可以轻松完成参考文献的生成。
\begin{itemize}
  \item 在\href{http://scholar.google.com/}{谷歌学术搜索}中,
        点击\href{http://scholar.google.com/scholar_preferences?hl=en&as_sdt=0,5}{学术搜索设置}。
  \item 页面打开之后,在\textbf{文献管理软件}选项中选择\textbf{显示导入~BibTeX~的链接},单击保存设置,退出。
  \item 在谷歌学术搜索中检索到文献后,在文献条目区域单击导入~BibTeX~选项,页面中出现文献的引用信息。
  \item 将文献引用信息的内容复制之后,添加到~references~文件夹下的~reference.bib~ 中。
\end{itemize}
\par{如在scholar中搜索“基于方差及方差梯度的指纹图像自适应分割算法”,那么其BibTex文件如下:
 "@article\{樊冬进2008基于方差及方差梯度的指纹图像自适应分割算法,
  title=\{基于方差及方差梯度的指纹图像自适应分割算法\},
  author=\{樊冬进 and 孙冰 and 封举富 and others\},
  year=\{2008\}
\}"
直接使用会报错,我们需要将”@article\{樊冬进2008基于方差及方差梯度的指纹图像自适应分割算法“改为@article\{anyEnglishName

}


在正文中标注参考文献时,在需要标注的地方输入~\verb|\cite{}|~指令,花括号内输入参考文献引用信息中的第一行信息即可(常常为文献的缩略信息),此时~\verb|[]|~符号在标注处的右上角显示。

\section{注意事项}

\begin{enumerate}
  \item 由于模板使用~UTF-8~编码,所以源文件应该保存成~UTF-8~格式,否则可能出现中文字符无法识别的错误。
  本模板中每一个~.tex~文件的文件的开头已经加上一行:\\
  \verb|% !Mode:: "TeX:UTF-8"|\\
     这样可以确保~.tex~文件默认使用~UTF-8~的格式打开。读者如果删去此行,很有可能会导致中文字符显示乱码。
     在~WinEdt~编辑器中可以使用以下两种方式保存成~UTF-8~格式:
      \begin{enumerate}
        \item 先建立~.tex~文件,另存为~.tex~文件时,选择用~UTF-8~ 格式保存。
        \item
            在~WinEdt~编辑器中,选择\\
            \mbox{~Document$\rightarrow$Document Settings$\rightarrow$Document Mode $\rightarrow$TeX:UTF-8} 同时在~WinEdt~最下面的状态栏中,可以看到该文档是~TeX~ 格式还是~TeX:UTF-8~格式。
            当文档为~TeX:UTF-8~格式时,状态栏一般显示:
            \makebox[\textwidth][l]{Wrap | Indent | INS | LINE |Spell | TeX:UTF-8 | -src~ 等。}
      \end{enumerate}
  \item 如果在~pdf~书签中,中文显示乱码的话,则注意以下说明:
    \begin{Verbatim}[breaklines=true, breaksymbol=, breakanywheresymbolpre=]
        \usepackage{CJKutf8}
        % 1. 如果使用CJKutf8
        %    Hyperref中应使用unicode参数
        % 2. 如果使用CJK
        %    Hyperref则使用CJKbookmarks参数
        %    可惜得到的PDF书签是乱码,建议弃用
        % 3. Unicode选项和CJKbookmarks不能同时使用
        \usepackage[
        %CJKbookmarks=true,
        unicode=true
        ]{hyperref}
     \end{Verbatim}
  \item 建议采用以下两种编译方式:
  \begin{enumerate}
    \item latex + bibtex + latex + latex + dvi2pdf. 在这种编译情况下,对应的~hnumain.tex~文件的第一行是\verb|\def\usewhat{dvipdfmx}|~ (缺省设置)。 此时,所有图片文件应该保存为~.eps~格式,如~figures~文件夹里~.eps~图片。
          如果您选择在命令行中操作,可以在编译的时候依次输入~latex hnumain, bibtex hnumain, latex hnumain, latex hnumain~和~dvipdfmx hnumain, 编译完成之后,需要手动打开~pdf~文件。需要说明的是,为了是操作简便,以上命令已经作为~pdfmake.bat~批处理文件放在目录中。在编译无误的前提下,双击此文件,可以一键生成~pdf~。

    \item pdflatex + pdflatex. 在这种编译情况下,对应的~hnumain.tex~文件的第一行应该改为\verb|\def\usewhat{pdflatex}|~。 此时, 编译不支持~.eps~图片格式,此时需要在命令行下使用~epstopdf~指令将~figures~文件夹下 的~.eps~文件转化成~.pdf~ 文件格式,命令行中操作格式为~epstopdf a.eps~。
          在命令行编译的时候,依次输入~pdflatex hnumain~和~pdflatex hnumain, 编译完成之后,需要手动打开~pdf~文件。
  \end{enumerate}
    \item  当参考文献在编辑的时候,第一行为标签行,为该文献的缩略信息。当复制中文参考文献的~BibTeX~页面到~reference.bib~文件中时,需要把原来含有中文的标签行改成英文书写,否则会报错。
\end{enumerate}

\section{系统要求}

     CTeX 2.8, MiKTeX 2.8, TeX Live 2009~或以上版本。我们推荐您使用最新的~CTeX~中文套装,~CTeX 2.9.2.164~Full~版本,内含~WinEdt 7.0~编辑器,可以完成文件的编辑和编译工作。本模板目前只在~Windows~操作系统下测试通过,尚未在~Mac~ 系统和~Linux~系统下测试。

\section{Why~\LaTeX~?}

选择使用~\TeX/\LaTeX~的理由包括:
\begin{itemize}
\item 免费软件;
\item 专业的排版效果;
\item 是事实上的专业数学排版标准;
\item 广泛的西文期刊接收甚或只接收~\LaTeX~格式的投稿;
\item[] ……
\end{itemize}
不选择使用~\TeX/\LaTeX~的理由包括:
\begin{itemize}
\item 需要相当精力学习;
\item 图文混合排版能力不够强;
\item 对于表格的支持较差;
\item 仅在数学、物理、计算机等领域流行;
\item 中文期刊的支持较差;
\item[] ……
\end{itemize}

如果想知道~\LaTeX~与~Word~的详细区别,请参见:~\url{http://zzg34b.w3.c361.com/homepage/compareWord.htm}。

\subsection{我应该看什么~\LaTeX~读物~?}

我觉得最好在使用中学会~\LaTeX~,至于参考资料百度、Google上很多。如果实在需要读物,那么可以在以下几本书上进行选择。

\begin{enumerate}
\item 我能阅读英文
\begin{enumerate}
\item 迅速入门:lshort.pdf (中文版名为:一份不太简短的~\LaTeX{}~介绍)
\item 系统学习:A Guide to LaTeX, 4th Edition, Addison-Wesley
                机械工业出版社的有影印版,名曰~《\LaTeX{}~ 实用教程》
\item 深入学习:Knuth 《TeXbook》:必读。 《LATEX Companion》:如果说 高老头的~TeXbook~是论语,那么这本
               书算是一本史记,全面而精妙,是所有~\LaTeX~书中的精品。
\end{enumerate}

\item 我更愿意阅读中文
\begin{enumerate}
\item 迅速入门:lnotes2.pdf (\LaTeX~Notes~雷太赫排版系统简介, v2.0, 包太雷)
\item 系统学习:《\LaTeXe{}~科技排版指南》,邓建松(电子版)
      如果不好找,可以阅读陈志杰等《\LaTeXe~入门与提高》第二版,或者胡伟《\LaTeXe~完全学习手册》
\item 深入学习:~TeXbook0.pdf~(特可爱原本,TeXbook~的中译,xianxian)
\item 具体问题释疑:~CTeX-FAQ.pdf~,\\
        吴凌云,~\url{http://www.ctex.org/CTeXFAQ}~\\
      ~ChinaTeXMathFAQ~$V1.1$~~China\TeX~ 数学排版常见问题集
\end{enumerate}
\end{enumerate}

遇见问题和解决问题的过程可以快速提高自己的技能,建议此时:
\begin{itemize}
 \item 清楚,扼要地提出你的问题。
 \item 使用~Google~搜索。
\end{itemize}

\section{免责声明}

本模板依据《本科生毕业论文(设计)封面》和《毕业论文(设计)排版打印格式》编写,作者希望能给使用者写作论文带来方便。然而,作者不保证本模板完全符合学校要求,也不对由此带来的风险和损失承担任何责任。

%% !Mode:: "TeX:UTF-8"

\chapter{图片的插入方法}

\section{海南大学对于插图的要求}

 \par{插图要求,所有插图按分章编号,如第1章的第1张插图为“图1-1 ****图”,字体可以用宋体5号字。所有插图均需有图注(图的说明),不能只有图编号。图号及图注应在图的下方居中标出;一幅图如有若干幅分图,均应编分图号,用(a),(b),(c)......按顺序编排;插图须紧跟文述,在正文中,一般应先见图号及图的内容后再见图(即:正文中先见“本系统功能架构如图3-2所示。”字样,再在段后见图3-2),一般情况下不能提前见图,特殊情况需延后的插图不应跨节。}
\par{图形符号及各种线型画法须按照现行的国家标准;坐标图中坐标上须注明标度值,并标明坐标轴所表示的物理量名称及量纲,应均按国际标准(SD)标注,例如:kW,  m/s, N, m....等,但对一些示意图例外;图应具有“自明性”(即只看图、图题和图例,不阅读正文,就可理解图意);图中用字最小为宋体小五号字;插图必须同内容密切联系,切忌与文字和表重复。}
\par{图的绘制一定要紧凑美观并且保证清晰,避免使用彩色和带底色的图,以免影响论文印刷。}
\par{切忌直接从他人文章、其他文献、书籍扫描和网上直接拷贝图形,全文不能出现非本人绘制的图形(系统实现部分的系统界面截图除外),必须使用他人插图时,须在图题正下方注明出处。
}

\section{\LaTeX~中推荐使用的图片格式}

在~\LaTeX~中应用最多的图片格式是~EPS(Encapsulated PostScript)格式,它是一种专用的打印机描述语言,常用于印刷或打印输出。
EPS~格式图片可通过多种方式生成在这里可以采用Adobe Acrobat 软件进行转换。选中图片右击转换为pdf然后,选择保存,在保存格式下面选择文件的格式为eps即可。,对于用visio写的流程图,其转换格式和普通图片转换成eps格式完全相同。

\section{单张图片的插入方法}
单张图片独自占一行的插入形式如图~\ref{fig:xml}~所示。
\begin{figure}[htbp]
\centering
\includegraphics[width=0.4\textwidth]{XML}
\caption{树状结构}\label{fig:xml}
\vspace{\baselineskip}
\end{figure}


其插入图片的代码及其说明如下。
\vspace{1em}\noindent\hrule
\begin{Verbatim}[breaklines=true, breaksymbol=, breakanywheresymbolpre=]
\begin{figure}[htbp]
\centering
\includegraphics[width=0.4\textwidth]{文件名(.eps)}
\caption{标题}\label{标签名(通常为 fig:labelname)}
\vspace{\baselineskip} % 表示图与正文空一行
\end{figure}
\end{Verbatim}

\noindent\hrule

\begin{Verbatim}[breaklines=true, breaksymbol=, breakanywheresymbolpre=]
figure环境的可选参数[htbp]表示浮动图形所放置的位置,h (here)表示当前位置,t (top)表示页芯顶部,b (bottom)表示页芯底部,p (page)表示单独一页。在Word等软件中,图片通常插入到当前位置,如果当前页的剩余空间不够,图片将被移动到下一页,当前页就会出现很大的空白,其人工调整工作非常不便。由LaTeX提供的浮动图片功能,总是会按h->t->b->p的次序处理选项中的字母,自动调整图片的位置,大大减轻了工作量。
\centering命令将后续内容转换成每行皆居中的格式。
"\includegraphics"的可选参数用来设置图片插入文中的水平宽度,一般表示为正文宽度(\textwidth)的倍数。
\caption命令可选参数“标签名”为英文形式,一般不以图片或表格的数字顺序作为标签,而应包含一定的图片或表格信息,以便于文中引用(若图片、表格、公式、章节和参考文献等在文中出现的先后顺序发生了变化,其标注序号及其文中引用序号也会跟着发生变化,这一点是Word等软件所不能做到的)。另外,图题或表题并不会因为分页而与图片或表格体分置于两页,章节等各级标题也不会置于某页的最底部,LaTeX系统会自动调整它们在正文中的位置,这也是Word等软件所无法匹敌的。
\vspace将产生一定高度的竖直空白,必选参数为负值表示将后续文字位置向上提升,参数值可自行调整。em为长度单位,相当于大写字母M的宽度。\vspace{\baselineskip} 表示图与正文空一行。
引用方法:“见图~\ref{fig:figname}”、“如图~\ref{fig:figname}~所示”等。
\end{Verbatim}

\noindent\hrule\vspace{1em}

若需要将~2~张及以上的图片并排插入到一行中,则需要采用\verb|minipage|环境,如图~\ref{fig:dd}~和图~\ref{fig:ds}~所示。
\begin{figure}[htbp]
\centering
\begin{minipage}{0.4\textwidth}
\centering
\includegraphics[width=\textwidth]{dataDimensions}
\caption{数据维数的变化}\label{fig:dd}
\end{minipage}
\begin{minipage}{0.4\textwidth}
\centering
\includegraphics[width=\textwidth]{dataSize}
\caption{数据规模的变化}\label{fig:ds}
\end{minipage}
\vspace{\baselineskip}
\end{figure}

其代码如下所示。
\vspace{1em}\noindent\hrule
\begin{Verbatim}[breaklines=true, breaksymbol=, breakanywheresymbolpre=]
\begin{figure}[htbp]
\centering
\begin{minipage}{0.4\textwidth}
\centering
\includegraphics[width=\textwidth]{文件名}
\caption{标题}\label{fig:f1}
\end{minipage}
\begin{minipage}{0.4\textwidth}
\centering
\includegraphics[width=\textwidth]{文件名}
\caption{标题}\label{fig:f2}
\end{minipage}\vspace{\baselineskip}
\end{figure}
\end{Verbatim}

\noindent\hrule

\begin{Verbatim}[breaklines=true, breaksymbol=, breakanywheresymbolpre=]
minipage环境的必选参数用来设置小页的宽度,若需要在一行中插入n个等宽图片,则每个小页的宽度应略小于(1/n)\textwidth。
\end{Verbatim}

\noindent\hrule

\section{具有子图的图片插入方法}

图中若含有子图时,需要调用~subfigure~宏包, 如图~\ref{fig:subfig}~所示。
\begin{figure}[htbp]
  \centering
  \subfigure[Data Dimensions]{\label{fig:subfig:datadim}
                \includegraphics[width=0.4\textwidth]{dataDimensions}}
  \subfigure[Data Size]{\label{fig:subfig:datasize}
                \includegraphics[width=0.4\textwidth]{dataSize}}
  \caption{Scalability of data}\label{fig:subfig}
\vspace{\baselineskip}
\end{figure}

其代码及其说明如下。
\vspace{1em}\noindent\hrule

\begin{Verbatim}[breaklines=true, breaksymbol=, breakanywheresymbolpre=]
\begin{figure}[htbp]
  \centering
  \subfigure[第1个子图标题]{
            \label{第1个子图标签(通常为 fig:subfig1:subsubfig1)}
            \includegraphics[width=0.4\textwidth]{文件名}}
  \subfigure[第2个子图标题]{
            \label{第2个子图标签(通常为 fig:subfig1:subsubfig2)}
            \includegraphics[width=0.4\textwidth]{文件名}}
  \caption{总标题}\label{总标签(通常为 fig:subfig1)}
\vspace{\baselineskip}
\end{figure}
\end{Verbatim}

\noindent\hrule

\begin{Verbatim}[breaklines=true, breaksymbol=, breakanywheresymbolpre=]
子图的标签实际上可以随意设定,只要不重复就行。但为了更好的可读性,我们建议fig:subfig:subsubfig格式命名,这样我们从标签名就可以知道这是一个子图引用。
引用方法:总图的引用方法同本章第1节,子图的引用方法用\ref{fig:subfig:subsubfig}来代替。
\end{Verbatim}

\noindent\hrule\vspace{1em}

子图的引用示例:如图~\ref{fig:subfig:datadim}~和图~\ref{fig:subfig:datasize}~所示。

若想获得插图方法的更多信息,参见网络上的~\href{ftp://ftp.tex.ac.uk/tex-archive/info/epslatex.pdf}{Using Imported Graphics in \LaTeX and pdf\LaTeX}~文档。

%% !Mode:: "TeX:UTF-8"

\chapter{表格的绘制方法}

\section{本科生毕业论文的绘表规范}

表中内容应与叙述文字内容相呼应,表的结构应简洁明了,表随文排,字体为宋体5号字注,一定使用单倍行距,要排版紧凑,以与正文内容区别。对于文中的各类表,一定注意先有引用文字(格式一般为“如表8-5所示”)然后见到表,插表表名要简明贴切,表序按章用阿拉伯字编列,表序末和表名末均不加标点符号,写在表的上方。表格较长如需转页,则在下页稿纸上重写表头,并在表的右上方写“续表”,表内全部数据的统一计数或计量单位应置于表的右上角,若表中各栏计量单位不同,则将单位分别列入表头的各栏中,将量的符号与单位符号之间用斜线隔开,即表中的数值用量与单位的比值形式表示。表内数据对应位上下对齐,一般以小数点为准;数字间夹有“—”,“/”号者,以这些符号对齐;无数据或文字处一律空白。相邻栏内数字相同时,应重复书写,勿用“同左”、“同上”等;表内文字说明,空一格起行,转行顶格,并正确使用标点符号,但每段最后一律不用标点符号。表内名词短语、数据需注释时,用脚注,即在所需加注名词或数据的右上角注符号“①,②,…”或星号,在表的底线下方写出相应的符号和注文,不出现“注”字,如对整个表加以说明时,可附注于底线下方,注文前应有“说明:”字样。

\section{普通表格的绘制方法}

表格应具有三线表格式,因此需要调用~booktabs~宏包,其标准格式如表~\ref{tab:table1}~所示。
\begin{table}[htbp]
\caption{符合本科生毕业论文绘图规范的表格}\label{tab:table1}
\vspace{0.5em}\centering\wuhao
\begin{tabular}{ccccc}
\toprule[1.5pt]
$D$(in) & $P_u$(lbs) & $u_u$(in) & $\beta$ & $G_f$(psi.in)\\
\midrule[1pt]
 5 & 269.8 & 0.000674 & 1.79 & 0.04089\\
10 & 421.0 & 0.001035 & 3.59 & 0.04089\\
20 & 640.2 & 0.001565 & 7.18 & 0.04089\\
 5 & 269.8 & 0.000674 & 1.79 & 0.04089\\
10 & 421.0 & 0.001035 & 3.59 & 0.04089\\
20 & 640.2 & 0.001565 & 7.18 & 0.04089\\
 5 & 269.8 & 0.000674 & 1.79 & 0.04089\\
10 & 421.0 & 0.001035 & 3.59 & 0.04089\\
20 & 640.2 & 0.001565 & 7.18 & 0.04089\\
 5 & 269.8 & 0.000674 & 1.79 & 0.04089\\
10 & 421.0 & 0.001035 & 3.59 & 0.04089\\
20 & 640.2 & 0.001565 & 7.18 & 0.04089\\
\bottomrule[1.5pt]
\end{tabular}
\vspace{\baselineskip}
\end{table}

其绘制表格的代码及其说明如下。
\vspace{1em}\noindent\hrule

\begin{VerbWithBreak}
\begin{table}[htbp]
\caption{表标题}\label{标签名(通常为 tab:tablename)}
\vspace{0.5em}\centering\wuhao
\begin{tabular}{cc...c}
\toprule[1.5pt]
表头第1个格   & 表头第2个格   & ... & 表头第n个格  \\
\midrule[1pt]
表中数据(1,1) & 表中数据(1,2) & ... & 表中数据(1,n)\\
表中数据(2,1) & 表中数据(2,2) & ... & 表中数据(2,n)\\
表中数据(3,1) & 表中数据(3,2) & ... & 表中数据(3,n)\\
表中数据(4,1) & 表中数据(4,2) & ... & 表中数据(4,n)\\
...................................................\\
表中数据(m,1) & 表中数据(m,2) & ... & 表中数据(m,n)\\
\bottomrule[1.5pt]
\end{tabular}
\vspace{\baselineskip}
\end{table}
\end{VerbWithBreak}

\noindent\hrule

\begin{VerbWithBreak}
table环境是一个将表格嵌入文本的浮动环境。
\wuhao命令将表格的字号设置为五号字(10.5pt),在绘制表格结束退出时,不需要将字号再改回为\xiaosi,正文字号默认为小四号字(12pt)。
tabular环境的必选参数由每列对应一个格式字符所组成:c表示居中,l表示左对齐,r表示右对齐,其总个数应与表的列数相同。此外,@{文本}可以出现在任意两个上述的列格式之间,其中的文本将被插入每一行的同一位置。表格的各行以\\分隔,同一行的各列则以&分隔。
\toprule、\midrule和\bottomrule三个命令是由booktabs宏包提供的,其中\toprule和\bottomrule分别用来绘制表格的第一条(表格最顶部)和第三条(表格最底部)水平线,\midrule用来绘制第二条(表头之下)水平线,且第一条和第三条水平线的线宽为1.5pt,第二条水平线的线宽为1pt。
引用方法:“如表~\ref{tab:tablename}~所示”。
\end{VerbWithBreak}

\noindent\hrule

\section{长表格的绘制方法}

长表格是当表格在当前页排不下而需要转页接排的情况下所采用的一种表格环境。若长表格仍按照普通表格的绘制方法来获得,
其所使用的~\verb|table|~浮动环境无法实现表格的换页接排功能,表格下方过长部分会排在表格第~1~页的页脚以下。为了能够实现长表格的转页接排功能,
需要调用~\verb|longtable|~宏包,由于长表格是跨页的文本内容,因此只需要单独的~\verb|longtable|~环境,所绘制的长表格的格式如表~\ref{tab:table2}~所示。

此长表格~\ref{tab:table2}~第~2~页的标题“编号(续表)”和表头是通过代码自动添加上去的,无需人工添加,若表格在页面中的竖直位置发生了变化,长表格在第~2~页
及之后各页的标题和表头位置能够始终处于各页的最顶部,也无需人工调整,\LaTeX~系统的这一优点是~Word~等软件所无法企及的。

下段内容是为了让下面的长表格分居两页,看到表标题“编号(续表)”的效果。此模板的完成时间正值雨后初霁的四月二十五日,故引用林徽因《你是人间的四月天》全文:
\begin{center}
\begin{minipage}[c]{0.5\textwidth}

\textbf{你是人间的四月天}

\vspace{12pt}
我说你是人间的四月天\\
笑音点亮了四面风\\
轻灵在春的光艳中交舞着变\\
你是四月早天里的云烟\\
黄昏吹着风的软\\
星子在无意中闪\\
细雨点洒在花前\\
那轻~~那娉婷\\
你是鲜妍\\
百花的冠冕你戴着\\
你是天真~~庄严~~你是夜夜的月圆\\
雪化后那片鹅黄\\
你像~~新鲜初放芽的绿\\
你是柔嫩喜悦\\
水光浮动着你梦期待中白莲\\
你是一树一树的花开\\
是燕~~在梁间呢喃\\
你是爱~~是暖\\
是希望\\
你是人间的四月天
\end{minipage}
\end{center}

					
\wuhao\begin{longtable}{ccc}
\caption{海南大学各学院名称一览}\label{tab:table2}
 \vspace{0.5em}\\
\toprule[1.5pt] 学院名称 & 网址 & 邮政编码  \\ \midrule[1pt]
\endfirsthead
\multicolumn{3}{c}{表~\thetable(续表)}\vspace{0.5em}\\
\toprule[1.5pt] 学院名称 & 网址 & 邮政编码  \\ \midrule[1pt]
\endhead
\bottomrule[1.5pt]
\endfoot
海南大学& \url{http://www.hainu.edu.cn/}& 570228\\
机电工程学院&  \url{http://www.hainu.edu.cn/jidian/}& 570228\\
\end{longtable}\xiaosi
\vspace{\baselineskip}

绘制长表格的代码及其说明如下。
\vspace{1em}\noindent\hrule

\begin{VerbWithBreak}
\wuhao\begin{longtable}{cc...c}
\caption{表标题}\label{标签名(通常为 tab:tablename)}\\
\toprule[1.5pt] 表头第1个格 & 表头第2个格 & ... & 表头第n个格\\ \midrule[1pt]
\endfirsthead
\multicolumn{n}{c}{表~\thetable(续表)}\vspace{0.5em}\\
\toprule[1.5pt] 表头第1个格 & 表头第2个格 & ... & 表头第n个格\\ \midrule[1pt]
\endhead
\bottomrule[1.5pt]
\endfoot
表中数据(1,1) & 表中数据(1,2) & ... & 表中数据(1,n)\\
表中数据(2,1) & 表中数据(2,2) & ... & 表中数据(2,n)\\
...................................................\\
表中数据(m,1) & 表中数据(m,2) & ... & 表中数据(m,n)\\
\end{longtable}\xiaosi
\end{VerbWithBreak}

\noindent\hrule
\begin{VerbWithBreak}
在绘制长表格的前面留出一个空白行,并在第2行的一开始全局定义长表格的字号为五号字,这样能够保证长表格之前段落的行距保持不变。
在绘制长表格结束后,需要\xiaosi命令重新将字号改为小四号字。
\endhead之前的文字描述的是第2页及其之后各页的标题或表头;
\endfirsthead之前的文字描述的是第1页的标题和表头,若无此命令,则第1页的表头和标题由\endhead命令确定;
同理,\endfoot之前的文字描述的是除最后一页之外每页的表格底部内容;
\endlastfoot之前的文字描述的是最后一页的表格底部内容,若无此命令,
则最后一页的表格底部内容由\endfoot命令确定;由于规范中长表格每页底部内容均相同(水平粗线),因此模板中没有用到\endlastfoot命令。
\end{VerbWithBreak}

\noindent\hrule
\section{列宽可调表格的绘制方法}
论文中能用到列宽可调表格的情况共有两种:一种是当插入的表格某一单元格内容过长以至于一行放不下的情况,
另一种是当对公式中首次出现的物理量符号进行注释的情况。这两种情况都需要调用~tabularx~宏包。下面将分别对这两种情况下可调表格的绘制方法进行阐述。
\subsection{表格内某单元格内容过长的情况}

首先给出这种情况下的一个例子如表~\ref{tab:table3}~所示。
\begin{table}[htbp]
\caption{最小的三个正整数的英文表示法}\label{tab:table3}
\vspace{0.5em}\wuhao
\begin{tabularx}{\textwidth}{llX}
\toprule[1.5pt]
Value & Name & Alternate names, and names for sets of the given size\\\midrule[1pt]
1 & One & ace, single, singleton, unary, unit, unity\\
2 & Two & binary, brace, couple, couplet, distich, deuce, double, doubleton, duad, duality, duet, duo, dyad, pair, snake eyes, span, twain, twosome, yoke\\
3 & Three & deuce-ace, leash, set, tercet, ternary, ternion, terzetto, threesome, tierce, trey, triad, trine, trinity, trio, triplet, troika, hat-trick\\\bottomrule[1.5pt]
\end{tabularx}
\vspace{\baselineskip}
\end{table}
绘制这种表格的代码及其说明如下。
\vspace{1em}\noindent\hrule
\begin{VerbWithBreak}
\begin{table}[htbp]
\caption{表标题}\label{标签名(通常为 tab:tablename)}
\vspace{0.5em}\wuhao
\begin{tabularx}{\textwidth}{l...X...l}
\toprule[1.5pt]
表头第1个格   & ... & 表头第X个格   & ... & 表头第n个格  \\
\midrule[1pt]
表中数据(1,1) & ... & 表中数据(1,X) & ... & 表中数据(1,n)\\
表中数据(2,1) & ... & 表中数据(2,X) & ... & 表中数据(2,n)\\
.........................................................\\
表中数据(m,1) & ... & 表中数据(m,X) & ... & 表中数据(m,n)\\
\bottomrule[1.5pt]
\end{tabularx}
\vspace{\baselineskip}
\end{table}
\end{VerbWithBreak}

\noindent\hrule
\begin{VerbWithBreak}
tabularx环境共有两个必选参数:第1个参数用来确定表格的总宽度,这里取为排版表格能达到的最大宽度——正文宽度\textwidth;第2 个参数用来确定每列格式,其中标为X的项表示该列的宽度可调,其宽度值由表格总宽度确定。
标为X的列一般选为单元格内容过长而无法置于一行的列,这样使得该列内容能够根据表格总宽度自动分行。若列格式中存在不止一个X 项,则这些标为X的列的列宽相同,因此,一般不将内容较短的列设为X。
标为X的列均为左对齐,因此其余列一般选为l(左对齐),这样可使得表格美观,但也可以选为c或r。如果想要X项为中心对齐,可以在后面加<{\centering},形同\begin{tabularx}{\textwidth}{cX<{\centering}X<{\centering}}
\end{VerbWithBreak}

\noindent\hrule
\subsection{对物理量符号进行注释的情况}
为使得对公式中物理量符号注释的转行与破折号“———”后第一个字对齐,此处最好采用表格环境。此表格无任何线条,左对齐,
且在破折号处对齐,一共有“式中”二字、物理量符号和注释三列,表格的总宽度可选为文本宽度,因此应该采用\verb|tabularx|环境。
由\verb|tabularx|环境生成的对公式中物理量符号进行注释的公式如式(\ref{eq:1})所示。
%\vspace*{10pt}

\begin{equation}\label{eq:1}
\ddot{\boldsymbol{\rho}}-\frac{\mu}{R_{t}^{3}}\left(3\mathbf{R_{t}}\frac{\mathbf{R_{t}\rho}}{R_{t}^{2}}-\boldsymbol{\rho}\right)=\mathbf{a}
\end{equation}

\begin{tabularx}{\textwidth}{@{}l@{\quad}r@{———}X@{}}
式中& $\bm{\rho}$ &追踪飞行器与目标飞行器之间的相对位置矢量;\\
&  $\bm{\ddot{\rho}}$&追踪飞行器与目标飞行器之间的相对加速度;\\
&  $\mathbf{a}$   &推力所产生的加速度;\\
&  $\mathbf{R_t}$ & 目标飞行器在惯性坐标系中的位置矢量;\\
&  $\omega_{t}$ & 目标飞行器的轨道角速度;\\
&  $\mathbf{g}$ & 重力加速度,$=\frac{\mu}{R_{t}^{3}}\left(
3\mathbf{R_{t}}\frac{\mathbf{R_{t}\rho}}{R_{t}^{2}}-\bm{\rho}\right)=\omega_{t}^{2}\frac{R_{t}}{p}\left(
3\mathbf{R_{t}}\frac{\mathbf{R_{t}\rho}}{R_{t}^{2}}-\bm{\rho}\right)$,这里~$p$~是目标飞行器的轨道半通径。
\end{tabularx}
\vspace{\wordsep}

其中生成注释部分的代码及其说明如下。

\vspace{1em}\noindent\hrule

\begin{VerbWithBreak}
\begin{tabularx}{\textwidth}{@{}l@{\quad}r@{— — —}X@{}}
式中 & symbol-1 & symbol-1的注释内容;\\
     & symbol-2 & symbol-2的注释内容;\\
     .............................;\\
     & symbol-m & symbol-m的注释内容。
\end{tabularx}\vspace{\wordsep}
\end{VerbWithBreak}

\noindent\hrule

\begin{VerbWithBreak}
   tabularx环境的第1个参数选为正文宽度,第2个参数里面各个符号的意义为:
   第1个@{}表示在“式中”二字左侧不插入任何文本,“式中”二字能够在正文中左对齐,若无此项,则“式中”二字左侧会留出一定的空白;
   @{\quad}表示在“式中”和物理量符号间插入一个空铅宽度的空白;
   @{— — —}实现插入破折号的功能,它由三个1/2的中文破折号构成;
   第2个@{}表示在注释内容靠近正文右边界的地方能够实现右对齐。
\end{VerbWithBreak}

\noindent\hrule\vspace{1em}

由此方法生成的注释内容应紧邻待注释公式并置于其下方,因此不能将代码放入~\verb|table|~浮动环境中。但此方法不能实现自动转页接排,
可能会在当前页剩余空间不够时,全部移动到下一页而导致当前页出现很大空白。因此在需要转页处理时,还请您手动将需要转页的代码放入一个
新的~\verb|tabularx|~环境中,将原来的一个~\verb|tabularx|~环境拆分为两个~\verb|tabularx|~环境。

\section{为表格添加注释的方法}
首先给出这种情况下的一个例子如表~\ref{tab:table4}~所示。
\begin{table}[!ht]
   \caption{海南省相关数据}\label{tab:table4}
   \vspace{0.5em}\centering\wuhao
   \begin{threeparttable}
   \begin{tabularx}{\textwidth}{c*5{X<{\centering}}}
   \toprule[1.5pt]
   年份/年 & 农业机械总动力/千瓦 & 农村耗电量/亿千瓦小时 & 农田有效灌溉面积/千公顷 & 互联网络宽带接入用户/万户 & 长途光缆线路长度/万公里\\
   \midrule[1pt]
   2011年 & 444.33 & 7.07 & 247.51 & 16.2 & 0.33\\
   2012年 & 479.66 & 8.59 & 286.75 & 21.4 & 0.33\\
   2013年 & 502.1 & 9.63 & 260.93 & 26 & 0.33\\
   2014年 & 517.31 & 10.89 & 259.92 & 29.7 & 0.33\\
   2015年 & 511.59 & 13.02 & 263.99 & 42.2 & 0.34\\
   2016年 & 516.57 & 13.89 & 289.95 & 60.2 & 0.34\\
   2017年 & 569.8 & 15.52 & 289.25 & 79.6 & 0.34\\
   2018年 & 565.82 & 17.41 & 290.48 & 82.6 & 0.09\\
   2019年 & 581.23 & 18.68 & 290.63 & 99.7 & 0.32 \\
   2020年 & 615.57 & 20.96 & 292.21 & 111.2 & 0.32 \\
   \bottomrule[1.5pt]
   \end{tabularx}
   \begin{tablenotes}
       \footnotesize
       \item[1] 数据来源:国家统计局
   \end{tablenotes}
   \end{threeparttable}
   \vspace{\baselineskip}
\end{table}

绘制这种表格的代码及其说明如下。
\vspace{1em}\noindent\hrule
\begin{VerbWithBreak}
   \begin{table}[!ht]
      \caption{海南省相关数据}
      \vspace{0.5em}\centering\wuhao
      \begin{threeparttable}
      \begin{tabularx}{\textwidth}{c*5{X<{\centering}}}
      \toprule[1.5pt]
      年份/年 & 农业机械总动力/千瓦 & 农村耗电量/亿千瓦小时 & 农田有效灌溉面积/千公顷 & 互联网络宽带接入用户/万户 & 长途光缆线路长度/万公里\\
      \midrule[1pt]
      2011年 & 444.33 & 7.07 & 247.51 & 16.2 & 0.33\\
      2012年 & 479.66 & 8.59 & 286.75 & 21.4 & 0.33\\
      2013年 & 502.1 & 9.63 & 260.93 & 26 & 0.33\\
      2014年 & 517.31 & 10.89 & 259.92 & 29.7 & 0.33\\
      2015年 & 511.59 & 13.02 & 263.99 & 42.2 & 0.34\\
      2016年 & 516.57 & 13.89 & 289.95 & 60.2 & 0.34\\
      2017年 & 569.8 & 15.52 & 289.25 & 79.6 & 0.34\\
      2018年 & 565.82 & 17.41 & 290.48 & 82.6 & 0.09\\
      2019年 & 581.23 & 18.68 & 290.63 & 99.7 & 0.32 \\
      2020年 & 615.57 & 20.96 & 292.21 & 111.2 & 0.32 \\
      \bottomrule[1.5pt]
      \end{tabularx}
      \begin{tablenotes}
          \footnotesize
          \item[1] 数据来源:国家统计局
      \end{tablenotes}
      \end{threeparttable}
      \vspace{\baselineskip}
   \end{table}
\end{VerbWithBreak}
   \noindent\hrule
\begin{VerbWithBreak}
   这里使用了threeparttable宏包。在tablenotes环境中添加注释即可。
\end{VerbWithBreak}

若想获得绘制表格的更多信息,参见网络上的~\href{http://www.tug.org/pracjourn/2007-1/mori/}{Tables in \LaTeXe: Packages and Methods}~文档。


%% !Mode:: "TeX:UTF-8"

\chapter{数学公式的输入方法}

\section{本科生毕业论文的公式规范}

公式均需有公式号;公式号按章编排,如式(2-3);公式中各物理量及量纲均按国际标准(SI)及国家规定的法定符号和法定计量单位标注,禁止使用已废弃的符号和计量单位;公式中用字、符号、字体要符合学科规范。图、表、公式等与正文之间要有6磅的行间距。 

公式中用斜线表示“除”的关系时应采用括号,以免含糊不清,如~$a/(b\cos x)$。通常“乘”的关系在前,如~$a\cos x/b$而不写成~$(a/b)\cos x$。

不能用文字形式表示等式,如:$\textnormal{刚度}=\frac{{\textnormal{受力}}}{{\textnormal{受力方向的位移}}}$。

对于数学公式的输入方法,网络上有一个比较全面权威的文档\textbf{~\href{http://tug.ctan.org/cgi-bin/ctanPackageInformation.py?id=voss-mathmode}{Math mode}}~请大家事先大概浏览一下。下面将对学位论文中主要用到的数学公式排版形式进行阐述。

\section{生成~\LaTeX~数学公式的两种方法}

对于先前没有接触过~\LaTeX~的人来说,编写~\LaTeX~数学公式是一件很繁琐的事,尤其是对复杂的数学公式来说,更可以说是一件难以完成的任务。
实际上,生成~\LaTeX~数学公式有两种较为简便的方法,一种是基于~MathType~ 数学公式编辑器的方法,另一种是基于~MATLAB~商业数学软件的方法,
下面将分别对这两种数学公式的生成方法作一下简单介绍。

\subsection{基于~MathType~软件的数学公式生成方法}

MathType~是一款功能强大的数学公式编辑器软件,能够用来在文本环境中插入~Windows OLE~ 图形格式的复杂数学公式,所以应用比较普遍。但此软件只有~30~天的试用期,之后若再继续使用则需要付费购买才行。网络上有很多破解版的~MathType~软件可供下载免费使用,
我们推荐下载安装版本号在~6.5~之上的破解版。

以~v6.8~版本(英文版)为例~在安装好~MathType~之后,若在输入窗口中编写数学公式,复制到剪贴板上的仍然是图形格式的对象。
若希望得到可插入到~\LaTeX~编辑器中的文本格式对象,则需要对~MathType~软件做一下简单的设置:在~MathType~最上排的按钮中依次选择“Preferences
$\rightarrow$ Cut and Copy Preferences”,在弹出的对话窗中选中“MathML or TeX:”,在转换下拉框中选择“LaTeX~2.09~and later”,并将对话框最下方的两个复选框全部勾掉,点击确定。这样,再从输入窗口中复制出来的对象就是文本格式,这样就可以直接将其粘贴到~\LaTeX~
编辑器中了。按照这种方法生成的数学公式两端分别有标记~\verb|\[|~和标记~\verb|\]|~,在这两个标记之间即是真正的数学公式代码。

若希望从~MathType~输入窗口中复制出来的对象为图形格式,则只需在单击“Preferences
$\rightarrow$ Cut and Copy Preferences”弹出后的窗口中,再选中“Equation object(Windows OLE graphic)” 即可。

\subsection{基于~MATLAB~软件的数学公式生成方法}


MATLAB~是矩阵实验室(Matrix Laboratory)的简称,是美国~MathWorks~公司出品的商业数学软件。它是当今科研领域最常用的应用软件之一,
具有强大的矩阵计算、符号运算和数据可视化功能,是一种简单易用、可扩展的系统开发环境和平台。


MATLAB~中提供了一个~latex~函数,它可将符号表达式转化为~\LaTeX~数学公式的形式。其语法形式为~latex(s),其中,~s~为符号表达式,
之后再将~latex~函数的运算结果直接粘贴到~\LaTeX~编辑器中。从~\LaTeX~数学公式中可以发现,其中可能包含如下符号组合:

\begin{verbatim*}
\qquad=两个空铅(quad)宽度
\quad=一个空铅宽度
\;=5/18空铅宽度
\:=4/18空铅宽度
\,=3/18空铅宽度
\!=-3/18空铅宽度
\ =一个空格
\end{verbatim*}

所以最好将上述符号组合从数学公式中删除,从而使数学公式显得匀称美观。对于~Word~等软件的使用者来说,在我们通过~MATLAB~运算得到符号表达式形式的运算结果时,在~Word~中插入运算结果需要借助于~MathType~软件,通过在~MathType~中输入和~MATLAB~运算结果相对应的数学表达形式,之后再将~MathType~数学表达式转换为图形格式粘贴到~Word~ 中。实际上,也可以将~MATLAB~中采用~latex~函数运行的结果直接粘贴到~MathType~中,再继续上述步骤,这样可以大大节省输入公式所需要的时间。此方法在~MathType~6.5c~上验证通过,若您粘入到~MathType~中的仍然为从~MATLAB~中导入的代码,请您更新~MathType~软件。

\section{数学字体}
在数学模式下,常用的数学字体命令有如下几种:

\begin{Verbatim}[breaklines=true, breaksymbol=, breakanywheresymbolpre=]
\mathnormal或无命令 用数学字体打印文本;
\mathit             用斜体(\itshape)打印文本;
\mathbf             用粗体(\bfseries)打印文本;
\mathrm             用罗马体(\rmfamily)打印文本;
\mathsf             用无衬线字体(\sffamily)打印文本;
\mathtt             用打印机字体(\ttfamily)打印文本;
\mathcal            用书写体打印文本;
\end{Verbatim}

在学位论文撰写中,只需要用到上面提到的~\verb|\mathit|、\verb|\mathbf|~ 和~\verb|\mathrm|~命令。若要得到~Times New Roman~ 的数学字体,则需要调用~txfonts~宏包(此宏包实际上采用的是~Nimbus Roman No9 L~字体,
它是开源系统中使用的免费字体,其字符字体与~Times New Roman~字体几乎完全相同);若要得到粗体数学字体,则需要调用~bm~宏包。表~\ref{tab:fonts}~中分别列出了得到阿拉伯数字、拉丁字母和希腊字母
各种数学字体的命令。

\begin{table}[htbp]
\caption{常用数学字体命令一览}\label{tab:fonts}
\vspace{0.5em}\centering\wuhao
\begin{tabular}{llll}
\toprule
 & 阿拉伯数字\&大写希腊字母 & 大小写拉丁字母 & 小写希腊字母  \\
\midrule
斜体 & \verb|\mathit{}| & \verb|无命令| & \verb|无命令|\\
粗斜体 & \verb|\bm{\mathit{}}| & \verb|\bm{}| & \verb|\bm{}|\\
直立体 & \verb|无命令| & \verb|\mathrm{}| & \verb|字母后加up|\\
粗体 & \verb|\mathbf{}或\bm{}| & \verb|\mathbf{}| & \verb|\bm{字母后加up}|\\
\bottomrule
\end{tabular}
\vspace{\baselineskip}
\end{table}

\noindent 下面列出了一些应采用直立数学字体的数学常数和数学符号。

\vspace{-0.5em}\begin{center}\begin{tabularx}{0.7\textwidth}{XX}
$\mathrm{d}$、 $\mathrm{D}$、 $\mathrm{p}$~———微分算子 & $\mathrm{e}$~———自然对数之底数\\
$\mathrm{i}$、 $\mathrm{j}$~———虚数单位 & $\piup$———圆周率\\
\end{tabularx}\end{center}

\section{行内公式}
出现在正文一行之内的公式称为行内公式,例如~$f(x)=\int_{a}^{b}\frac{\sin{x}}{x}\mathrm{d}x$。对于非矩阵和非多行形式的行内公式,一般不会使得行距发生变化,而~Word~等软件却会根据行内公式的竖直距离而自动调节行距,如图~\ref{fig:hangju}~所示。

\begin{figure}[htbp]
\centering
\subfigure[由~\LaTeX~系统生成的行内公式]{\label{fig:subfig:latex}
                \fbox{\includegraphics[width=0.55\textwidth]{latex}}}
\subfigure[由~Word软件生成的~.doc~格式行内公式]{\label{fig:subfig:word}
                \fbox{\includegraphics[width=0.55\textwidth]{word}}}
\subfigure[由~Word软件生成的~.pdf~格式行内公式]{\label{fig:subfig:pdf}
                \fbox{\includegraphics[width=0.55\textwidth]{pdf}}}

\caption{由~\LaTeX~和~Word~生成的~3~种行内公式屏显效果}\label{fig:hangju}
\vspace{-1em}
\end{figure}

这三幅图分别为~\LaTeX~和~Word~生成的行内公式屏显效果,从图中可看出,在~\LaTeX~文本含有公式的行内,在正文与公式之间对接工整,行距不变;而在~Word~文本含有公式的行内,在正文与公式之间对接不齐,行距变大。因此从这一点来说,
\LaTeX~系统在数学公式的排版上具有很大优势。

\LaTeX~提供的行内公式最简单、最有效的方法是采用~\TeX~本来的标记———开始和结束标记都写作~\$,例如本段开始的例子可由下面的输入得到。
\verb|$f(x)=\int_{a}^{b}\frac{\sin{x}}{x}\mathrm{d}x$|

\section{行间公式}
位于两行之间的公式称为行间公式,每个公式都是一个单独的段落,例如
\[\int_a^b{f\left(x\right)\mathrm{d}x}=\lim_{\left\|\Delta{x_i}\right\|\to 0}\sum_i{f\left(\xi_i\right)\Delta{x_i}}\]
除人工编号外,\LaTeX~各种类型行间公式的标记见表~\ref{tab:eqtag}。
\begin{table}[htbp]
\caption{各种类型行间公式的标记}\label{tab:eqtag}
\vspace{0.5em}\centering\wuhao
\begin{tabularx}{\textwidth}{cll}
\toprule
& 无编号 & 自动编号\\
\midrule
单行公式& \verb|\begin{displaymath}... \end{displaymath}|& \verb|\begin{equation}... \end{equation}|\\
        & 或~\verb|\[...\]| & \\
多行公式& \verb|\begin{eqnarray*}... \end{eqnarray*}|& \verb|\begin{eqnarray}... \end{eqnarray}|\\
\bottomrule
\end{tabularx}
\end{table}

另外,在自动编号的某行公式行尾添加标签~\verb|\nonumber|,可将该行转换为无编号形式。

行间多行公式需采用~\verb|eqnarray|~或~\verb|eqnarray*|~环境,它默认是一个列格式为~\verb|rcl|~的~3~列矩阵,并且中间列的字号要小一些,因此通常只将需要对齐的运算符号(通常为等号“=”)置于中间列。

\section{可自动调整大小的定界符}

若在左右两个定界符之前分别添加命令~\verb|\left|~和~\verb|\right|,则定界符可根据所包围公式大小自动调整其尺寸,这可从式(\ref{nodelimiter}) 和式(\ref{delimiter})中看出。
\begin{equation}\label{nodelimiter}
(\sum_{k=\frac12}^{N^2})
\end{equation}
\begin{equation}\label{delimiter}
\left(\sum_{k=\frac12}^{N^2}\right)
\end{equation}
式(\ref{nodelimiter})和式(\ref{delimiter})是在~\LaTeX~中分别输入如下代码得到的。
\begin{Verbatim}[breaklines=true, breaksymbol=, breakanywheresymbolpre=]
(\sum_{k=\frac12}^{N^2})
\left(\sum_{k=\frac12}^{N^2}\right)
\end{Verbatim}
\verb|\left|~和~\verb|\right|~总是成对出现的,若只需在公式一侧有可自动调整大小的定界符,则只要用“.”代替另一侧那个无需打印出来的定界符即可。
若想获得关于此部分内容的更多信息,可参见~\href{http://tug.ctan.org/cgi-bin/ctanPackageInformation.py?id=voss-mathmode}{Math mode}~文档的第~8~章“Brackets, braces and parentheses”。

\section{数学重音符号}
数学重音符号通常用来区分同一字母表示的不同变量,输入方法如下(需要调用~\verb|amsmath|~宏包):

\vspace{0.5em}\noindent\wuhao\begin{tabularx}{\textwidth}{Xc|Xc|Xc}
 \verb|\acute| & $\acute{a}$ & \verb|\mathring| & $\mathring{a}$ & \verb|\underbrace| & $\underbrace{a}$ \\
 \verb|\bar| & $\bar{a}$ & \verb|\overbrace| & $\overbrace{a}$ & \verb|\underleftarrow| & $\underleftarrow{a}$ \\
 \verb|\breve| & $\breve{a}$ & \verb|\overleftarrow| & $\overleftarrow{a}$ & \verb|\underleftrightarrow| & $\underleftrightarrow{a}$ \\
 \verb|\check| & $\check{a}$ & \verb|\overleftrightarrow| & $\overleftrightarrow{a}$ & \verb|\underline| & $\underline{a}$ \\
 \verb|\dddot| & $\dddot{a}$ & \verb|\overline| & $\overline{a}$ & \verb|\underrightarrow| & $\underrightarrow{a}$ \\
 \verb|\ddot| & $\ddot{a}$ & \verb|\overrightarrow| & $\overrightarrow{a}$ & \verb|\vec| & $\vec{a}$ \\
 \verb|\dot| & $\dot{a}$ & \verb|\tilde| & $\tilde{a}$ & \verb|\widehat| & $\widehat{a}$ \\
 \verb|\grave| & $\grave{a}$ & \verb|\underbar| & $\underbar{a}$ & \verb|\widetilde| & $\widetilde{a}$ \\
 \verb|\hat| & $\hat{a}$
\end{tabularx}\vspace{0.5em}

\xiaosi 当需要在字母~$i$~和~$j$~的上方添加重音符号时,为了去掉这两个字母顶上的小点,这两个字母应该分别改用~\verb|\imath|~ 和~\verb|\jmath|。

如果遇到某些符号不知道该采用什么命令能输出它时,则可通过~\href{http://detexify.kirelabs.org/classify.html}{Detexify$^2$~ 网站}来获取符号命令。若用鼠标左键在此网页的方框区域内画出你所要找的符号形状,则会在网页右方列出和你所画符号形状相近的~5~个符号及其相对应的~\LaTeX~输入命令。若所列出的符号中不包括你所要找的符号,还可通过点击“Select from the complete list!”的链接以得分从低到高的顺序列出所有符号及其相对应的~\LaTeX~输入命令。

最后,建议大家还以~\href{http://tug.ctan.org/cgi-bin/ctanPackageInformation.py?id=voss-mathmode}{Math mode}~这篇~pdf~文档作为主要参考。若要获得最为标准、美观的数学公式排版形式,可以查查文档中是否有和你所要的排版形式相同或相近的代码段,通过修改代码段以获得你所要的数学公式排版形式。如果出现问题,我们推荐大家查阅~ChinaTeXMathFAQ~V1.1。

%% !Mode:: "TeX:UTF-8"

\chapter{罗列、定理和代码环境使用方法}

\section{单层罗列环境}

海南大学学位论文一般可采用两种罗列环境:一种是并列条目有同样标签的~\verb|itemize|~罗列环境,另一种是具有自动排序编号符号的~\verb|enumerate|~罗列环境。这两种罗列环境的样式参数可参考图~\ref{fig:list}。
\begin{figure}[htbp]
\centering
\includegraphics[width = 0.6\textwidth]{list}
\caption{罗列环境参数示意图}\label{fig:list}\vspace{-1em}
\end{figure}

通过调用~enumitem~宏包可以很方便地控制罗列环境的布局,其~format.tex~文件中的~\verb|\setitemize|~和~\verb|\setenumerate|~命令分别用来设置~\verb|itemize|~和~\verb|enumerate|~环境的样式参数。采用~\verb|itemize|~单层罗列环境的排版形式如下:

\begin{itemize}
\item 第一个条目文本内容
\item 第二个条目文本内容
\item 第三个条目文本内容
\end{itemize}

其代码如下

\begin{Verbatim}[breaklines=true, breaksymbol=, breakanywheresymbolpre=]
\begin{itemize}
  \item 第一个条目文本内容
  \item 第二个条目文本内容
  ...
  \item 第三个条目文本内容
\end{itemize}
\end{Verbatim}

采用~\verb|enumerate|~单层罗列环境的排版形式如下:

\begin{enumerate}
\item 第一个条目文本内容
\item 第二个条目文本内容
\item 第三个条目文本内容
\end{enumerate}

其代码如下

\begin{Verbatim}[breaklines=true, breaksymbol=, breakanywheresymbolpre=]
\begin{enumerate}
  \item 第一个条目文本内容
  \item 第二个条目文本内容
  ...
  \item 第三个条目文本内容
\end{enumerate}
\end{Verbatim}



\section{定理环境}

\begin{definition}[谱半径]\label{def:def1}
  称~$n$~阶方阵~$\mathbf{A}$~的全体特征值~$\lambda_1,\cdots,\lambda_n$~组成的集合为~$\mathbf{A}$~的谱,称
  $$\rho(\mathbf{A})=\max{\{|\lambda_1|,\cdots,|\lambda_n|\}}$$
\end{definition}
\begin{theorem}[相似充要条件]\label{lemma:l1}
  方阵$A$和$B$相似的充要条件是:~$A$~和~$B$~有全同的不变因子。
\end{theorem}
\begin{corollary}[推论1]\label{cor:cor1}
在赋范空间~$(X,\|\cdot\|)$~上定义~$d(x,y)=\|x-y\|$, 对任意~$x,y\in X$,~则~$(X,d)$~是距离空间。
\end{corollary}
\begin{proof}
  只需证明~$d(x,y)$~是距离。
\end{proof}
\newpage

定义代码如下:
\begin{Verbatim}[breaklines=true, breaksymbol=, breakanywheresymbolpre=]
 \begin{definition}[谱半径]\label{def:def1}
  称~$n$~阶方阵~$\mathbf{A}$~的全体特征值
  $\lambda_1,\cdots,\lambda_n$组成的集合为~$\mathbf{A}$~的谱,称
  $$\rho(\mathbf{A})=\max{\{|\lambda_1|,\cdots,|\lambda_n|\}}$$
\end{definition}
\end{Verbatim}
\noindent\hrule

\vspace{0.1em}\noindent\hrule
\vspace{1em}
定理代码如下:
\begin{Verbatim}[breaklines=true, breaksymbol=, breakanywheresymbolpre=]
\begin{theorem}[相似充要条件]\label{lemma:l1}
  方阵$A$和$B$相似的充要条件是:$A$和$B$有全同的不变因子。
\end{theorem}
\end{Verbatim}

\noindent\hrule\vspace{0.1em}

\noindent\hrule
\vspace{1em}
推论和证明代码如下:
\begin{Verbatim}[breaklines=true, breaksymbol=, breakanywheresymbolpre=]
\begin{corollary}[推论1]\label{cor:cor1}
在赋范空间~$(X,\|\cdot\|)$~上定义$d(x,y)=\|x-y\|$,
对任意$x,y\in X$,则$(X,d)$是距离空间。
\end{corollary}
\begin{proof}
  只需证明$d(x,y)$是距离。
\end{proof}
\end{Verbatim}
\noindent\hrule\vspace{1em}

定理定义[]中是可选参数,用来说明定理的名称。其他环境格式书写与上面定理、定义、推论格式相同,可自己调用其他环境。
若需要书写定理定义等内容,而且带有顺序编号,需要采用如下环境。除了~\verb|proof|~环境之外,其余~9~个环境都可以有一个可选参数作为附加标题。

\begin{center}
\vspace{0.5em}\noindent\wuhao\begin{tabularx}{0.7\textwidth}{lX|lX}
定理 & \verb|theorem|~环境 & 定义 & \verb|definition|~环境 \\
例 & \verb|example|~环境 & 算法 & \verb|algorithm|~环境 \\
公理 & \verb|axiom|~环境 & 命题 & \verb|proposition|~环境 \\
引理 & \verb|lemma|~环境 & 推论 & \verb|corollary|~环境 \\
注解 & \verb|remark|~环境 & 证明 & \verb|proof|~环境 \\
\end{tabularx}
\end{center}
\section{代码环境}
很多和计算机专业背景相关的同学都会使用到代码环境,使用~\verb|\verb|~指令或者是~\verb|verbatim|~环境固然是一种选择,但是比不上专门的~lstlisting~环境这么专业。
\begin{lstlisting}
int main(int argc, char ** argv)
{
printf("Hello world!\n");
return 0;
}
\end{lstlisting}

\noindent\hrule
\vspace{0.1em}\noindent\hrule

\vspace{1em}

\noindent 代码如下:
\begin{Verbatim}[breaklines=true, breaksymbol=, breakanywheresymbolpre=]
\begin{lstlisting}
int main(int argc, char ** argv)
{
printf("Hello world!\n");
return 0;
}
\end{lstlisting}
\end{Verbatim}
\noindent\hrule\vspace{1em}

在代码中显示的关键字为蓝色,框的左侧显示的是行号,这样便于读者阅读和查找代码,同时添加了浅蓝色的阴影边框,达到了美观的效果。
代码环境的设置已在~package~中~\verb|\lstset|~指令中定义。如果要修改要插入的代码的语言直接将language定义为要插入的语言即可,如插入Java那么language=Java即可。定义中支持跨页显示,可以将较长的代码置于~lstlisting~环境中。
\section{算法环境}
很多和计算机专业背景相关的同学会使用到算法环境,之前使用到的定理环境固然是一种选择,但是比不上专门的~algorithm2e~环境这么专业。为了实现专业和接近完美,本版本支持算法环境。如下所示:
\begin{algorithm}[H]
    \caption{算法标题}
    \label{alg:demoAlgo} % 贴上标签以便交叉引用
    \begin{algorithmic}[1]  % 这个 1 表示每一行都显示数字
    \STATE 初始化...
    \FOR{$i=0;i\le M; i\rightarrow i + 1$}
        \STATE 执行语句~1;
        \STATE 执行语句~2;
        \STATE ...
    \ENDFOR
    \STATE ...
    \WHILE{某条件}
        \STATE 执行语句~1;
        \STATE 执行语句~2;
        \STATE ...
    \ENDWHILE
    \STATE ...
    \end{algorithmic}
\end{algorithm}
\noindent\hrule
\vspace{0.1em}\noindent\hrule

\vspace{1em}

\noindent 代码如下:
\begin{Verbatim}[breaklines=true, breaksymbol=, breakanywheresymbolpre=]
  \begin{algorithm}[H]
    \caption{算法标题}
    \label{alg:demoAlgo} % 贴上标签以便交叉引用
    \begin{algorithmic}[1]  % 这个 1 表示每一行都显示数字
    \STATE 初始化...
    \FOR{$i=0;i\le M; i\rightarrow i + 1$}
        \STATE 执行语句~1;
        \STATE 执行语句~2;
        \STATE ...
    \ENDFOR
    \STATE ...
    \WHILE{某条件}
        \STATE 执行语句~1;
        \STATE 执行语句~2;
        \STATE ...
    \ENDWHILE
    \STATE ...
    \end{algorithmic}
\end{algorithm}
\end{Verbatim}


%\include{body/conclusion}
\end{Verbatim}
那么,编译的时候就只编译未加~\%~的一章,在这个例子中,即本章~intros。

理论上,并不一定要把每章放在不同的文件中。但是这种自顶向下,分章节写作、编译的方法有利于提高效率,大大减少~Debug~过程中的编译时间,同时减小风险。

\section{参考文献生成和标注方法}

\LaTeX~具有插入参考文献的能力。Google Scholar~网站上存在兼容~BibTeX~的参考文献信息,通过以下几个步骤,可以轻松完成参考文献的生成。
\begin{itemize}
  \item 在\href{http://scholar.google.com/}{谷歌学术搜索}中,
        点击\href{http://scholar.google.com/scholar_preferences?hl=en&as_sdt=0,5}{学术搜索设置}。
  \item 页面打开之后,在\textbf{文献管理软件}选项中选择\textbf{显示导入~BibTeX~的链接},单击保存设置,退出。
  \item 在谷歌学术搜索中检索到文献后,在文献条目区域单击导入~BibTeX~选项,页面中出现文献的引用信息。
  \item 将文献引用信息的内容复制之后,添加到~references~文件夹下的~reference.bib~ 中。
\end{itemize}
\par{如在scholar中搜索“基于方差及方差梯度的指纹图像自适应分割算法”,那么其BibTex文件如下:
 "@article\{樊冬进2008基于方差及方差梯度的指纹图像自适应分割算法,
  title=\{基于方差及方差梯度的指纹图像自适应分割算法\},
  author=\{樊冬进 and 孙冰 and 封举富 and others\},
  year=\{2008\}
\}"
直接使用会报错,我们需要将”@article\{樊冬进2008基于方差及方差梯度的指纹图像自适应分割算法“改为@article\{anyEnglishName

}


在正文中标注参考文献时,在需要标注的地方输入~\verb|\cite{}|~指令,花括号内输入参考文献引用信息中的第一行信息即可(常常为文献的缩略信息),此时~\verb|[]|~符号在标注处的右上角显示。

\section{注意事项}

\begin{enumerate}
  \item 由于模板使用~UTF-8~编码,所以源文件应该保存成~UTF-8~格式,否则可能出现中文字符无法识别的错误。
  本模板中每一个~.tex~文件的文件的开头已经加上一行:\\
  \verb|% !Mode:: "TeX:UTF-8"|\\
     这样可以确保~.tex~文件默认使用~UTF-8~的格式打开。读者如果删去此行,很有可能会导致中文字符显示乱码。
     在~WinEdt~编辑器中可以使用以下两种方式保存成~UTF-8~格式:
      \begin{enumerate}
        \item 先建立~.tex~文件,另存为~.tex~文件时,选择用~UTF-8~ 格式保存。
        \item
            在~WinEdt~编辑器中,选择\\
            \mbox{~Document$\rightarrow$Document Settings$\rightarrow$Document Mode $\rightarrow$TeX:UTF-8} 同时在~WinEdt~最下面的状态栏中,可以看到该文档是~TeX~ 格式还是~TeX:UTF-8~格式。
            当文档为~TeX:UTF-8~格式时,状态栏一般显示:
            \makebox[\textwidth][l]{Wrap | Indent | INS | LINE |Spell | TeX:UTF-8 | -src~ 等。}
      \end{enumerate}
  \item 如果在~pdf~书签中,中文显示乱码的话,则注意以下说明:
    \begin{Verbatim}[breaklines=true, breaksymbol=, breakanywheresymbolpre=]
        \usepackage{CJKutf8}
        % 1. 如果使用CJKutf8
        %    Hyperref中应使用unicode参数
        % 2. 如果使用CJK
        %    Hyperref则使用CJKbookmarks参数
        %    可惜得到的PDF书签是乱码,建议弃用
        % 3. Unicode选项和CJKbookmarks不能同时使用
        \usepackage[
        %CJKbookmarks=true,
        unicode=true
        ]{hyperref}
     \end{Verbatim}
  \item 建议采用以下两种编译方式:
  \begin{enumerate}
    \item latex + bibtex + latex + latex + dvi2pdf. 在这种编译情况下,对应的~hnumain.tex~文件的第一行是\verb|\def\usewhat{dvipdfmx}|~ (缺省设置)。 此时,所有图片文件应该保存为~.eps~格式,如~figures~文件夹里~.eps~图片。
          如果您选择在命令行中操作,可以在编译的时候依次输入~latex hnumain, bibtex hnumain, latex hnumain, latex hnumain~和~dvipdfmx hnumain, 编译完成之后,需要手动打开~pdf~文件。需要说明的是,为了是操作简便,以上命令已经作为~pdfmake.bat~批处理文件放在目录中。在编译无误的前提下,双击此文件,可以一键生成~pdf~。

    \item pdflatex + pdflatex. 在这种编译情况下,对应的~hnumain.tex~文件的第一行应该改为\verb|\def\usewhat{pdflatex}|~。 此时, 编译不支持~.eps~图片格式,此时需要在命令行下使用~epstopdf~指令将~figures~文件夹下 的~.eps~文件转化成~.pdf~ 文件格式,命令行中操作格式为~epstopdf a.eps~。
          在命令行编译的时候,依次输入~pdflatex hnumain~和~pdflatex hnumain, 编译完成之后,需要手动打开~pdf~文件。
  \end{enumerate}
    \item  当参考文献在编辑的时候,第一行为标签行,为该文献的缩略信息。当复制中文参考文献的~BibTeX~页面到~reference.bib~文件中时,需要把原来含有中文的标签行改成英文书写,否则会报错。
\end{enumerate}

\section{系统要求}

     CTeX 2.8, MiKTeX 2.8, TeX Live 2009~或以上版本。我们推荐您使用最新的~CTeX~中文套装,~CTeX 2.9.2.164~Full~版本,内含~WinEdt 7.0~编辑器,可以完成文件的编辑和编译工作。本模板目前只在~Windows~操作系统下测试通过,尚未在~Mac~ 系统和~Linux~系统下测试。

\section{Why~\LaTeX~?}

选择使用~\TeX/\LaTeX~的理由包括:
\begin{itemize}
\item 免费软件;
\item 专业的排版效果;
\item 是事实上的专业数学排版标准;
\item 广泛的西文期刊接收甚或只接收~\LaTeX~格式的投稿;
\item[] ……
\end{itemize}
不选择使用~\TeX/\LaTeX~的理由包括:
\begin{itemize}
\item 需要相当精力学习;
\item 图文混合排版能力不够强;
\item 对于表格的支持较差;
\item 仅在数学、物理、计算机等领域流行;
\item 中文期刊的支持较差;
\item[] ……
\end{itemize}

如果想知道~\LaTeX~与~Word~的详细区别,请参见:~\url{http://zzg34b.w3.c361.com/homepage/compareWord.htm}。

\subsection{我应该看什么~\LaTeX~读物~?}

我觉得最好在使用中学会~\LaTeX~,至于参考资料百度、Google上很多。如果实在需要读物,那么可以在以下几本书上进行选择。

\begin{enumerate}
\item 我能阅读英文
\begin{enumerate}
\item 迅速入门:lshort.pdf (中文版名为:一份不太简短的~\LaTeX{}~介绍)
\item 系统学习:A Guide to LaTeX, 4th Edition, Addison-Wesley
                机械工业出版社的有影印版,名曰~《\LaTeX{}~ 实用教程》
\item 深入学习:Knuth 《TeXbook》:必读。 《LATEX Companion》:如果说 高老头的~TeXbook~是论语,那么这本
               书算是一本史记,全面而精妙,是所有~\LaTeX~书中的精品。
\end{enumerate}

\item 我更愿意阅读中文
\begin{enumerate}
\item 迅速入门:lnotes2.pdf (\LaTeX~Notes~雷太赫排版系统简介, v2.0, 包太雷)
\item 系统学习:《\LaTeXe{}~科技排版指南》,邓建松(电子版)
      如果不好找,可以阅读陈志杰等《\LaTeXe~入门与提高》第二版,或者胡伟《\LaTeXe~完全学习手册》
\item 深入学习:~TeXbook0.pdf~(特可爱原本,TeXbook~的中译,xianxian)
\item 具体问题释疑:~CTeX-FAQ.pdf~,\\
        吴凌云,~\url{http://www.ctex.org/CTeXFAQ}~\\
      ~ChinaTeXMathFAQ~$V1.1$~~China\TeX~ 数学排版常见问题集
\end{enumerate}
\end{enumerate}

遇见问题和解决问题的过程可以快速提高自己的技能,建议此时:
\begin{itemize}
 \item 清楚,扼要地提出你的问题。
 \item 使用~Google~搜索。
\end{itemize}

\section{免责声明}

本模板依据《本科生毕业论文(设计)封面》和《毕业论文(设计)排版打印格式》编写,作者希望能给使用者写作论文带来方便。然而,作者不保证本模板完全符合学校要求,也不对由此带来的风险和损失承担任何责任。

%% !Mode:: "TeX:UTF-8"

\chapter{图片的插入方法}

\section{海南大学对于插图的要求}

 \par{插图要求,所有插图按分章编号,如第1章的第1张插图为“图1-1 ****图”,字体可以用宋体5号字。所有插图均需有图注(图的说明),不能只有图编号。图号及图注应在图的下方居中标出;一幅图如有若干幅分图,均应编分图号,用(a),(b),(c)......按顺序编排;插图须紧跟文述,在正文中,一般应先见图号及图的内容后再见图(即:正文中先见“本系统功能架构如图3-2所示。”字样,再在段后见图3-2),一般情况下不能提前见图,特殊情况需延后的插图不应跨节。}
\par{图形符号及各种线型画法须按照现行的国家标准;坐标图中坐标上须注明标度值,并标明坐标轴所表示的物理量名称及量纲,应均按国际标准(SD)标注,例如:kW,  m/s, N, m....等,但对一些示意图例外;图应具有“自明性”(即只看图、图题和图例,不阅读正文,就可理解图意);图中用字最小为宋体小五号字;插图必须同内容密切联系,切忌与文字和表重复。}
\par{图的绘制一定要紧凑美观并且保证清晰,避免使用彩色和带底色的图,以免影响论文印刷。}
\par{切忌直接从他人文章、其他文献、书籍扫描和网上直接拷贝图形,全文不能出现非本人绘制的图形(系统实现部分的系统界面截图除外),必须使用他人插图时,须在图题正下方注明出处。
}

\section{\LaTeX~中推荐使用的图片格式}

在~\LaTeX~中应用最多的图片格式是~EPS(Encapsulated PostScript)格式,它是一种专用的打印机描述语言,常用于印刷或打印输出。
EPS~格式图片可通过多种方式生成在这里可以采用Adobe Acrobat 软件进行转换。选中图片右击转换为pdf然后,选择保存,在保存格式下面选择文件的格式为eps即可。,对于用visio写的流程图,其转换格式和普通图片转换成eps格式完全相同。

\section{单张图片的插入方法}
单张图片独自占一行的插入形式如图~\ref{fig:xml}~所示。
\begin{figure}[htbp]
\centering
\includegraphics[width=0.4\textwidth]{XML}
\caption{树状结构}\label{fig:xml}
\vspace{\baselineskip}
\end{figure}


其插入图片的代码及其说明如下。
\vspace{1em}\noindent\hrule
\begin{Verbatim}[breaklines=true, breaksymbol=, breakanywheresymbolpre=]
\begin{figure}[htbp]
\centering
\includegraphics[width=0.4\textwidth]{文件名(.eps)}
\caption{标题}\label{标签名(通常为 fig:labelname)}
\vspace{\baselineskip} % 表示图与正文空一行
\end{figure}
\end{Verbatim}

\noindent\hrule

\begin{Verbatim}[breaklines=true, breaksymbol=, breakanywheresymbolpre=]
figure环境的可选参数[htbp]表示浮动图形所放置的位置,h (here)表示当前位置,t (top)表示页芯顶部,b (bottom)表示页芯底部,p (page)表示单独一页。在Word等软件中,图片通常插入到当前位置,如果当前页的剩余空间不够,图片将被移动到下一页,当前页就会出现很大的空白,其人工调整工作非常不便。由LaTeX提供的浮动图片功能,总是会按h->t->b->p的次序处理选项中的字母,自动调整图片的位置,大大减轻了工作量。
\centering命令将后续内容转换成每行皆居中的格式。
"\includegraphics"的可选参数用来设置图片插入文中的水平宽度,一般表示为正文宽度(\textwidth)的倍数。
\caption命令可选参数“标签名”为英文形式,一般不以图片或表格的数字顺序作为标签,而应包含一定的图片或表格信息,以便于文中引用(若图片、表格、公式、章节和参考文献等在文中出现的先后顺序发生了变化,其标注序号及其文中引用序号也会跟着发生变化,这一点是Word等软件所不能做到的)。另外,图题或表题并不会因为分页而与图片或表格体分置于两页,章节等各级标题也不会置于某页的最底部,LaTeX系统会自动调整它们在正文中的位置,这也是Word等软件所无法匹敌的。
\vspace将产生一定高度的竖直空白,必选参数为负值表示将后续文字位置向上提升,参数值可自行调整。em为长度单位,相当于大写字母M的宽度。\vspace{\baselineskip} 表示图与正文空一行。
引用方法:“见图~\ref{fig:figname}”、“如图~\ref{fig:figname}~所示”等。
\end{Verbatim}

\noindent\hrule\vspace{1em}

若需要将~2~张及以上的图片并排插入到一行中,则需要采用\verb|minipage|环境,如图~\ref{fig:dd}~和图~\ref{fig:ds}~所示。
\begin{figure}[htbp]
\centering
\begin{minipage}{0.4\textwidth}
\centering
\includegraphics[width=\textwidth]{dataDimensions}
\caption{数据维数的变化}\label{fig:dd}
\end{minipage}
\begin{minipage}{0.4\textwidth}
\centering
\includegraphics[width=\textwidth]{dataSize}
\caption{数据规模的变化}\label{fig:ds}
\end{minipage}
\vspace{\baselineskip}
\end{figure}

其代码如下所示。
\vspace{1em}\noindent\hrule
\begin{Verbatim}[breaklines=true, breaksymbol=, breakanywheresymbolpre=]
\begin{figure}[htbp]
\centering
\begin{minipage}{0.4\textwidth}
\centering
\includegraphics[width=\textwidth]{文件名}
\caption{标题}\label{fig:f1}
\end{minipage}
\begin{minipage}{0.4\textwidth}
\centering
\includegraphics[width=\textwidth]{文件名}
\caption{标题}\label{fig:f2}
\end{minipage}\vspace{\baselineskip}
\end{figure}
\end{Verbatim}

\noindent\hrule

\begin{Verbatim}[breaklines=true, breaksymbol=, breakanywheresymbolpre=]
minipage环境的必选参数用来设置小页的宽度,若需要在一行中插入n个等宽图片,则每个小页的宽度应略小于(1/n)\textwidth。
\end{Verbatim}

\noindent\hrule

\section{具有子图的图片插入方法}

图中若含有子图时,需要调用~subfigure~宏包, 如图~\ref{fig:subfig}~所示。
\begin{figure}[htbp]
  \centering
  \subfigure[Data Dimensions]{\label{fig:subfig:datadim}
                \includegraphics[width=0.4\textwidth]{dataDimensions}}
  \subfigure[Data Size]{\label{fig:subfig:datasize}
                \includegraphics[width=0.4\textwidth]{dataSize}}
  \caption{Scalability of data}\label{fig:subfig}
\vspace{\baselineskip}
\end{figure}

其代码及其说明如下。
\vspace{1em}\noindent\hrule

\begin{Verbatim}[breaklines=true, breaksymbol=, breakanywheresymbolpre=]
\begin{figure}[htbp]
  \centering
  \subfigure[第1个子图标题]{
            \label{第1个子图标签(通常为 fig:subfig1:subsubfig1)}
            \includegraphics[width=0.4\textwidth]{文件名}}
  \subfigure[第2个子图标题]{
            \label{第2个子图标签(通常为 fig:subfig1:subsubfig2)}
            \includegraphics[width=0.4\textwidth]{文件名}}
  \caption{总标题}\label{总标签(通常为 fig:subfig1)}
\vspace{\baselineskip}
\end{figure}
\end{Verbatim}

\noindent\hrule

\begin{Verbatim}[breaklines=true, breaksymbol=, breakanywheresymbolpre=]
子图的标签实际上可以随意设定,只要不重复就行。但为了更好的可读性,我们建议fig:subfig:subsubfig格式命名,这样我们从标签名就可以知道这是一个子图引用。
引用方法:总图的引用方法同本章第1节,子图的引用方法用\ref{fig:subfig:subsubfig}来代替。
\end{Verbatim}

\noindent\hrule\vspace{1em}

子图的引用示例:如图~\ref{fig:subfig:datadim}~和图~\ref{fig:subfig:datasize}~所示。

若想获得插图方法的更多信息,参见网络上的~\href{ftp://ftp.tex.ac.uk/tex-archive/info/epslatex.pdf}{Using Imported Graphics in \LaTeX and pdf\LaTeX}~文档。

%% !Mode:: "TeX:UTF-8"

\chapter{表格的绘制方法}

\section{本科生毕业论文的绘表规范}

表中内容应与叙述文字内容相呼应,表的结构应简洁明了,表随文排,字体为宋体5号字注,一定使用单倍行距,要排版紧凑,以与正文内容区别。对于文中的各类表,一定注意先有引用文字(格式一般为“如表8-5所示”)然后见到表,插表表名要简明贴切,表序按章用阿拉伯字编列,表序末和表名末均不加标点符号,写在表的上方。表格较长如需转页,则在下页稿纸上重写表头,并在表的右上方写“续表”,表内全部数据的统一计数或计量单位应置于表的右上角,若表中各栏计量单位不同,则将单位分别列入表头的各栏中,将量的符号与单位符号之间用斜线隔开,即表中的数值用量与单位的比值形式表示。表内数据对应位上下对齐,一般以小数点为准;数字间夹有“—”,“/”号者,以这些符号对齐;无数据或文字处一律空白。相邻栏内数字相同时,应重复书写,勿用“同左”、“同上”等;表内文字说明,空一格起行,转行顶格,并正确使用标点符号,但每段最后一律不用标点符号。表内名词短语、数据需注释时,用脚注,即在所需加注名词或数据的右上角注符号“①,②,…”或星号,在表的底线下方写出相应的符号和注文,不出现“注”字,如对整个表加以说明时,可附注于底线下方,注文前应有“说明:”字样。

\section{普通表格的绘制方法}

表格应具有三线表格式,因此需要调用~booktabs~宏包,其标准格式如表~\ref{tab:table1}~所示。
\begin{table}[htbp]
\caption{符合本科生毕业论文绘图规范的表格}\label{tab:table1}
\vspace{0.5em}\centering\wuhao
\begin{tabular}{ccccc}
\toprule[1.5pt]
$D$(in) & $P_u$(lbs) & $u_u$(in) & $\beta$ & $G_f$(psi.in)\\
\midrule[1pt]
 5 & 269.8 & 0.000674 & 1.79 & 0.04089\\
10 & 421.0 & 0.001035 & 3.59 & 0.04089\\
20 & 640.2 & 0.001565 & 7.18 & 0.04089\\
 5 & 269.8 & 0.000674 & 1.79 & 0.04089\\
10 & 421.0 & 0.001035 & 3.59 & 0.04089\\
20 & 640.2 & 0.001565 & 7.18 & 0.04089\\
 5 & 269.8 & 0.000674 & 1.79 & 0.04089\\
10 & 421.0 & 0.001035 & 3.59 & 0.04089\\
20 & 640.2 & 0.001565 & 7.18 & 0.04089\\
 5 & 269.8 & 0.000674 & 1.79 & 0.04089\\
10 & 421.0 & 0.001035 & 3.59 & 0.04089\\
20 & 640.2 & 0.001565 & 7.18 & 0.04089\\
\bottomrule[1.5pt]
\end{tabular}
\vspace{\baselineskip}
\end{table}

其绘制表格的代码及其说明如下。
\vspace{1em}\noindent\hrule

\begin{VerbWithBreak}
\begin{table}[htbp]
\caption{表标题}\label{标签名(通常为 tab:tablename)}
\vspace{0.5em}\centering\wuhao
\begin{tabular}{cc...c}
\toprule[1.5pt]
表头第1个格   & 表头第2个格   & ... & 表头第n个格  \\
\midrule[1pt]
表中数据(1,1) & 表中数据(1,2) & ... & 表中数据(1,n)\\
表中数据(2,1) & 表中数据(2,2) & ... & 表中数据(2,n)\\
表中数据(3,1) & 表中数据(3,2) & ... & 表中数据(3,n)\\
表中数据(4,1) & 表中数据(4,2) & ... & 表中数据(4,n)\\
...................................................\\
表中数据(m,1) & 表中数据(m,2) & ... & 表中数据(m,n)\\
\bottomrule[1.5pt]
\end{tabular}
\vspace{\baselineskip}
\end{table}
\end{VerbWithBreak}

\noindent\hrule

\begin{VerbWithBreak}
table环境是一个将表格嵌入文本的浮动环境。
\wuhao命令将表格的字号设置为五号字(10.5pt),在绘制表格结束退出时,不需要将字号再改回为\xiaosi,正文字号默认为小四号字(12pt)。
tabular环境的必选参数由每列对应一个格式字符所组成:c表示居中,l表示左对齐,r表示右对齐,其总个数应与表的列数相同。此外,@{文本}可以出现在任意两个上述的列格式之间,其中的文本将被插入每一行的同一位置。表格的各行以\\分隔,同一行的各列则以&分隔。
\toprule、\midrule和\bottomrule三个命令是由booktabs宏包提供的,其中\toprule和\bottomrule分别用来绘制表格的第一条(表格最顶部)和第三条(表格最底部)水平线,\midrule用来绘制第二条(表头之下)水平线,且第一条和第三条水平线的线宽为1.5pt,第二条水平线的线宽为1pt。
引用方法:“如表~\ref{tab:tablename}~所示”。
\end{VerbWithBreak}

\noindent\hrule

\section{长表格的绘制方法}

长表格是当表格在当前页排不下而需要转页接排的情况下所采用的一种表格环境。若长表格仍按照普通表格的绘制方法来获得,
其所使用的~\verb|table|~浮动环境无法实现表格的换页接排功能,表格下方过长部分会排在表格第~1~页的页脚以下。为了能够实现长表格的转页接排功能,
需要调用~\verb|longtable|~宏包,由于长表格是跨页的文本内容,因此只需要单独的~\verb|longtable|~环境,所绘制的长表格的格式如表~\ref{tab:table2}~所示。

此长表格~\ref{tab:table2}~第~2~页的标题“编号(续表)”和表头是通过代码自动添加上去的,无需人工添加,若表格在页面中的竖直位置发生了变化,长表格在第~2~页
及之后各页的标题和表头位置能够始终处于各页的最顶部,也无需人工调整,\LaTeX~系统的这一优点是~Word~等软件所无法企及的。

下段内容是为了让下面的长表格分居两页,看到表标题“编号(续表)”的效果。此模板的完成时间正值雨后初霁的四月二十五日,故引用林徽因《你是人间的四月天》全文:
\begin{center}
\begin{minipage}[c]{0.5\textwidth}

\textbf{你是人间的四月天}

\vspace{12pt}
我说你是人间的四月天\\
笑音点亮了四面风\\
轻灵在春的光艳中交舞着变\\
你是四月早天里的云烟\\
黄昏吹着风的软\\
星子在无意中闪\\
细雨点洒在花前\\
那轻~~那娉婷\\
你是鲜妍\\
百花的冠冕你戴着\\
你是天真~~庄严~~你是夜夜的月圆\\
雪化后那片鹅黄\\
你像~~新鲜初放芽的绿\\
你是柔嫩喜悦\\
水光浮动着你梦期待中白莲\\
你是一树一树的花开\\
是燕~~在梁间呢喃\\
你是爱~~是暖\\
是希望\\
你是人间的四月天
\end{minipage}
\end{center}

					
\wuhao\begin{longtable}{ccc}
\caption{海南大学各学院名称一览}\label{tab:table2}
 \vspace{0.5em}\\
\toprule[1.5pt] 学院名称 & 网址 & 邮政编码  \\ \midrule[1pt]
\endfirsthead
\multicolumn{3}{c}{表~\thetable(续表)}\vspace{0.5em}\\
\toprule[1.5pt] 学院名称 & 网址 & 邮政编码  \\ \midrule[1pt]
\endhead
\bottomrule[1.5pt]
\endfoot
海南大学& \url{http://www.hainu.edu.cn/}& 570228\\
机电工程学院&  \url{http://www.hainu.edu.cn/jidian/}& 570228\\
\end{longtable}\xiaosi
\vspace{\baselineskip}

绘制长表格的代码及其说明如下。
\vspace{1em}\noindent\hrule

\begin{VerbWithBreak}
\wuhao\begin{longtable}{cc...c}
\caption{表标题}\label{标签名(通常为 tab:tablename)}\\
\toprule[1.5pt] 表头第1个格 & 表头第2个格 & ... & 表头第n个格\\ \midrule[1pt]
\endfirsthead
\multicolumn{n}{c}{表~\thetable(续表)}\vspace{0.5em}\\
\toprule[1.5pt] 表头第1个格 & 表头第2个格 & ... & 表头第n个格\\ \midrule[1pt]
\endhead
\bottomrule[1.5pt]
\endfoot
表中数据(1,1) & 表中数据(1,2) & ... & 表中数据(1,n)\\
表中数据(2,1) & 表中数据(2,2) & ... & 表中数据(2,n)\\
...................................................\\
表中数据(m,1) & 表中数据(m,2) & ... & 表中数据(m,n)\\
\end{longtable}\xiaosi
\end{VerbWithBreak}

\noindent\hrule
\begin{VerbWithBreak}
在绘制长表格的前面留出一个空白行,并在第2行的一开始全局定义长表格的字号为五号字,这样能够保证长表格之前段落的行距保持不变。
在绘制长表格结束后,需要\xiaosi命令重新将字号改为小四号字。
\endhead之前的文字描述的是第2页及其之后各页的标题或表头;
\endfirsthead之前的文字描述的是第1页的标题和表头,若无此命令,则第1页的表头和标题由\endhead命令确定;
同理,\endfoot之前的文字描述的是除最后一页之外每页的表格底部内容;
\endlastfoot之前的文字描述的是最后一页的表格底部内容,若无此命令,
则最后一页的表格底部内容由\endfoot命令确定;由于规范中长表格每页底部内容均相同(水平粗线),因此模板中没有用到\endlastfoot命令。
\end{VerbWithBreak}

\noindent\hrule
\section{列宽可调表格的绘制方法}
论文中能用到列宽可调表格的情况共有两种:一种是当插入的表格某一单元格内容过长以至于一行放不下的情况,
另一种是当对公式中首次出现的物理量符号进行注释的情况。这两种情况都需要调用~tabularx~宏包。下面将分别对这两种情况下可调表格的绘制方法进行阐述。
\subsection{表格内某单元格内容过长的情况}

首先给出这种情况下的一个例子如表~\ref{tab:table3}~所示。
\begin{table}[htbp]
\caption{最小的三个正整数的英文表示法}\label{tab:table3}
\vspace{0.5em}\wuhao
\begin{tabularx}{\textwidth}{llX}
\toprule[1.5pt]
Value & Name & Alternate names, and names for sets of the given size\\\midrule[1pt]
1 & One & ace, single, singleton, unary, unit, unity\\
2 & Two & binary, brace, couple, couplet, distich, deuce, double, doubleton, duad, duality, duet, duo, dyad, pair, snake eyes, span, twain, twosome, yoke\\
3 & Three & deuce-ace, leash, set, tercet, ternary, ternion, terzetto, threesome, tierce, trey, triad, trine, trinity, trio, triplet, troika, hat-trick\\\bottomrule[1.5pt]
\end{tabularx}
\vspace{\baselineskip}
\end{table}
绘制这种表格的代码及其说明如下。
\vspace{1em}\noindent\hrule
\begin{VerbWithBreak}
\begin{table}[htbp]
\caption{表标题}\label{标签名(通常为 tab:tablename)}
\vspace{0.5em}\wuhao
\begin{tabularx}{\textwidth}{l...X...l}
\toprule[1.5pt]
表头第1个格   & ... & 表头第X个格   & ... & 表头第n个格  \\
\midrule[1pt]
表中数据(1,1) & ... & 表中数据(1,X) & ... & 表中数据(1,n)\\
表中数据(2,1) & ... & 表中数据(2,X) & ... & 表中数据(2,n)\\
.........................................................\\
表中数据(m,1) & ... & 表中数据(m,X) & ... & 表中数据(m,n)\\
\bottomrule[1.5pt]
\end{tabularx}
\vspace{\baselineskip}
\end{table}
\end{VerbWithBreak}

\noindent\hrule
\begin{VerbWithBreak}
tabularx环境共有两个必选参数:第1个参数用来确定表格的总宽度,这里取为排版表格能达到的最大宽度——正文宽度\textwidth;第2 个参数用来确定每列格式,其中标为X的项表示该列的宽度可调,其宽度值由表格总宽度确定。
标为X的列一般选为单元格内容过长而无法置于一行的列,这样使得该列内容能够根据表格总宽度自动分行。若列格式中存在不止一个X 项,则这些标为X的列的列宽相同,因此,一般不将内容较短的列设为X。
标为X的列均为左对齐,因此其余列一般选为l(左对齐),这样可使得表格美观,但也可以选为c或r。如果想要X项为中心对齐,可以在后面加<{\centering},形同\begin{tabularx}{\textwidth}{cX<{\centering}X<{\centering}}
\end{VerbWithBreak}

\noindent\hrule
\subsection{对物理量符号进行注释的情况}
为使得对公式中物理量符号注释的转行与破折号“———”后第一个字对齐,此处最好采用表格环境。此表格无任何线条,左对齐,
且在破折号处对齐,一共有“式中”二字、物理量符号和注释三列,表格的总宽度可选为文本宽度,因此应该采用\verb|tabularx|环境。
由\verb|tabularx|环境生成的对公式中物理量符号进行注释的公式如式(\ref{eq:1})所示。
%\vspace*{10pt}

\begin{equation}\label{eq:1}
\ddot{\boldsymbol{\rho}}-\frac{\mu}{R_{t}^{3}}\left(3\mathbf{R_{t}}\frac{\mathbf{R_{t}\rho}}{R_{t}^{2}}-\boldsymbol{\rho}\right)=\mathbf{a}
\end{equation}

\begin{tabularx}{\textwidth}{@{}l@{\quad}r@{———}X@{}}
式中& $\bm{\rho}$ &追踪飞行器与目标飞行器之间的相对位置矢量;\\
&  $\bm{\ddot{\rho}}$&追踪飞行器与目标飞行器之间的相对加速度;\\
&  $\mathbf{a}$   &推力所产生的加速度;\\
&  $\mathbf{R_t}$ & 目标飞行器在惯性坐标系中的位置矢量;\\
&  $\omega_{t}$ & 目标飞行器的轨道角速度;\\
&  $\mathbf{g}$ & 重力加速度,$=\frac{\mu}{R_{t}^{3}}\left(
3\mathbf{R_{t}}\frac{\mathbf{R_{t}\rho}}{R_{t}^{2}}-\bm{\rho}\right)=\omega_{t}^{2}\frac{R_{t}}{p}\left(
3\mathbf{R_{t}}\frac{\mathbf{R_{t}\rho}}{R_{t}^{2}}-\bm{\rho}\right)$,这里~$p$~是目标飞行器的轨道半通径。
\end{tabularx}
\vspace{\wordsep}

其中生成注释部分的代码及其说明如下。

\vspace{1em}\noindent\hrule

\begin{VerbWithBreak}
\begin{tabularx}{\textwidth}{@{}l@{\quad}r@{— — —}X@{}}
式中 & symbol-1 & symbol-1的注释内容;\\
     & symbol-2 & symbol-2的注释内容;\\
     .............................;\\
     & symbol-m & symbol-m的注释内容。
\end{tabularx}\vspace{\wordsep}
\end{VerbWithBreak}

\noindent\hrule

\begin{VerbWithBreak}
   tabularx环境的第1个参数选为正文宽度,第2个参数里面各个符号的意义为:
   第1个@{}表示在“式中”二字左侧不插入任何文本,“式中”二字能够在正文中左对齐,若无此项,则“式中”二字左侧会留出一定的空白;
   @{\quad}表示在“式中”和物理量符号间插入一个空铅宽度的空白;
   @{— — —}实现插入破折号的功能,它由三个1/2的中文破折号构成;
   第2个@{}表示在注释内容靠近正文右边界的地方能够实现右对齐。
\end{VerbWithBreak}

\noindent\hrule\vspace{1em}

由此方法生成的注释内容应紧邻待注释公式并置于其下方,因此不能将代码放入~\verb|table|~浮动环境中。但此方法不能实现自动转页接排,
可能会在当前页剩余空间不够时,全部移动到下一页而导致当前页出现很大空白。因此在需要转页处理时,还请您手动将需要转页的代码放入一个
新的~\verb|tabularx|~环境中,将原来的一个~\verb|tabularx|~环境拆分为两个~\verb|tabularx|~环境。

\section{为表格添加注释的方法}
首先给出这种情况下的一个例子如表~\ref{tab:table4}~所示。
\begin{table}[!ht]
   \caption{海南省相关数据}\label{tab:table4}
   \vspace{0.5em}\centering\wuhao
   \begin{threeparttable}
   \begin{tabularx}{\textwidth}{c*5{X<{\centering}}}
   \toprule[1.5pt]
   年份/年 & 农业机械总动力/千瓦 & 农村耗电量/亿千瓦小时 & 农田有效灌溉面积/千公顷 & 互联网络宽带接入用户/万户 & 长途光缆线路长度/万公里\\
   \midrule[1pt]
   2011年 & 444.33 & 7.07 & 247.51 & 16.2 & 0.33\\
   2012年 & 479.66 & 8.59 & 286.75 & 21.4 & 0.33\\
   2013年 & 502.1 & 9.63 & 260.93 & 26 & 0.33\\
   2014年 & 517.31 & 10.89 & 259.92 & 29.7 & 0.33\\
   2015年 & 511.59 & 13.02 & 263.99 & 42.2 & 0.34\\
   2016年 & 516.57 & 13.89 & 289.95 & 60.2 & 0.34\\
   2017年 & 569.8 & 15.52 & 289.25 & 79.6 & 0.34\\
   2018年 & 565.82 & 17.41 & 290.48 & 82.6 & 0.09\\
   2019年 & 581.23 & 18.68 & 290.63 & 99.7 & 0.32 \\
   2020年 & 615.57 & 20.96 & 292.21 & 111.2 & 0.32 \\
   \bottomrule[1.5pt]
   \end{tabularx}
   \begin{tablenotes}
       \footnotesize
       \item[1] 数据来源:国家统计局
   \end{tablenotes}
   \end{threeparttable}
   \vspace{\baselineskip}
\end{table}

绘制这种表格的代码及其说明如下。
\vspace{1em}\noindent\hrule
\begin{VerbWithBreak}
   \begin{table}[!ht]
      \caption{海南省相关数据}
      \vspace{0.5em}\centering\wuhao
      \begin{threeparttable}
      \begin{tabularx}{\textwidth}{c*5{X<{\centering}}}
      \toprule[1.5pt]
      年份/年 & 农业机械总动力/千瓦 & 农村耗电量/亿千瓦小时 & 农田有效灌溉面积/千公顷 & 互联网络宽带接入用户/万户 & 长途光缆线路长度/万公里\\
      \midrule[1pt]
      2011年 & 444.33 & 7.07 & 247.51 & 16.2 & 0.33\\
      2012年 & 479.66 & 8.59 & 286.75 & 21.4 & 0.33\\
      2013年 & 502.1 & 9.63 & 260.93 & 26 & 0.33\\
      2014年 & 517.31 & 10.89 & 259.92 & 29.7 & 0.33\\
      2015年 & 511.59 & 13.02 & 263.99 & 42.2 & 0.34\\
      2016年 & 516.57 & 13.89 & 289.95 & 60.2 & 0.34\\
      2017年 & 569.8 & 15.52 & 289.25 & 79.6 & 0.34\\
      2018年 & 565.82 & 17.41 & 290.48 & 82.6 & 0.09\\
      2019年 & 581.23 & 18.68 & 290.63 & 99.7 & 0.32 \\
      2020年 & 615.57 & 20.96 & 292.21 & 111.2 & 0.32 \\
      \bottomrule[1.5pt]
      \end{tabularx}
      \begin{tablenotes}
          \footnotesize
          \item[1] 数据来源:国家统计局
      \end{tablenotes}
      \end{threeparttable}
      \vspace{\baselineskip}
   \end{table}
\end{VerbWithBreak}
   \noindent\hrule
\begin{VerbWithBreak}
   这里使用了threeparttable宏包。在tablenotes环境中添加注释即可。
\end{VerbWithBreak}

若想获得绘制表格的更多信息,参见网络上的~\href{http://www.tug.org/pracjourn/2007-1/mori/}{Tables in \LaTeXe: Packages and Methods}~文档。


%% !Mode:: "TeX:UTF-8"

\chapter{数学公式的输入方法}

\section{本科生毕业论文的公式规范}

公式均需有公式号;公式号按章编排,如式(2-3);公式中各物理量及量纲均按国际标准(SI)及国家规定的法定符号和法定计量单位标注,禁止使用已废弃的符号和计量单位;公式中用字、符号、字体要符合学科规范。图、表、公式等与正文之间要有6磅的行间距。 

公式中用斜线表示“除”的关系时应采用括号,以免含糊不清,如~$a/(b\cos x)$。通常“乘”的关系在前,如~$a\cos x/b$而不写成~$(a/b)\cos x$。

不能用文字形式表示等式,如:$\textnormal{刚度}=\frac{{\textnormal{受力}}}{{\textnormal{受力方向的位移}}}$。

对于数学公式的输入方法,网络上有一个比较全面权威的文档\textbf{~\href{http://tug.ctan.org/cgi-bin/ctanPackageInformation.py?id=voss-mathmode}{Math mode}}~请大家事先大概浏览一下。下面将对学位论文中主要用到的数学公式排版形式进行阐述。

\section{生成~\LaTeX~数学公式的两种方法}

对于先前没有接触过~\LaTeX~的人来说,编写~\LaTeX~数学公式是一件很繁琐的事,尤其是对复杂的数学公式来说,更可以说是一件难以完成的任务。
实际上,生成~\LaTeX~数学公式有两种较为简便的方法,一种是基于~MathType~ 数学公式编辑器的方法,另一种是基于~MATLAB~商业数学软件的方法,
下面将分别对这两种数学公式的生成方法作一下简单介绍。

\subsection{基于~MathType~软件的数学公式生成方法}

MathType~是一款功能强大的数学公式编辑器软件,能够用来在文本环境中插入~Windows OLE~ 图形格式的复杂数学公式,所以应用比较普遍。但此软件只有~30~天的试用期,之后若再继续使用则需要付费购买才行。网络上有很多破解版的~MathType~软件可供下载免费使用,
我们推荐下载安装版本号在~6.5~之上的破解版。

以~v6.8~版本(英文版)为例~在安装好~MathType~之后,若在输入窗口中编写数学公式,复制到剪贴板上的仍然是图形格式的对象。
若希望得到可插入到~\LaTeX~编辑器中的文本格式对象,则需要对~MathType~软件做一下简单的设置:在~MathType~最上排的按钮中依次选择“Preferences
$\rightarrow$ Cut and Copy Preferences”,在弹出的对话窗中选中“MathML or TeX:”,在转换下拉框中选择“LaTeX~2.09~and later”,并将对话框最下方的两个复选框全部勾掉,点击确定。这样,再从输入窗口中复制出来的对象就是文本格式,这样就可以直接将其粘贴到~\LaTeX~
编辑器中了。按照这种方法生成的数学公式两端分别有标记~\verb|\[|~和标记~\verb|\]|~,在这两个标记之间即是真正的数学公式代码。

若希望从~MathType~输入窗口中复制出来的对象为图形格式,则只需在单击“Preferences
$\rightarrow$ Cut and Copy Preferences”弹出后的窗口中,再选中“Equation object(Windows OLE graphic)” 即可。

\subsection{基于~MATLAB~软件的数学公式生成方法}


MATLAB~是矩阵实验室(Matrix Laboratory)的简称,是美国~MathWorks~公司出品的商业数学软件。它是当今科研领域最常用的应用软件之一,
具有强大的矩阵计算、符号运算和数据可视化功能,是一种简单易用、可扩展的系统开发环境和平台。


MATLAB~中提供了一个~latex~函数,它可将符号表达式转化为~\LaTeX~数学公式的形式。其语法形式为~latex(s),其中,~s~为符号表达式,
之后再将~latex~函数的运算结果直接粘贴到~\LaTeX~编辑器中。从~\LaTeX~数学公式中可以发现,其中可能包含如下符号组合:

\begin{verbatim*}
\qquad=两个空铅(quad)宽度
\quad=一个空铅宽度
\;=5/18空铅宽度
\:=4/18空铅宽度
\,=3/18空铅宽度
\!=-3/18空铅宽度
\ =一个空格
\end{verbatim*}

所以最好将上述符号组合从数学公式中删除,从而使数学公式显得匀称美观。对于~Word~等软件的使用者来说,在我们通过~MATLAB~运算得到符号表达式形式的运算结果时,在~Word~中插入运算结果需要借助于~MathType~软件,通过在~MathType~中输入和~MATLAB~运算结果相对应的数学表达形式,之后再将~MathType~数学表达式转换为图形格式粘贴到~Word~ 中。实际上,也可以将~MATLAB~中采用~latex~函数运行的结果直接粘贴到~MathType~中,再继续上述步骤,这样可以大大节省输入公式所需要的时间。此方法在~MathType~6.5c~上验证通过,若您粘入到~MathType~中的仍然为从~MATLAB~中导入的代码,请您更新~MathType~软件。

\section{数学字体}
在数学模式下,常用的数学字体命令有如下几种:

\begin{Verbatim}[breaklines=true, breaksymbol=, breakanywheresymbolpre=]
\mathnormal或无命令 用数学字体打印文本;
\mathit             用斜体(\itshape)打印文本;
\mathbf             用粗体(\bfseries)打印文本;
\mathrm             用罗马体(\rmfamily)打印文本;
\mathsf             用无衬线字体(\sffamily)打印文本;
\mathtt             用打印机字体(\ttfamily)打印文本;
\mathcal            用书写体打印文本;
\end{Verbatim}

在学位论文撰写中,只需要用到上面提到的~\verb|\mathit|、\verb|\mathbf|~ 和~\verb|\mathrm|~命令。若要得到~Times New Roman~ 的数学字体,则需要调用~txfonts~宏包(此宏包实际上采用的是~Nimbus Roman No9 L~字体,
它是开源系统中使用的免费字体,其字符字体与~Times New Roman~字体几乎完全相同);若要得到粗体数学字体,则需要调用~bm~宏包。表~\ref{tab:fonts}~中分别列出了得到阿拉伯数字、拉丁字母和希腊字母
各种数学字体的命令。

\begin{table}[htbp]
\caption{常用数学字体命令一览}\label{tab:fonts}
\vspace{0.5em}\centering\wuhao
\begin{tabular}{llll}
\toprule
 & 阿拉伯数字\&大写希腊字母 & 大小写拉丁字母 & 小写希腊字母  \\
\midrule
斜体 & \verb|\mathit{}| & \verb|无命令| & \verb|无命令|\\
粗斜体 & \verb|\bm{\mathit{}}| & \verb|\bm{}| & \verb|\bm{}|\\
直立体 & \verb|无命令| & \verb|\mathrm{}| & \verb|字母后加up|\\
粗体 & \verb|\mathbf{}或\bm{}| & \verb|\mathbf{}| & \verb|\bm{字母后加up}|\\
\bottomrule
\end{tabular}
\vspace{\baselineskip}
\end{table}

\noindent 下面列出了一些应采用直立数学字体的数学常数和数学符号。

\vspace{-0.5em}\begin{center}\begin{tabularx}{0.7\textwidth}{XX}
$\mathrm{d}$、 $\mathrm{D}$、 $\mathrm{p}$~———微分算子 & $\mathrm{e}$~———自然对数之底数\\
$\mathrm{i}$、 $\mathrm{j}$~———虚数单位 & $\piup$———圆周率\\
\end{tabularx}\end{center}

\section{行内公式}
出现在正文一行之内的公式称为行内公式,例如~$f(x)=\int_{a}^{b}\frac{\sin{x}}{x}\mathrm{d}x$。对于非矩阵和非多行形式的行内公式,一般不会使得行距发生变化,而~Word~等软件却会根据行内公式的竖直距离而自动调节行距,如图~\ref{fig:hangju}~所示。

\begin{figure}[htbp]
\centering
\subfigure[由~\LaTeX~系统生成的行内公式]{\label{fig:subfig:latex}
                \fbox{\includegraphics[width=0.55\textwidth]{latex}}}
\subfigure[由~Word软件生成的~.doc~格式行内公式]{\label{fig:subfig:word}
                \fbox{\includegraphics[width=0.55\textwidth]{word}}}
\subfigure[由~Word软件生成的~.pdf~格式行内公式]{\label{fig:subfig:pdf}
                \fbox{\includegraphics[width=0.55\textwidth]{pdf}}}

\caption{由~\LaTeX~和~Word~生成的~3~种行内公式屏显效果}\label{fig:hangju}
\vspace{-1em}
\end{figure}

这三幅图分别为~\LaTeX~和~Word~生成的行内公式屏显效果,从图中可看出,在~\LaTeX~文本含有公式的行内,在正文与公式之间对接工整,行距不变;而在~Word~文本含有公式的行内,在正文与公式之间对接不齐,行距变大。因此从这一点来说,
\LaTeX~系统在数学公式的排版上具有很大优势。

\LaTeX~提供的行内公式最简单、最有效的方法是采用~\TeX~本来的标记———开始和结束标记都写作~\$,例如本段开始的例子可由下面的输入得到。
\verb|$f(x)=\int_{a}^{b}\frac{\sin{x}}{x}\mathrm{d}x$|

\section{行间公式}
位于两行之间的公式称为行间公式,每个公式都是一个单独的段落,例如
\[\int_a^b{f\left(x\right)\mathrm{d}x}=\lim_{\left\|\Delta{x_i}\right\|\to 0}\sum_i{f\left(\xi_i\right)\Delta{x_i}}\]
除人工编号外,\LaTeX~各种类型行间公式的标记见表~\ref{tab:eqtag}。
\begin{table}[htbp]
\caption{各种类型行间公式的标记}\label{tab:eqtag}
\vspace{0.5em}\centering\wuhao
\begin{tabularx}{\textwidth}{cll}
\toprule
& 无编号 & 自动编号\\
\midrule
单行公式& \verb|\begin{displaymath}... \end{displaymath}|& \verb|\begin{equation}... \end{equation}|\\
        & 或~\verb|\[...\]| & \\
多行公式& \verb|\begin{eqnarray*}... \end{eqnarray*}|& \verb|\begin{eqnarray}... \end{eqnarray}|\\
\bottomrule
\end{tabularx}
\end{table}

另外,在自动编号的某行公式行尾添加标签~\verb|\nonumber|,可将该行转换为无编号形式。

行间多行公式需采用~\verb|eqnarray|~或~\verb|eqnarray*|~环境,它默认是一个列格式为~\verb|rcl|~的~3~列矩阵,并且中间列的字号要小一些,因此通常只将需要对齐的运算符号(通常为等号“=”)置于中间列。

\section{可自动调整大小的定界符}

若在左右两个定界符之前分别添加命令~\verb|\left|~和~\verb|\right|,则定界符可根据所包围公式大小自动调整其尺寸,这可从式(\ref{nodelimiter}) 和式(\ref{delimiter})中看出。
\begin{equation}\label{nodelimiter}
(\sum_{k=\frac12}^{N^2})
\end{equation}
\begin{equation}\label{delimiter}
\left(\sum_{k=\frac12}^{N^2}\right)
\end{equation}
式(\ref{nodelimiter})和式(\ref{delimiter})是在~\LaTeX~中分别输入如下代码得到的。
\begin{Verbatim}[breaklines=true, breaksymbol=, breakanywheresymbolpre=]
(\sum_{k=\frac12}^{N^2})
\left(\sum_{k=\frac12}^{N^2}\right)
\end{Verbatim}
\verb|\left|~和~\verb|\right|~总是成对出现的,若只需在公式一侧有可自动调整大小的定界符,则只要用“.”代替另一侧那个无需打印出来的定界符即可。
若想获得关于此部分内容的更多信息,可参见~\href{http://tug.ctan.org/cgi-bin/ctanPackageInformation.py?id=voss-mathmode}{Math mode}~文档的第~8~章“Brackets, braces and parentheses”。

\section{数学重音符号}
数学重音符号通常用来区分同一字母表示的不同变量,输入方法如下(需要调用~\verb|amsmath|~宏包):

\vspace{0.5em}\noindent\wuhao\begin{tabularx}{\textwidth}{Xc|Xc|Xc}
 \verb|\acute| & $\acute{a}$ & \verb|\mathring| & $\mathring{a}$ & \verb|\underbrace| & $\underbrace{a}$ \\
 \verb|\bar| & $\bar{a}$ & \verb|\overbrace| & $\overbrace{a}$ & \verb|\underleftarrow| & $\underleftarrow{a}$ \\
 \verb|\breve| & $\breve{a}$ & \verb|\overleftarrow| & $\overleftarrow{a}$ & \verb|\underleftrightarrow| & $\underleftrightarrow{a}$ \\
 \verb|\check| & $\check{a}$ & \verb|\overleftrightarrow| & $\overleftrightarrow{a}$ & \verb|\underline| & $\underline{a}$ \\
 \verb|\dddot| & $\dddot{a}$ & \verb|\overline| & $\overline{a}$ & \verb|\underrightarrow| & $\underrightarrow{a}$ \\
 \verb|\ddot| & $\ddot{a}$ & \verb|\overrightarrow| & $\overrightarrow{a}$ & \verb|\vec| & $\vec{a}$ \\
 \verb|\dot| & $\dot{a}$ & \verb|\tilde| & $\tilde{a}$ & \verb|\widehat| & $\widehat{a}$ \\
 \verb|\grave| & $\grave{a}$ & \verb|\underbar| & $\underbar{a}$ & \verb|\widetilde| & $\widetilde{a}$ \\
 \verb|\hat| & $\hat{a}$
\end{tabularx}\vspace{0.5em}

\xiaosi 当需要在字母~$i$~和~$j$~的上方添加重音符号时,为了去掉这两个字母顶上的小点,这两个字母应该分别改用~\verb|\imath|~ 和~\verb|\jmath|。

如果遇到某些符号不知道该采用什么命令能输出它时,则可通过~\href{http://detexify.kirelabs.org/classify.html}{Detexify$^2$~ 网站}来获取符号命令。若用鼠标左键在此网页的方框区域内画出你所要找的符号形状,则会在网页右方列出和你所画符号形状相近的~5~个符号及其相对应的~\LaTeX~输入命令。若所列出的符号中不包括你所要找的符号,还可通过点击“Select from the complete list!”的链接以得分从低到高的顺序列出所有符号及其相对应的~\LaTeX~输入命令。

最后,建议大家还以~\href{http://tug.ctan.org/cgi-bin/ctanPackageInformation.py?id=voss-mathmode}{Math mode}~这篇~pdf~文档作为主要参考。若要获得最为标准、美观的数学公式排版形式,可以查查文档中是否有和你所要的排版形式相同或相近的代码段,通过修改代码段以获得你所要的数学公式排版形式。如果出现问题,我们推荐大家查阅~ChinaTeXMathFAQ~V1.1。

%% !Mode:: "TeX:UTF-8"

\chapter{罗列、定理和代码环境使用方法}

\section{单层罗列环境}

海南大学学位论文一般可采用两种罗列环境:一种是并列条目有同样标签的~\verb|itemize|~罗列环境,另一种是具有自动排序编号符号的~\verb|enumerate|~罗列环境。这两种罗列环境的样式参数可参考图~\ref{fig:list}。
\begin{figure}[htbp]
\centering
\includegraphics[width = 0.6\textwidth]{list}
\caption{罗列环境参数示意图}\label{fig:list}\vspace{-1em}
\end{figure}

通过调用~enumitem~宏包可以很方便地控制罗列环境的布局,其~format.tex~文件中的~\verb|\setitemize|~和~\verb|\setenumerate|~命令分别用来设置~\verb|itemize|~和~\verb|enumerate|~环境的样式参数。采用~\verb|itemize|~单层罗列环境的排版形式如下:

\begin{itemize}
\item 第一个条目文本内容
\item 第二个条目文本内容
\item 第三个条目文本内容
\end{itemize}

其代码如下

\begin{Verbatim}[breaklines=true, breaksymbol=, breakanywheresymbolpre=]
\begin{itemize}
  \item 第一个条目文本内容
  \item 第二个条目文本内容
  ...
  \item 第三个条目文本内容
\end{itemize}
\end{Verbatim}

采用~\verb|enumerate|~单层罗列环境的排版形式如下:

\begin{enumerate}
\item 第一个条目文本内容
\item 第二个条目文本内容
\item 第三个条目文本内容
\end{enumerate}

其代码如下

\begin{Verbatim}[breaklines=true, breaksymbol=, breakanywheresymbolpre=]
\begin{enumerate}
  \item 第一个条目文本内容
  \item 第二个条目文本内容
  ...
  \item 第三个条目文本内容
\end{enumerate}
\end{Verbatim}



\section{定理环境}

\begin{definition}[谱半径]\label{def:def1}
  称~$n$~阶方阵~$\mathbf{A}$~的全体特征值~$\lambda_1,\cdots,\lambda_n$~组成的集合为~$\mathbf{A}$~的谱,称
  $$\rho(\mathbf{A})=\max{\{|\lambda_1|,\cdots,|\lambda_n|\}}$$
\end{definition}
\begin{theorem}[相似充要条件]\label{lemma:l1}
  方阵$A$和$B$相似的充要条件是:~$A$~和~$B$~有全同的不变因子。
\end{theorem}
\begin{corollary}[推论1]\label{cor:cor1}
在赋范空间~$(X,\|\cdot\|)$~上定义~$d(x,y)=\|x-y\|$, 对任意~$x,y\in X$,~则~$(X,d)$~是距离空间。
\end{corollary}
\begin{proof}
  只需证明~$d(x,y)$~是距离。
\end{proof}
\newpage

定义代码如下:
\begin{Verbatim}[breaklines=true, breaksymbol=, breakanywheresymbolpre=]
 \begin{definition}[谱半径]\label{def:def1}
  称~$n$~阶方阵~$\mathbf{A}$~的全体特征值
  $\lambda_1,\cdots,\lambda_n$组成的集合为~$\mathbf{A}$~的谱,称
  $$\rho(\mathbf{A})=\max{\{|\lambda_1|,\cdots,|\lambda_n|\}}$$
\end{definition}
\end{Verbatim}
\noindent\hrule

\vspace{0.1em}\noindent\hrule
\vspace{1em}
定理代码如下:
\begin{Verbatim}[breaklines=true, breaksymbol=, breakanywheresymbolpre=]
\begin{theorem}[相似充要条件]\label{lemma:l1}
  方阵$A$和$B$相似的充要条件是:$A$和$B$有全同的不变因子。
\end{theorem}
\end{Verbatim}

\noindent\hrule\vspace{0.1em}

\noindent\hrule
\vspace{1em}
推论和证明代码如下:
\begin{Verbatim}[breaklines=true, breaksymbol=, breakanywheresymbolpre=]
\begin{corollary}[推论1]\label{cor:cor1}
在赋范空间~$(X,\|\cdot\|)$~上定义$d(x,y)=\|x-y\|$,
对任意$x,y\in X$,则$(X,d)$是距离空间。
\end{corollary}
\begin{proof}
  只需证明$d(x,y)$是距离。
\end{proof}
\end{Verbatim}
\noindent\hrule\vspace{1em}

定理定义[]中是可选参数,用来说明定理的名称。其他环境格式书写与上面定理、定义、推论格式相同,可自己调用其他环境。
若需要书写定理定义等内容,而且带有顺序编号,需要采用如下环境。除了~\verb|proof|~环境之外,其余~9~个环境都可以有一个可选参数作为附加标题。

\begin{center}
\vspace{0.5em}\noindent\wuhao\begin{tabularx}{0.7\textwidth}{lX|lX}
定理 & \verb|theorem|~环境 & 定义 & \verb|definition|~环境 \\
例 & \verb|example|~环境 & 算法 & \verb|algorithm|~环境 \\
公理 & \verb|axiom|~环境 & 命题 & \verb|proposition|~环境 \\
引理 & \verb|lemma|~环境 & 推论 & \verb|corollary|~环境 \\
注解 & \verb|remark|~环境 & 证明 & \verb|proof|~环境 \\
\end{tabularx}
\end{center}
\section{代码环境}
很多和计算机专业背景相关的同学都会使用到代码环境,使用~\verb|\verb|~指令或者是~\verb|verbatim|~环境固然是一种选择,但是比不上专门的~lstlisting~环境这么专业。
\begin{lstlisting}
int main(int argc, char ** argv)
{
printf("Hello world!\n");
return 0;
}
\end{lstlisting}

\noindent\hrule
\vspace{0.1em}\noindent\hrule

\vspace{1em}

\noindent 代码如下:
\begin{Verbatim}[breaklines=true, breaksymbol=, breakanywheresymbolpre=]
\begin{lstlisting}
int main(int argc, char ** argv)
{
printf("Hello world!\n");
return 0;
}
\end{lstlisting}
\end{Verbatim}
\noindent\hrule\vspace{1em}

在代码中显示的关键字为蓝色,框的左侧显示的是行号,这样便于读者阅读和查找代码,同时添加了浅蓝色的阴影边框,达到了美观的效果。
代码环境的设置已在~package~中~\verb|\lstset|~指令中定义。如果要修改要插入的代码的语言直接将language定义为要插入的语言即可,如插入Java那么language=Java即可。定义中支持跨页显示,可以将较长的代码置于~lstlisting~环境中。
\section{算法环境}
很多和计算机专业背景相关的同学会使用到算法环境,之前使用到的定理环境固然是一种选择,但是比不上专门的~algorithm2e~环境这么专业。为了实现专业和接近完美,本版本支持算法环境。如下所示:
\begin{algorithm}[H]
    \caption{算法标题}
    \label{alg:demoAlgo} % 贴上标签以便交叉引用
    \begin{algorithmic}[1]  % 这个 1 表示每一行都显示数字
    \STATE 初始化...
    \FOR{$i=0;i\le M; i\rightarrow i + 1$}
        \STATE 执行语句~1;
        \STATE 执行语句~2;
        \STATE ...
    \ENDFOR
    \STATE ...
    \WHILE{某条件}
        \STATE 执行语句~1;
        \STATE 执行语句~2;
        \STATE ...
    \ENDWHILE
    \STATE ...
    \end{algorithmic}
\end{algorithm}
\noindent\hrule
\vspace{0.1em}\noindent\hrule

\vspace{1em}

\noindent 代码如下:
\begin{Verbatim}[breaklines=true, breaksymbol=, breakanywheresymbolpre=]
  \begin{algorithm}[H]
    \caption{算法标题}
    \label{alg:demoAlgo} % 贴上标签以便交叉引用
    \begin{algorithmic}[1]  % 这个 1 表示每一行都显示数字
    \STATE 初始化...
    \FOR{$i=0;i\le M; i\rightarrow i + 1$}
        \STATE 执行语句~1;
        \STATE 执行语句~2;
        \STATE ...
    \ENDFOR
    \STATE ...
    \WHILE{某条件}
        \STATE 执行语句~1;
        \STATE 执行语句~2;
        \STATE ...
    \ENDWHILE
    \STATE ...
    \end{algorithmic}
\end{algorithm}
\end{Verbatim}


%\include{body/conclusion}
\end{Verbatim}
那么,编译的时候就只编译未加~\%~的一章,在这个例子中,即本章~intros。

理论上,并不一定要把每章放在不同的文件中。但是这种自顶向下,分章节写作、编译的方法有利于提高效率,大大减少~Debug~过程中的编译时间,同时减小风险。

\section{参考文献生成和标注方法}

\LaTeX~具有插入参考文献的能力。Google Scholar~网站上存在兼容~BibTeX~的参考文献信息,通过以下几个步骤,可以轻松完成参考文献的生成。
\begin{itemize}
  \item 在\href{http://scholar.google.com/}{谷歌学术搜索}中,
        点击\href{http://scholar.google.com/scholar_preferences?hl=en&as_sdt=0,5}{学术搜索设置}。
  \item 页面打开之后,在\textbf{文献管理软件}选项中选择\textbf{显示导入~BibTeX~的链接},单击保存设置,退出。
  \item 在谷歌学术搜索中检索到文献后,在文献条目区域单击导入~BibTeX~选项,页面中出现文献的引用信息。
  \item 将文献引用信息的内容复制之后,添加到~references~文件夹下的~reference.bib~ 中。
\end{itemize}
\par{如在scholar中搜索“基于方差及方差梯度的指纹图像自适应分割算法”,那么其BibTex文件如下:
 "@article\{樊冬进2008基于方差及方差梯度的指纹图像自适应分割算法,
  title=\{基于方差及方差梯度的指纹图像自适应分割算法\},
  author=\{樊冬进 and 孙冰 and 封举富 and others\},
  year=\{2008\}
\}"
直接使用会报错,我们需要将”@article\{樊冬进2008基于方差及方差梯度的指纹图像自适应分割算法“改为@article\{anyEnglishName

}


在正文中标注参考文献时,在需要标注的地方输入~\verb|\cite{}|~指令,花括号内输入参考文献引用信息中的第一行信息即可(常常为文献的缩略信息),此时~\verb|[]|~符号在标注处的右上角显示。

\section{注意事项}

\begin{enumerate}
  \item 由于模板使用~UTF-8~编码,所以源文件应该保存成~UTF-8~格式,否则可能出现中文字符无法识别的错误。
  本模板中每一个~.tex~文件的文件的开头已经加上一行:\\
  \verb|% !Mode:: "TeX:UTF-8"|\\
     这样可以确保~.tex~文件默认使用~UTF-8~的格式打开。读者如果删去此行,很有可能会导致中文字符显示乱码。
     在~WinEdt~编辑器中可以使用以下两种方式保存成~UTF-8~格式:
      \begin{enumerate}
        \item 先建立~.tex~文件,另存为~.tex~文件时,选择用~UTF-8~ 格式保存。
        \item
            在~WinEdt~编辑器中,选择\\
            \mbox{~Document$\rightarrow$Document Settings$\rightarrow$Document Mode $\rightarrow$TeX:UTF-8} 同时在~WinEdt~最下面的状态栏中,可以看到该文档是~TeX~ 格式还是~TeX:UTF-8~格式。
            当文档为~TeX:UTF-8~格式时,状态栏一般显示:
            \makebox[\textwidth][l]{Wrap | Indent | INS | LINE |Spell | TeX:UTF-8 | -src~ 等。}
      \end{enumerate}
  \item 如果在~pdf~书签中,中文显示乱码的话,则注意以下说明:
    \begin{Verbatim}[breaklines=true, breaksymbol=, breakanywheresymbolpre=]
        \usepackage{CJKutf8}
        % 1. 如果使用CJKutf8
        %    Hyperref中应使用unicode参数
        % 2. 如果使用CJK
        %    Hyperref则使用CJKbookmarks参数
        %    可惜得到的PDF书签是乱码,建议弃用
        % 3. Unicode选项和CJKbookmarks不能同时使用
        \usepackage[
        %CJKbookmarks=true,
        unicode=true
        ]{hyperref}
     \end{Verbatim}
  \item 建议采用以下两种编译方式:
  \begin{enumerate}
    \item latex + bibtex + latex + latex + dvi2pdf. 在这种编译情况下,对应的~hnumain.tex~文件的第一行是\verb|\def\usewhat{dvipdfmx}|~ (缺省设置)。 此时,所有图片文件应该保存为~.eps~格式,如~figures~文件夹里~.eps~图片。
          如果您选择在命令行中操作,可以在编译的时候依次输入~latex hnumain, bibtex hnumain, latex hnumain, latex hnumain~和~dvipdfmx hnumain, 编译完成之后,需要手动打开~pdf~文件。需要说明的是,为了是操作简便,以上命令已经作为~pdfmake.bat~批处理文件放在目录中。在编译无误的前提下,双击此文件,可以一键生成~pdf~。

    \item pdflatex + pdflatex. 在这种编译情况下,对应的~hnumain.tex~文件的第一行应该改为\verb|\def\usewhat{pdflatex}|~。 此时, 编译不支持~.eps~图片格式,此时需要在命令行下使用~epstopdf~指令将~figures~文件夹下 的~.eps~文件转化成~.pdf~ 文件格式,命令行中操作格式为~epstopdf a.eps~。
          在命令行编译的时候,依次输入~pdflatex hnumain~和~pdflatex hnumain, 编译完成之后,需要手动打开~pdf~文件。
  \end{enumerate}
    \item  当参考文献在编辑的时候,第一行为标签行,为该文献的缩略信息。当复制中文参考文献的~BibTeX~页面到~reference.bib~文件中时,需要把原来含有中文的标签行改成英文书写,否则会报错。
\end{enumerate}

\section{系统要求}

     CTeX 2.8, MiKTeX 2.8, TeX Live 2009~或以上版本。我们推荐您使用最新的~CTeX~中文套装,~CTeX 2.9.2.164~Full~版本,内含~WinEdt 7.0~编辑器,可以完成文件的编辑和编译工作。本模板目前只在~Windows~操作系统下测试通过,尚未在~Mac~ 系统和~Linux~系统下测试。

\section{Why~\LaTeX~?}

选择使用~\TeX/\LaTeX~的理由包括:
\begin{itemize}
\item 免费软件;
\item 专业的排版效果;
\item 是事实上的专业数学排版标准;
\item 广泛的西文期刊接收甚或只接收~\LaTeX~格式的投稿;
\item[] ……
\end{itemize}
不选择使用~\TeX/\LaTeX~的理由包括:
\begin{itemize}
\item 需要相当精力学习;
\item 图文混合排版能力不够强;
\item 对于表格的支持较差;
\item 仅在数学、物理、计算机等领域流行;
\item 中文期刊的支持较差;
\item[] ……
\end{itemize}

如果想知道~\LaTeX~与~Word~的详细区别,请参见:~\url{http://zzg34b.w3.c361.com/homepage/compareWord.htm}。

\subsection{我应该看什么~\LaTeX~读物~?}

我觉得最好在使用中学会~\LaTeX~,至于参考资料百度、Google上很多。如果实在需要读物,那么可以在以下几本书上进行选择。

\begin{enumerate}
\item 我能阅读英文
\begin{enumerate}
\item 迅速入门:lshort.pdf (中文版名为:一份不太简短的~\LaTeX{}~介绍)
\item 系统学习:A Guide to LaTeX, 4th Edition, Addison-Wesley
                机械工业出版社的有影印版,名曰~《\LaTeX{}~ 实用教程》
\item 深入学习:Knuth 《TeXbook》:必读。 《LATEX Companion》:如果说 高老头的~TeXbook~是论语,那么这本
               书算是一本史记,全面而精妙,是所有~\LaTeX~书中的精品。
\end{enumerate}

\item 我更愿意阅读中文
\begin{enumerate}
\item 迅速入门:lnotes2.pdf (\LaTeX~Notes~雷太赫排版系统简介, v2.0, 包太雷)
\item 系统学习:《\LaTeXe{}~科技排版指南》,邓建松(电子版)
      如果不好找,可以阅读陈志杰等《\LaTeXe~入门与提高》第二版,或者胡伟《\LaTeXe~完全学习手册》
\item 深入学习:~TeXbook0.pdf~(特可爱原本,TeXbook~的中译,xianxian)
\item 具体问题释疑:~CTeX-FAQ.pdf~,\\
        吴凌云,~\url{http://www.ctex.org/CTeXFAQ}~\\
      ~ChinaTeXMathFAQ~$V1.1$~~China\TeX~ 数学排版常见问题集
\end{enumerate}
\end{enumerate}

遇见问题和解决问题的过程可以快速提高自己的技能,建议此时:
\begin{itemize}
 \item 清楚,扼要地提出你的问题。
 \item 使用~Google~搜索。
\end{itemize}

\section{免责声明}

本模板依据《本科生毕业论文(设计)封面》和《毕业论文(设计)排版打印格式》编写,作者希望能给使用者写作论文带来方便。然而,作者不保证本模板完全符合学校要求,也不对由此带来的风险和损失承担任何责任。

%% !Mode:: "TeX:UTF-8"

\chapter{图片的插入方法}

\section{海南大学对于插图的要求}

 \par{插图要求,所有插图按分章编号,如第1章的第1张插图为“图1-1 ****图”,字体可以用宋体5号字。所有插图均需有图注(图的说明),不能只有图编号。图号及图注应在图的下方居中标出;一幅图如有若干幅分图,均应编分图号,用(a),(b),(c)......按顺序编排;插图须紧跟文述,在正文中,一般应先见图号及图的内容后再见图(即:正文中先见“本系统功能架构如图3-2所示。”字样,再在段后见图3-2),一般情况下不能提前见图,特殊情况需延后的插图不应跨节。}
\par{图形符号及各种线型画法须按照现行的国家标准;坐标图中坐标上须注明标度值,并标明坐标轴所表示的物理量名称及量纲,应均按国际标准(SD)标注,例如:kW,  m/s, N, m....等,但对一些示意图例外;图应具有“自明性”(即只看图、图题和图例,不阅读正文,就可理解图意);图中用字最小为宋体小五号字;插图必须同内容密切联系,切忌与文字和表重复。}
\par{图的绘制一定要紧凑美观并且保证清晰,避免使用彩色和带底色的图,以免影响论文印刷。}
\par{切忌直接从他人文章、其他文献、书籍扫描和网上直接拷贝图形,全文不能出现非本人绘制的图形(系统实现部分的系统界面截图除外),必须使用他人插图时,须在图题正下方注明出处。
}

\section{\LaTeX~中推荐使用的图片格式}

在~\LaTeX~中应用最多的图片格式是~EPS(Encapsulated PostScript)格式,它是一种专用的打印机描述语言,常用于印刷或打印输出。
EPS~格式图片可通过多种方式生成在这里可以采用Adobe Acrobat 软件进行转换。选中图片右击转换为pdf然后,选择保存,在保存格式下面选择文件的格式为eps即可。,对于用visio写的流程图,其转换格式和普通图片转换成eps格式完全相同。

\section{单张图片的插入方法}
单张图片独自占一行的插入形式如图~\ref{fig:xml}~所示。
\begin{figure}[htbp]
\centering
\includegraphics[width=0.4\textwidth]{XML}
\caption{树状结构}\label{fig:xml}
\vspace{\baselineskip}
\end{figure}


其插入图片的代码及其说明如下。
\vspace{1em}\noindent\hrule
\begin{Verbatim}[breaklines=true, breaksymbol=, breakanywheresymbolpre=]
\begin{figure}[htbp]
\centering
\includegraphics[width=0.4\textwidth]{文件名(.eps)}
\caption{标题}\label{标签名(通常为 fig:labelname)}
\vspace{\baselineskip} % 表示图与正文空一行
\end{figure}
\end{Verbatim}

\noindent\hrule

\begin{Verbatim}[breaklines=true, breaksymbol=, breakanywheresymbolpre=]
figure环境的可选参数[htbp]表示浮动图形所放置的位置,h (here)表示当前位置,t (top)表示页芯顶部,b (bottom)表示页芯底部,p (page)表示单独一页。在Word等软件中,图片通常插入到当前位置,如果当前页的剩余空间不够,图片将被移动到下一页,当前页就会出现很大的空白,其人工调整工作非常不便。由LaTeX提供的浮动图片功能,总是会按h->t->b->p的次序处理选项中的字母,自动调整图片的位置,大大减轻了工作量。
\centering命令将后续内容转换成每行皆居中的格式。
"\includegraphics"的可选参数用来设置图片插入文中的水平宽度,一般表示为正文宽度(\textwidth)的倍数。
\caption命令可选参数“标签名”为英文形式,一般不以图片或表格的数字顺序作为标签,而应包含一定的图片或表格信息,以便于文中引用(若图片、表格、公式、章节和参考文献等在文中出现的先后顺序发生了变化,其标注序号及其文中引用序号也会跟着发生变化,这一点是Word等软件所不能做到的)。另外,图题或表题并不会因为分页而与图片或表格体分置于两页,章节等各级标题也不会置于某页的最底部,LaTeX系统会自动调整它们在正文中的位置,这也是Word等软件所无法匹敌的。
\vspace将产生一定高度的竖直空白,必选参数为负值表示将后续文字位置向上提升,参数值可自行调整。em为长度单位,相当于大写字母M的宽度。\vspace{\baselineskip} 表示图与正文空一行。
引用方法:“见图~\ref{fig:figname}”、“如图~\ref{fig:figname}~所示”等。
\end{Verbatim}

\noindent\hrule\vspace{1em}

若需要将~2~张及以上的图片并排插入到一行中,则需要采用\verb|minipage|环境,如图~\ref{fig:dd}~和图~\ref{fig:ds}~所示。
\begin{figure}[htbp]
\centering
\begin{minipage}{0.4\textwidth}
\centering
\includegraphics[width=\textwidth]{dataDimensions}
\caption{数据维数的变化}\label{fig:dd}
\end{minipage}
\begin{minipage}{0.4\textwidth}
\centering
\includegraphics[width=\textwidth]{dataSize}
\caption{数据规模的变化}\label{fig:ds}
\end{minipage}
\vspace{\baselineskip}
\end{figure}

其代码如下所示。
\vspace{1em}\noindent\hrule
\begin{Verbatim}[breaklines=true, breaksymbol=, breakanywheresymbolpre=]
\begin{figure}[htbp]
\centering
\begin{minipage}{0.4\textwidth}
\centering
\includegraphics[width=\textwidth]{文件名}
\caption{标题}\label{fig:f1}
\end{minipage}
\begin{minipage}{0.4\textwidth}
\centering
\includegraphics[width=\textwidth]{文件名}
\caption{标题}\label{fig:f2}
\end{minipage}\vspace{\baselineskip}
\end{figure}
\end{Verbatim}

\noindent\hrule

\begin{Verbatim}[breaklines=true, breaksymbol=, breakanywheresymbolpre=]
minipage环境的必选参数用来设置小页的宽度,若需要在一行中插入n个等宽图片,则每个小页的宽度应略小于(1/n)\textwidth。
\end{Verbatim}

\noindent\hrule

\section{具有子图的图片插入方法}

图中若含有子图时,需要调用~subfigure~宏包, 如图~\ref{fig:subfig}~所示。
\begin{figure}[htbp]
  \centering
  \subfigure[Data Dimensions]{\label{fig:subfig:datadim}
                \includegraphics[width=0.4\textwidth]{dataDimensions}}
  \subfigure[Data Size]{\label{fig:subfig:datasize}
                \includegraphics[width=0.4\textwidth]{dataSize}}
  \caption{Scalability of data}\label{fig:subfig}
\vspace{\baselineskip}
\end{figure}

其代码及其说明如下。
\vspace{1em}\noindent\hrule

\begin{Verbatim}[breaklines=true, breaksymbol=, breakanywheresymbolpre=]
\begin{figure}[htbp]
  \centering
  \subfigure[第1个子图标题]{
            \label{第1个子图标签(通常为 fig:subfig1:subsubfig1)}
            \includegraphics[width=0.4\textwidth]{文件名}}
  \subfigure[第2个子图标题]{
            \label{第2个子图标签(通常为 fig:subfig1:subsubfig2)}
            \includegraphics[width=0.4\textwidth]{文件名}}
  \caption{总标题}\label{总标签(通常为 fig:subfig1)}
\vspace{\baselineskip}
\end{figure}
\end{Verbatim}

\noindent\hrule

\begin{Verbatim}[breaklines=true, breaksymbol=, breakanywheresymbolpre=]
子图的标签实际上可以随意设定,只要不重复就行。但为了更好的可读性,我们建议fig:subfig:subsubfig格式命名,这样我们从标签名就可以知道这是一个子图引用。
引用方法:总图的引用方法同本章第1节,子图的引用方法用\ref{fig:subfig:subsubfig}来代替。
\end{Verbatim}

\noindent\hrule\vspace{1em}

子图的引用示例:如图~\ref{fig:subfig:datadim}~和图~\ref{fig:subfig:datasize}~所示。

若想获得插图方法的更多信息,参见网络上的~\href{ftp://ftp.tex.ac.uk/tex-archive/info/epslatex.pdf}{Using Imported Graphics in \LaTeX and pdf\LaTeX}~文档。

%% !Mode:: "TeX:UTF-8"

\chapter{表格的绘制方法}

\section{本科生毕业论文的绘表规范}

表中内容应与叙述文字内容相呼应,表的结构应简洁明了,表随文排,字体为宋体5号字注,一定使用单倍行距,要排版紧凑,以与正文内容区别。对于文中的各类表,一定注意先有引用文字(格式一般为“如表8-5所示”)然后见到表,插表表名要简明贴切,表序按章用阿拉伯字编列,表序末和表名末均不加标点符号,写在表的上方。表格较长如需转页,则在下页稿纸上重写表头,并在表的右上方写“续表”,表内全部数据的统一计数或计量单位应置于表的右上角,若表中各栏计量单位不同,则将单位分别列入表头的各栏中,将量的符号与单位符号之间用斜线隔开,即表中的数值用量与单位的比值形式表示。表内数据对应位上下对齐,一般以小数点为准;数字间夹有“—”,“/”号者,以这些符号对齐;无数据或文字处一律空白。相邻栏内数字相同时,应重复书写,勿用“同左”、“同上”等;表内文字说明,空一格起行,转行顶格,并正确使用标点符号,但每段最后一律不用标点符号。表内名词短语、数据需注释时,用脚注,即在所需加注名词或数据的右上角注符号“①,②,…”或星号,在表的底线下方写出相应的符号和注文,不出现“注”字,如对整个表加以说明时,可附注于底线下方,注文前应有“说明:”字样。

\section{普通表格的绘制方法}

表格应具有三线表格式,因此需要调用~booktabs~宏包,其标准格式如表~\ref{tab:table1}~所示。
\begin{table}[htbp]
\caption{符合本科生毕业论文绘图规范的表格}\label{tab:table1}
\vspace{0.5em}\centering\wuhao
\begin{tabular}{ccccc}
\toprule[1.5pt]
$D$(in) & $P_u$(lbs) & $u_u$(in) & $\beta$ & $G_f$(psi.in)\\
\midrule[1pt]
 5 & 269.8 & 0.000674 & 1.79 & 0.04089\\
10 & 421.0 & 0.001035 & 3.59 & 0.04089\\
20 & 640.2 & 0.001565 & 7.18 & 0.04089\\
 5 & 269.8 & 0.000674 & 1.79 & 0.04089\\
10 & 421.0 & 0.001035 & 3.59 & 0.04089\\
20 & 640.2 & 0.001565 & 7.18 & 0.04089\\
 5 & 269.8 & 0.000674 & 1.79 & 0.04089\\
10 & 421.0 & 0.001035 & 3.59 & 0.04089\\
20 & 640.2 & 0.001565 & 7.18 & 0.04089\\
 5 & 269.8 & 0.000674 & 1.79 & 0.04089\\
10 & 421.0 & 0.001035 & 3.59 & 0.04089\\
20 & 640.2 & 0.001565 & 7.18 & 0.04089\\
\bottomrule[1.5pt]
\end{tabular}
\vspace{\baselineskip}
\end{table}

其绘制表格的代码及其说明如下。
\vspace{1em}\noindent\hrule

\begin{VerbWithBreak}
\begin{table}[htbp]
\caption{表标题}\label{标签名(通常为 tab:tablename)}
\vspace{0.5em}\centering\wuhao
\begin{tabular}{cc...c}
\toprule[1.5pt]
表头第1个格   & 表头第2个格   & ... & 表头第n个格  \\
\midrule[1pt]
表中数据(1,1) & 表中数据(1,2) & ... & 表中数据(1,n)\\
表中数据(2,1) & 表中数据(2,2) & ... & 表中数据(2,n)\\
表中数据(3,1) & 表中数据(3,2) & ... & 表中数据(3,n)\\
表中数据(4,1) & 表中数据(4,2) & ... & 表中数据(4,n)\\
...................................................\\
表中数据(m,1) & 表中数据(m,2) & ... & 表中数据(m,n)\\
\bottomrule[1.5pt]
\end{tabular}
\vspace{\baselineskip}
\end{table}
\end{VerbWithBreak}

\noindent\hrule

\begin{VerbWithBreak}
table环境是一个将表格嵌入文本的浮动环境。
\wuhao命令将表格的字号设置为五号字(10.5pt),在绘制表格结束退出时,不需要将字号再改回为\xiaosi,正文字号默认为小四号字(12pt)。
tabular环境的必选参数由每列对应一个格式字符所组成:c表示居中,l表示左对齐,r表示右对齐,其总个数应与表的列数相同。此外,@{文本}可以出现在任意两个上述的列格式之间,其中的文本将被插入每一行的同一位置。表格的各行以\\分隔,同一行的各列则以&分隔。
\toprule、\midrule和\bottomrule三个命令是由booktabs宏包提供的,其中\toprule和\bottomrule分别用来绘制表格的第一条(表格最顶部)和第三条(表格最底部)水平线,\midrule用来绘制第二条(表头之下)水平线,且第一条和第三条水平线的线宽为1.5pt,第二条水平线的线宽为1pt。
引用方法:“如表~\ref{tab:tablename}~所示”。
\end{VerbWithBreak}

\noindent\hrule

\section{长表格的绘制方法}

长表格是当表格在当前页排不下而需要转页接排的情况下所采用的一种表格环境。若长表格仍按照普通表格的绘制方法来获得,
其所使用的~\verb|table|~浮动环境无法实现表格的换页接排功能,表格下方过长部分会排在表格第~1~页的页脚以下。为了能够实现长表格的转页接排功能,
需要调用~\verb|longtable|~宏包,由于长表格是跨页的文本内容,因此只需要单独的~\verb|longtable|~环境,所绘制的长表格的格式如表~\ref{tab:table2}~所示。

此长表格~\ref{tab:table2}~第~2~页的标题“编号(续表)”和表头是通过代码自动添加上去的,无需人工添加,若表格在页面中的竖直位置发生了变化,长表格在第~2~页
及之后各页的标题和表头位置能够始终处于各页的最顶部,也无需人工调整,\LaTeX~系统的这一优点是~Word~等软件所无法企及的。

下段内容是为了让下面的长表格分居两页,看到表标题“编号(续表)”的效果。此模板的完成时间正值雨后初霁的四月二十五日,故引用林徽因《你是人间的四月天》全文:
\begin{center}
\begin{minipage}[c]{0.5\textwidth}

\textbf{你是人间的四月天}

\vspace{12pt}
我说你是人间的四月天\\
笑音点亮了四面风\\
轻灵在春的光艳中交舞着变\\
你是四月早天里的云烟\\
黄昏吹着风的软\\
星子在无意中闪\\
细雨点洒在花前\\
那轻~~那娉婷\\
你是鲜妍\\
百花的冠冕你戴着\\
你是天真~~庄严~~你是夜夜的月圆\\
雪化后那片鹅黄\\
你像~~新鲜初放芽的绿\\
你是柔嫩喜悦\\
水光浮动着你梦期待中白莲\\
你是一树一树的花开\\
是燕~~在梁间呢喃\\
你是爱~~是暖\\
是希望\\
你是人间的四月天
\end{minipage}
\end{center}

					
\wuhao\begin{longtable}{ccc}
\caption{海南大学各学院名称一览}\label{tab:table2}
 \vspace{0.5em}\\
\toprule[1.5pt] 学院名称 & 网址 & 邮政编码  \\ \midrule[1pt]
\endfirsthead
\multicolumn{3}{c}{表~\thetable(续表)}\vspace{0.5em}\\
\toprule[1.5pt] 学院名称 & 网址 & 邮政编码  \\ \midrule[1pt]
\endhead
\bottomrule[1.5pt]
\endfoot
海南大学& \url{http://www.hainu.edu.cn/}& 570228\\
机电工程学院&  \url{http://www.hainu.edu.cn/jidian/}& 570228\\
\end{longtable}\xiaosi
\vspace{\baselineskip}

绘制长表格的代码及其说明如下。
\vspace{1em}\noindent\hrule

\begin{VerbWithBreak}
\wuhao\begin{longtable}{cc...c}
\caption{表标题}\label{标签名(通常为 tab:tablename)}\\
\toprule[1.5pt] 表头第1个格 & 表头第2个格 & ... & 表头第n个格\\ \midrule[1pt]
\endfirsthead
\multicolumn{n}{c}{表~\thetable(续表)}\vspace{0.5em}\\
\toprule[1.5pt] 表头第1个格 & 表头第2个格 & ... & 表头第n个格\\ \midrule[1pt]
\endhead
\bottomrule[1.5pt]
\endfoot
表中数据(1,1) & 表中数据(1,2) & ... & 表中数据(1,n)\\
表中数据(2,1) & 表中数据(2,2) & ... & 表中数据(2,n)\\
...................................................\\
表中数据(m,1) & 表中数据(m,2) & ... & 表中数据(m,n)\\
\end{longtable}\xiaosi
\end{VerbWithBreak}

\noindent\hrule
\begin{VerbWithBreak}
在绘制长表格的前面留出一个空白行,并在第2行的一开始全局定义长表格的字号为五号字,这样能够保证长表格之前段落的行距保持不变。
在绘制长表格结束后,需要\xiaosi命令重新将字号改为小四号字。
\endhead之前的文字描述的是第2页及其之后各页的标题或表头;
\endfirsthead之前的文字描述的是第1页的标题和表头,若无此命令,则第1页的表头和标题由\endhead命令确定;
同理,\endfoot之前的文字描述的是除最后一页之外每页的表格底部内容;
\endlastfoot之前的文字描述的是最后一页的表格底部内容,若无此命令,
则最后一页的表格底部内容由\endfoot命令确定;由于规范中长表格每页底部内容均相同(水平粗线),因此模板中没有用到\endlastfoot命令。
\end{VerbWithBreak}

\noindent\hrule
\section{列宽可调表格的绘制方法}
论文中能用到列宽可调表格的情况共有两种:一种是当插入的表格某一单元格内容过长以至于一行放不下的情况,
另一种是当对公式中首次出现的物理量符号进行注释的情况。这两种情况都需要调用~tabularx~宏包。下面将分别对这两种情况下可调表格的绘制方法进行阐述。
\subsection{表格内某单元格内容过长的情况}

首先给出这种情况下的一个例子如表~\ref{tab:table3}~所示。
\begin{table}[htbp]
\caption{最小的三个正整数的英文表示法}\label{tab:table3}
\vspace{0.5em}\wuhao
\begin{tabularx}{\textwidth}{llX}
\toprule[1.5pt]
Value & Name & Alternate names, and names for sets of the given size\\\midrule[1pt]
1 & One & ace, single, singleton, unary, unit, unity\\
2 & Two & binary, brace, couple, couplet, distich, deuce, double, doubleton, duad, duality, duet, duo, dyad, pair, snake eyes, span, twain, twosome, yoke\\
3 & Three & deuce-ace, leash, set, tercet, ternary, ternion, terzetto, threesome, tierce, trey, triad, trine, trinity, trio, triplet, troika, hat-trick\\\bottomrule[1.5pt]
\end{tabularx}
\vspace{\baselineskip}
\end{table}
绘制这种表格的代码及其说明如下。
\vspace{1em}\noindent\hrule
\begin{VerbWithBreak}
\begin{table}[htbp]
\caption{表标题}\label{标签名(通常为 tab:tablename)}
\vspace{0.5em}\wuhao
\begin{tabularx}{\textwidth}{l...X...l}
\toprule[1.5pt]
表头第1个格   & ... & 表头第X个格   & ... & 表头第n个格  \\
\midrule[1pt]
表中数据(1,1) & ... & 表中数据(1,X) & ... & 表中数据(1,n)\\
表中数据(2,1) & ... & 表中数据(2,X) & ... & 表中数据(2,n)\\
.........................................................\\
表中数据(m,1) & ... & 表中数据(m,X) & ... & 表中数据(m,n)\\
\bottomrule[1.5pt]
\end{tabularx}
\vspace{\baselineskip}
\end{table}
\end{VerbWithBreak}

\noindent\hrule
\begin{VerbWithBreak}
tabularx环境共有两个必选参数:第1个参数用来确定表格的总宽度,这里取为排版表格能达到的最大宽度——正文宽度\textwidth;第2 个参数用来确定每列格式,其中标为X的项表示该列的宽度可调,其宽度值由表格总宽度确定。
标为X的列一般选为单元格内容过长而无法置于一行的列,这样使得该列内容能够根据表格总宽度自动分行。若列格式中存在不止一个X 项,则这些标为X的列的列宽相同,因此,一般不将内容较短的列设为X。
标为X的列均为左对齐,因此其余列一般选为l(左对齐),这样可使得表格美观,但也可以选为c或r。如果想要X项为中心对齐,可以在后面加<{\centering},形同\begin{tabularx}{\textwidth}{cX<{\centering}X<{\centering}}
\end{VerbWithBreak}

\noindent\hrule
\subsection{对物理量符号进行注释的情况}
为使得对公式中物理量符号注释的转行与破折号“———”后第一个字对齐,此处最好采用表格环境。此表格无任何线条,左对齐,
且在破折号处对齐,一共有“式中”二字、物理量符号和注释三列,表格的总宽度可选为文本宽度,因此应该采用\verb|tabularx|环境。
由\verb|tabularx|环境生成的对公式中物理量符号进行注释的公式如式(\ref{eq:1})所示。
%\vspace*{10pt}

\begin{equation}\label{eq:1}
\ddot{\boldsymbol{\rho}}-\frac{\mu}{R_{t}^{3}}\left(3\mathbf{R_{t}}\frac{\mathbf{R_{t}\rho}}{R_{t}^{2}}-\boldsymbol{\rho}\right)=\mathbf{a}
\end{equation}

\begin{tabularx}{\textwidth}{@{}l@{\quad}r@{———}X@{}}
式中& $\bm{\rho}$ &追踪飞行器与目标飞行器之间的相对位置矢量;\\
&  $\bm{\ddot{\rho}}$&追踪飞行器与目标飞行器之间的相对加速度;\\
&  $\mathbf{a}$   &推力所产生的加速度;\\
&  $\mathbf{R_t}$ & 目标飞行器在惯性坐标系中的位置矢量;\\
&  $\omega_{t}$ & 目标飞行器的轨道角速度;\\
&  $\mathbf{g}$ & 重力加速度,$=\frac{\mu}{R_{t}^{3}}\left(
3\mathbf{R_{t}}\frac{\mathbf{R_{t}\rho}}{R_{t}^{2}}-\bm{\rho}\right)=\omega_{t}^{2}\frac{R_{t}}{p}\left(
3\mathbf{R_{t}}\frac{\mathbf{R_{t}\rho}}{R_{t}^{2}}-\bm{\rho}\right)$,这里~$p$~是目标飞行器的轨道半通径。
\end{tabularx}
\vspace{\wordsep}

其中生成注释部分的代码及其说明如下。

\vspace{1em}\noindent\hrule

\begin{VerbWithBreak}
\begin{tabularx}{\textwidth}{@{}l@{\quad}r@{— — —}X@{}}
式中 & symbol-1 & symbol-1的注释内容;\\
     & symbol-2 & symbol-2的注释内容;\\
     .............................;\\
     & symbol-m & symbol-m的注释内容。
\end{tabularx}\vspace{\wordsep}
\end{VerbWithBreak}

\noindent\hrule

\begin{VerbWithBreak}
   tabularx环境的第1个参数选为正文宽度,第2个参数里面各个符号的意义为:
   第1个@{}表示在“式中”二字左侧不插入任何文本,“式中”二字能够在正文中左对齐,若无此项,则“式中”二字左侧会留出一定的空白;
   @{\quad}表示在“式中”和物理量符号间插入一个空铅宽度的空白;
   @{— — —}实现插入破折号的功能,它由三个1/2的中文破折号构成;
   第2个@{}表示在注释内容靠近正文右边界的地方能够实现右对齐。
\end{VerbWithBreak}

\noindent\hrule\vspace{1em}

由此方法生成的注释内容应紧邻待注释公式并置于其下方,因此不能将代码放入~\verb|table|~浮动环境中。但此方法不能实现自动转页接排,
可能会在当前页剩余空间不够时,全部移动到下一页而导致当前页出现很大空白。因此在需要转页处理时,还请您手动将需要转页的代码放入一个
新的~\verb|tabularx|~环境中,将原来的一个~\verb|tabularx|~环境拆分为两个~\verb|tabularx|~环境。

\section{为表格添加注释的方法}
首先给出这种情况下的一个例子如表~\ref{tab:table4}~所示。
\begin{table}[!ht]
   \caption{海南省相关数据}\label{tab:table4}
   \vspace{0.5em}\centering\wuhao
   \begin{threeparttable}
   \begin{tabularx}{\textwidth}{c*5{X<{\centering}}}
   \toprule[1.5pt]
   年份/年 & 农业机械总动力/千瓦 & 农村耗电量/亿千瓦小时 & 农田有效灌溉面积/千公顷 & 互联网络宽带接入用户/万户 & 长途光缆线路长度/万公里\\
   \midrule[1pt]
   2011年 & 444.33 & 7.07 & 247.51 & 16.2 & 0.33\\
   2012年 & 479.66 & 8.59 & 286.75 & 21.4 & 0.33\\
   2013年 & 502.1 & 9.63 & 260.93 & 26 & 0.33\\
   2014年 & 517.31 & 10.89 & 259.92 & 29.7 & 0.33\\
   2015年 & 511.59 & 13.02 & 263.99 & 42.2 & 0.34\\
   2016年 & 516.57 & 13.89 & 289.95 & 60.2 & 0.34\\
   2017年 & 569.8 & 15.52 & 289.25 & 79.6 & 0.34\\
   2018年 & 565.82 & 17.41 & 290.48 & 82.6 & 0.09\\
   2019年 & 581.23 & 18.68 & 290.63 & 99.7 & 0.32 \\
   2020年 & 615.57 & 20.96 & 292.21 & 111.2 & 0.32 \\
   \bottomrule[1.5pt]
   \end{tabularx}
   \begin{tablenotes}
       \footnotesize
       \item[1] 数据来源:国家统计局
   \end{tablenotes}
   \end{threeparttable}
   \vspace{\baselineskip}
\end{table}

绘制这种表格的代码及其说明如下。
\vspace{1em}\noindent\hrule
\begin{VerbWithBreak}
   \begin{table}[!ht]
      \caption{海南省相关数据}
      \vspace{0.5em}\centering\wuhao
      \begin{threeparttable}
      \begin{tabularx}{\textwidth}{c*5{X<{\centering}}}
      \toprule[1.5pt]
      年份/年 & 农业机械总动力/千瓦 & 农村耗电量/亿千瓦小时 & 农田有效灌溉面积/千公顷 & 互联网络宽带接入用户/万户 & 长途光缆线路长度/万公里\\
      \midrule[1pt]
      2011年 & 444.33 & 7.07 & 247.51 & 16.2 & 0.33\\
      2012年 & 479.66 & 8.59 & 286.75 & 21.4 & 0.33\\
      2013年 & 502.1 & 9.63 & 260.93 & 26 & 0.33\\
      2014年 & 517.31 & 10.89 & 259.92 & 29.7 & 0.33\\
      2015年 & 511.59 & 13.02 & 263.99 & 42.2 & 0.34\\
      2016年 & 516.57 & 13.89 & 289.95 & 60.2 & 0.34\\
      2017年 & 569.8 & 15.52 & 289.25 & 79.6 & 0.34\\
      2018年 & 565.82 & 17.41 & 290.48 & 82.6 & 0.09\\
      2019年 & 581.23 & 18.68 & 290.63 & 99.7 & 0.32 \\
      2020年 & 615.57 & 20.96 & 292.21 & 111.2 & 0.32 \\
      \bottomrule[1.5pt]
      \end{tabularx}
      \begin{tablenotes}
          \footnotesize
          \item[1] 数据来源:国家统计局
      \end{tablenotes}
      \end{threeparttable}
      \vspace{\baselineskip}
   \end{table}
\end{VerbWithBreak}
   \noindent\hrule
\begin{VerbWithBreak}
   这里使用了threeparttable宏包。在tablenotes环境中添加注释即可。
\end{VerbWithBreak}

若想获得绘制表格的更多信息,参见网络上的~\href{http://www.tug.org/pracjourn/2007-1/mori/}{Tables in \LaTeXe: Packages and Methods}~文档。


%% !Mode:: "TeX:UTF-8"

\chapter{数学公式的输入方法}

\section{本科生毕业论文的公式规范}

公式均需有公式号;公式号按章编排,如式(2-3);公式中各物理量及量纲均按国际标准(SI)及国家规定的法定符号和法定计量单位标注,禁止使用已废弃的符号和计量单位;公式中用字、符号、字体要符合学科规范。图、表、公式等与正文之间要有6磅的行间距。 

公式中用斜线表示“除”的关系时应采用括号,以免含糊不清,如~$a/(b\cos x)$。通常“乘”的关系在前,如~$a\cos x/b$而不写成~$(a/b)\cos x$。

不能用文字形式表示等式,如:$\textnormal{刚度}=\frac{{\textnormal{受力}}}{{\textnormal{受力方向的位移}}}$。

对于数学公式的输入方法,网络上有一个比较全面权威的文档\textbf{~\href{http://tug.ctan.org/cgi-bin/ctanPackageInformation.py?id=voss-mathmode}{Math mode}}~请大家事先大概浏览一下。下面将对学位论文中主要用到的数学公式排版形式进行阐述。

\section{生成~\LaTeX~数学公式的两种方法}

对于先前没有接触过~\LaTeX~的人来说,编写~\LaTeX~数学公式是一件很繁琐的事,尤其是对复杂的数学公式来说,更可以说是一件难以完成的任务。
实际上,生成~\LaTeX~数学公式有两种较为简便的方法,一种是基于~MathType~ 数学公式编辑器的方法,另一种是基于~MATLAB~商业数学软件的方法,
下面将分别对这两种数学公式的生成方法作一下简单介绍。

\subsection{基于~MathType~软件的数学公式生成方法}

MathType~是一款功能强大的数学公式编辑器软件,能够用来在文本环境中插入~Windows OLE~ 图形格式的复杂数学公式,所以应用比较普遍。但此软件只有~30~天的试用期,之后若再继续使用则需要付费购买才行。网络上有很多破解版的~MathType~软件可供下载免费使用,
我们推荐下载安装版本号在~6.5~之上的破解版。

以~v6.8~版本(英文版)为例~在安装好~MathType~之后,若在输入窗口中编写数学公式,复制到剪贴板上的仍然是图形格式的对象。
若希望得到可插入到~\LaTeX~编辑器中的文本格式对象,则需要对~MathType~软件做一下简单的设置:在~MathType~最上排的按钮中依次选择“Preferences
$\rightarrow$ Cut and Copy Preferences”,在弹出的对话窗中选中“MathML or TeX:”,在转换下拉框中选择“LaTeX~2.09~and later”,并将对话框最下方的两个复选框全部勾掉,点击确定。这样,再从输入窗口中复制出来的对象就是文本格式,这样就可以直接将其粘贴到~\LaTeX~
编辑器中了。按照这种方法生成的数学公式两端分别有标记~\verb|\[|~和标记~\verb|\]|~,在这两个标记之间即是真正的数学公式代码。

若希望从~MathType~输入窗口中复制出来的对象为图形格式,则只需在单击“Preferences
$\rightarrow$ Cut and Copy Preferences”弹出后的窗口中,再选中“Equation object(Windows OLE graphic)” 即可。

\subsection{基于~MATLAB~软件的数学公式生成方法}


MATLAB~是矩阵实验室(Matrix Laboratory)的简称,是美国~MathWorks~公司出品的商业数学软件。它是当今科研领域最常用的应用软件之一,
具有强大的矩阵计算、符号运算和数据可视化功能,是一种简单易用、可扩展的系统开发环境和平台。


MATLAB~中提供了一个~latex~函数,它可将符号表达式转化为~\LaTeX~数学公式的形式。其语法形式为~latex(s),其中,~s~为符号表达式,
之后再将~latex~函数的运算结果直接粘贴到~\LaTeX~编辑器中。从~\LaTeX~数学公式中可以发现,其中可能包含如下符号组合:

\begin{verbatim*}
\qquad=两个空铅(quad)宽度
\quad=一个空铅宽度
\;=5/18空铅宽度
\:=4/18空铅宽度
\,=3/18空铅宽度
\!=-3/18空铅宽度
\ =一个空格
\end{verbatim*}

所以最好将上述符号组合从数学公式中删除,从而使数学公式显得匀称美观。对于~Word~等软件的使用者来说,在我们通过~MATLAB~运算得到符号表达式形式的运算结果时,在~Word~中插入运算结果需要借助于~MathType~软件,通过在~MathType~中输入和~MATLAB~运算结果相对应的数学表达形式,之后再将~MathType~数学表达式转换为图形格式粘贴到~Word~ 中。实际上,也可以将~MATLAB~中采用~latex~函数运行的结果直接粘贴到~MathType~中,再继续上述步骤,这样可以大大节省输入公式所需要的时间。此方法在~MathType~6.5c~上验证通过,若您粘入到~MathType~中的仍然为从~MATLAB~中导入的代码,请您更新~MathType~软件。

\section{数学字体}
在数学模式下,常用的数学字体命令有如下几种:

\begin{Verbatim}[breaklines=true, breaksymbol=, breakanywheresymbolpre=]
\mathnormal或无命令 用数学字体打印文本;
\mathit             用斜体(\itshape)打印文本;
\mathbf             用粗体(\bfseries)打印文本;
\mathrm             用罗马体(\rmfamily)打印文本;
\mathsf             用无衬线字体(\sffamily)打印文本;
\mathtt             用打印机字体(\ttfamily)打印文本;
\mathcal            用书写体打印文本;
\end{Verbatim}

在学位论文撰写中,只需要用到上面提到的~\verb|\mathit|、\verb|\mathbf|~ 和~\verb|\mathrm|~命令。若要得到~Times New Roman~ 的数学字体,则需要调用~txfonts~宏包(此宏包实际上采用的是~Nimbus Roman No9 L~字体,
它是开源系统中使用的免费字体,其字符字体与~Times New Roman~字体几乎完全相同);若要得到粗体数学字体,则需要调用~bm~宏包。表~\ref{tab:fonts}~中分别列出了得到阿拉伯数字、拉丁字母和希腊字母
各种数学字体的命令。

\begin{table}[htbp]
\caption{常用数学字体命令一览}\label{tab:fonts}
\vspace{0.5em}\centering\wuhao
\begin{tabular}{llll}
\toprule
 & 阿拉伯数字\&大写希腊字母 & 大小写拉丁字母 & 小写希腊字母  \\
\midrule
斜体 & \verb|\mathit{}| & \verb|无命令| & \verb|无命令|\\
粗斜体 & \verb|\bm{\mathit{}}| & \verb|\bm{}| & \verb|\bm{}|\\
直立体 & \verb|无命令| & \verb|\mathrm{}| & \verb|字母后加up|\\
粗体 & \verb|\mathbf{}或\bm{}| & \verb|\mathbf{}| & \verb|\bm{字母后加up}|\\
\bottomrule
\end{tabular}
\vspace{\baselineskip}
\end{table}

\noindent 下面列出了一些应采用直立数学字体的数学常数和数学符号。

\vspace{-0.5em}\begin{center}\begin{tabularx}{0.7\textwidth}{XX}
$\mathrm{d}$、 $\mathrm{D}$、 $\mathrm{p}$~———微分算子 & $\mathrm{e}$~———自然对数之底数\\
$\mathrm{i}$、 $\mathrm{j}$~———虚数单位 & $\piup$———圆周率\\
\end{tabularx}\end{center}

\section{行内公式}
出现在正文一行之内的公式称为行内公式,例如~$f(x)=\int_{a}^{b}\frac{\sin{x}}{x}\mathrm{d}x$。对于非矩阵和非多行形式的行内公式,一般不会使得行距发生变化,而~Word~等软件却会根据行内公式的竖直距离而自动调节行距,如图~\ref{fig:hangju}~所示。

\begin{figure}[htbp]
\centering
\subfigure[由~\LaTeX~系统生成的行内公式]{\label{fig:subfig:latex}
                \fbox{\includegraphics[width=0.55\textwidth]{latex}}}
\subfigure[由~Word软件生成的~.doc~格式行内公式]{\label{fig:subfig:word}
                \fbox{\includegraphics[width=0.55\textwidth]{word}}}
\subfigure[由~Word软件生成的~.pdf~格式行内公式]{\label{fig:subfig:pdf}
                \fbox{\includegraphics[width=0.55\textwidth]{pdf}}}

\caption{由~\LaTeX~和~Word~生成的~3~种行内公式屏显效果}\label{fig:hangju}
\vspace{-1em}
\end{figure}

这三幅图分别为~\LaTeX~和~Word~生成的行内公式屏显效果,从图中可看出,在~\LaTeX~文本含有公式的行内,在正文与公式之间对接工整,行距不变;而在~Word~文本含有公式的行内,在正文与公式之间对接不齐,行距变大。因此从这一点来说,
\LaTeX~系统在数学公式的排版上具有很大优势。

\LaTeX~提供的行内公式最简单、最有效的方法是采用~\TeX~本来的标记———开始和结束标记都写作~\$,例如本段开始的例子可由下面的输入得到。
\verb|$f(x)=\int_{a}^{b}\frac{\sin{x}}{x}\mathrm{d}x$|

\section{行间公式}
位于两行之间的公式称为行间公式,每个公式都是一个单独的段落,例如
\[\int_a^b{f\left(x\right)\mathrm{d}x}=\lim_{\left\|\Delta{x_i}\right\|\to 0}\sum_i{f\left(\xi_i\right)\Delta{x_i}}\]
除人工编号外,\LaTeX~各种类型行间公式的标记见表~\ref{tab:eqtag}。
\begin{table}[htbp]
\caption{各种类型行间公式的标记}\label{tab:eqtag}
\vspace{0.5em}\centering\wuhao
\begin{tabularx}{\textwidth}{cll}
\toprule
& 无编号 & 自动编号\\
\midrule
单行公式& \verb|\begin{displaymath}... \end{displaymath}|& \verb|\begin{equation}... \end{equation}|\\
        & 或~\verb|\[...\]| & \\
多行公式& \verb|\begin{eqnarray*}... \end{eqnarray*}|& \verb|\begin{eqnarray}... \end{eqnarray}|\\
\bottomrule
\end{tabularx}
\end{table}

另外,在自动编号的某行公式行尾添加标签~\verb|\nonumber|,可将该行转换为无编号形式。

行间多行公式需采用~\verb|eqnarray|~或~\verb|eqnarray*|~环境,它默认是一个列格式为~\verb|rcl|~的~3~列矩阵,并且中间列的字号要小一些,因此通常只将需要对齐的运算符号(通常为等号“=”)置于中间列。

\section{可自动调整大小的定界符}

若在左右两个定界符之前分别添加命令~\verb|\left|~和~\verb|\right|,则定界符可根据所包围公式大小自动调整其尺寸,这可从式(\ref{nodelimiter}) 和式(\ref{delimiter})中看出。
\begin{equation}\label{nodelimiter}
(\sum_{k=\frac12}^{N^2})
\end{equation}
\begin{equation}\label{delimiter}
\left(\sum_{k=\frac12}^{N^2}\right)
\end{equation}
式(\ref{nodelimiter})和式(\ref{delimiter})是在~\LaTeX~中分别输入如下代码得到的。
\begin{Verbatim}[breaklines=true, breaksymbol=, breakanywheresymbolpre=]
(\sum_{k=\frac12}^{N^2})
\left(\sum_{k=\frac12}^{N^2}\right)
\end{Verbatim}
\verb|\left|~和~\verb|\right|~总是成对出现的,若只需在公式一侧有可自动调整大小的定界符,则只要用“.”代替另一侧那个无需打印出来的定界符即可。
若想获得关于此部分内容的更多信息,可参见~\href{http://tug.ctan.org/cgi-bin/ctanPackageInformation.py?id=voss-mathmode}{Math mode}~文档的第~8~章“Brackets, braces and parentheses”。

\section{数学重音符号}
数学重音符号通常用来区分同一字母表示的不同变量,输入方法如下(需要调用~\verb|amsmath|~宏包):

\vspace{0.5em}\noindent\wuhao\begin{tabularx}{\textwidth}{Xc|Xc|Xc}
 \verb|\acute| & $\acute{a}$ & \verb|\mathring| & $\mathring{a}$ & \verb|\underbrace| & $\underbrace{a}$ \\
 \verb|\bar| & $\bar{a}$ & \verb|\overbrace| & $\overbrace{a}$ & \verb|\underleftarrow| & $\underleftarrow{a}$ \\
 \verb|\breve| & $\breve{a}$ & \verb|\overleftarrow| & $\overleftarrow{a}$ & \verb|\underleftrightarrow| & $\underleftrightarrow{a}$ \\
 \verb|\check| & $\check{a}$ & \verb|\overleftrightarrow| & $\overleftrightarrow{a}$ & \verb|\underline| & $\underline{a}$ \\
 \verb|\dddot| & $\dddot{a}$ & \verb|\overline| & $\overline{a}$ & \verb|\underrightarrow| & $\underrightarrow{a}$ \\
 \verb|\ddot| & $\ddot{a}$ & \verb|\overrightarrow| & $\overrightarrow{a}$ & \verb|\vec| & $\vec{a}$ \\
 \verb|\dot| & $\dot{a}$ & \verb|\tilde| & $\tilde{a}$ & \verb|\widehat| & $\widehat{a}$ \\
 \verb|\grave| & $\grave{a}$ & \verb|\underbar| & $\underbar{a}$ & \verb|\widetilde| & $\widetilde{a}$ \\
 \verb|\hat| & $\hat{a}$
\end{tabularx}\vspace{0.5em}

\xiaosi 当需要在字母~$i$~和~$j$~的上方添加重音符号时,为了去掉这两个字母顶上的小点,这两个字母应该分别改用~\verb|\imath|~ 和~\verb|\jmath|。

如果遇到某些符号不知道该采用什么命令能输出它时,则可通过~\href{http://detexify.kirelabs.org/classify.html}{Detexify$^2$~ 网站}来获取符号命令。若用鼠标左键在此网页的方框区域内画出你所要找的符号形状,则会在网页右方列出和你所画符号形状相近的~5~个符号及其相对应的~\LaTeX~输入命令。若所列出的符号中不包括你所要找的符号,还可通过点击“Select from the complete list!”的链接以得分从低到高的顺序列出所有符号及其相对应的~\LaTeX~输入命令。

最后,建议大家还以~\href{http://tug.ctan.org/cgi-bin/ctanPackageInformation.py?id=voss-mathmode}{Math mode}~这篇~pdf~文档作为主要参考。若要获得最为标准、美观的数学公式排版形式,可以查查文档中是否有和你所要的排版形式相同或相近的代码段,通过修改代码段以获得你所要的数学公式排版形式。如果出现问题,我们推荐大家查阅~ChinaTeXMathFAQ~V1.1。

%% !Mode:: "TeX:UTF-8"

\chapter{罗列、定理和代码环境使用方法}

\section{单层罗列环境}

海南大学学位论文一般可采用两种罗列环境:一种是并列条目有同样标签的~\verb|itemize|~罗列环境,另一种是具有自动排序编号符号的~\verb|enumerate|~罗列环境。这两种罗列环境的样式参数可参考图~\ref{fig:list}。
\begin{figure}[htbp]
\centering
\includegraphics[width = 0.6\textwidth]{list}
\caption{罗列环境参数示意图}\label{fig:list}\vspace{-1em}
\end{figure}

通过调用~enumitem~宏包可以很方便地控制罗列环境的布局,其~format.tex~文件中的~\verb|\setitemize|~和~\verb|\setenumerate|~命令分别用来设置~\verb|itemize|~和~\verb|enumerate|~环境的样式参数。采用~\verb|itemize|~单层罗列环境的排版形式如下:

\begin{itemize}
\item 第一个条目文本内容
\item 第二个条目文本内容
\item 第三个条目文本内容
\end{itemize}

其代码如下

\begin{Verbatim}[breaklines=true, breaksymbol=, breakanywheresymbolpre=]
\begin{itemize}
  \item 第一个条目文本内容
  \item 第二个条目文本内容
  ...
  \item 第三个条目文本内容
\end{itemize}
\end{Verbatim}

采用~\verb|enumerate|~单层罗列环境的排版形式如下:

\begin{enumerate}
\item 第一个条目文本内容
\item 第二个条目文本内容
\item 第三个条目文本内容
\end{enumerate}

其代码如下

\begin{Verbatim}[breaklines=true, breaksymbol=, breakanywheresymbolpre=]
\begin{enumerate}
  \item 第一个条目文本内容
  \item 第二个条目文本内容
  ...
  \item 第三个条目文本内容
\end{enumerate}
\end{Verbatim}



\section{定理环境}

\begin{definition}[谱半径]\label{def:def1}
  称~$n$~阶方阵~$\mathbf{A}$~的全体特征值~$\lambda_1,\cdots,\lambda_n$~组成的集合为~$\mathbf{A}$~的谱,称
  $$\rho(\mathbf{A})=\max{\{|\lambda_1|,\cdots,|\lambda_n|\}}$$
\end{definition}
\begin{theorem}[相似充要条件]\label{lemma:l1}
  方阵$A$和$B$相似的充要条件是:~$A$~和~$B$~有全同的不变因子。
\end{theorem}
\begin{corollary}[推论1]\label{cor:cor1}
在赋范空间~$(X,\|\cdot\|)$~上定义~$d(x,y)=\|x-y\|$, 对任意~$x,y\in X$,~则~$(X,d)$~是距离空间。
\end{corollary}
\begin{proof}
  只需证明~$d(x,y)$~是距离。
\end{proof}
\newpage

定义代码如下:
\begin{Verbatim}[breaklines=true, breaksymbol=, breakanywheresymbolpre=]
 \begin{definition}[谱半径]\label{def:def1}
  称~$n$~阶方阵~$\mathbf{A}$~的全体特征值
  $\lambda_1,\cdots,\lambda_n$组成的集合为~$\mathbf{A}$~的谱,称
  $$\rho(\mathbf{A})=\max{\{|\lambda_1|,\cdots,|\lambda_n|\}}$$
\end{definition}
\end{Verbatim}
\noindent\hrule

\vspace{0.1em}\noindent\hrule
\vspace{1em}
定理代码如下:
\begin{Verbatim}[breaklines=true, breaksymbol=, breakanywheresymbolpre=]
\begin{theorem}[相似充要条件]\label{lemma:l1}
  方阵$A$和$B$相似的充要条件是:$A$和$B$有全同的不变因子。
\end{theorem}
\end{Verbatim}

\noindent\hrule\vspace{0.1em}

\noindent\hrule
\vspace{1em}
推论和证明代码如下:
\begin{Verbatim}[breaklines=true, breaksymbol=, breakanywheresymbolpre=]
\begin{corollary}[推论1]\label{cor:cor1}
在赋范空间~$(X,\|\cdot\|)$~上定义$d(x,y)=\|x-y\|$,
对任意$x,y\in X$,则$(X,d)$是距离空间。
\end{corollary}
\begin{proof}
  只需证明$d(x,y)$是距离。
\end{proof}
\end{Verbatim}
\noindent\hrule\vspace{1em}

定理定义[]中是可选参数,用来说明定理的名称。其他环境格式书写与上面定理、定义、推论格式相同,可自己调用其他环境。
若需要书写定理定义等内容,而且带有顺序编号,需要采用如下环境。除了~\verb|proof|~环境之外,其余~9~个环境都可以有一个可选参数作为附加标题。

\begin{center}
\vspace{0.5em}\noindent\wuhao\begin{tabularx}{0.7\textwidth}{lX|lX}
定理 & \verb|theorem|~环境 & 定义 & \verb|definition|~环境 \\
例 & \verb|example|~环境 & 算法 & \verb|algorithm|~环境 \\
公理 & \verb|axiom|~环境 & 命题 & \verb|proposition|~环境 \\
引理 & \verb|lemma|~环境 & 推论 & \verb|corollary|~环境 \\
注解 & \verb|remark|~环境 & 证明 & \verb|proof|~环境 \\
\end{tabularx}
\end{center}
\section{代码环境}
很多和计算机专业背景相关的同学都会使用到代码环境,使用~\verb|\verb|~指令或者是~\verb|verbatim|~环境固然是一种选择,但是比不上专门的~lstlisting~环境这么专业。
\begin{lstlisting}
int main(int argc, char ** argv)
{
printf("Hello world!\n");
return 0;
}
\end{lstlisting}

\noindent\hrule
\vspace{0.1em}\noindent\hrule

\vspace{1em}

\noindent 代码如下:
\begin{Verbatim}[breaklines=true, breaksymbol=, breakanywheresymbolpre=]
\begin{lstlisting}
int main(int argc, char ** argv)
{
printf("Hello world!\n");
return 0;
}
\end{lstlisting}
\end{Verbatim}
\noindent\hrule\vspace{1em}

在代码中显示的关键字为蓝色,框的左侧显示的是行号,这样便于读者阅读和查找代码,同时添加了浅蓝色的阴影边框,达到了美观的效果。
代码环境的设置已在~package~中~\verb|\lstset|~指令中定义。如果要修改要插入的代码的语言直接将language定义为要插入的语言即可,如插入Java那么language=Java即可。定义中支持跨页显示,可以将较长的代码置于~lstlisting~环境中。
\section{算法环境}
很多和计算机专业背景相关的同学会使用到算法环境,之前使用到的定理环境固然是一种选择,但是比不上专门的~algorithm2e~环境这么专业。为了实现专业和接近完美,本版本支持算法环境。如下所示:
\begin{algorithm}[H]
    \caption{算法标题}
    \label{alg:demoAlgo} % 贴上标签以便交叉引用
    \begin{algorithmic}[1]  % 这个 1 表示每一行都显示数字
    \STATE 初始化...
    \FOR{$i=0;i\le M; i\rightarrow i + 1$}
        \STATE 执行语句~1;
        \STATE 执行语句~2;
        \STATE ...
    \ENDFOR
    \STATE ...
    \WHILE{某条件}
        \STATE 执行语句~1;
        \STATE 执行语句~2;
        \STATE ...
    \ENDWHILE
    \STATE ...
    \end{algorithmic}
\end{algorithm}
\noindent\hrule
\vspace{0.1em}\noindent\hrule

\vspace{1em}

\noindent 代码如下:
\begin{Verbatim}[breaklines=true, breaksymbol=, breakanywheresymbolpre=]
  \begin{algorithm}[H]
    \caption{算法标题}
    \label{alg:demoAlgo} % 贴上标签以便交叉引用
    \begin{algorithmic}[1]  % 这个 1 表示每一行都显示数字
    \STATE 初始化...
    \FOR{$i=0;i\le M; i\rightarrow i + 1$}
        \STATE 执行语句~1;
        \STATE 执行语句~2;
        \STATE ...
    \ENDFOR
    \STATE ...
    \WHILE{某条件}
        \STATE 执行语句~1;
        \STATE 执行语句~2;
        \STATE ...
    \ENDWHILE
    \STATE ...
    \end{algorithmic}
\end{algorithm}
\end{Verbatim}


%\include{body/conclusion}
\end{VerbWithBreak}
那么,编译的时候就只编译未加~\%~的一章,在这个例子中,即本章~intros。

理论上,并不一定要把每章放在不同的文件中。但是这种自顶向下,分章节写作、编译的方法有利于提高效率,大大减少~Debug~过程中的编译时间,同时减小风险。

\section{参考文献生成和标注方法}

\LaTeX~具有插入参考文献的能力。Google Scholar~网站上存在兼容~BibTeX~的参考文献信息,通过以下几个步骤,可以轻松完成参考文献的生成。
\begin{itemize}
  \item 在\href{http://scholar.google.com/}{谷歌学术搜索}中,
        点击\href{http://scholar.google.com/scholar_preferences?hl=en&as_sdt=0,5}{学术搜索设置}。
  \item 页面打开之后,在\textbf{文献管理软件}选项中选择\textbf{显示导入~BibTeX~的链接},单击保存设置,退出。
  \item 在谷歌学术搜索中检索到文献后,在文献条目区域单击导入~BibTeX~选项,页面中出现文献的引用信息。
  \item 将文献引用信息的内容复制之后,添加到~references~文件夹下的~reference.bib~ 中。
\end{itemize}
\par{如在scholar中搜索“基于方差及方差梯度的指纹图像自适应分割算法”,那么其BibTex文件如下:
 "@article\{樊冬进2008基于方差及方差梯度的指纹图像自适应分割算法,
  title=\{基于方差及方差梯度的指纹图像自适应分割算法\},
  author=\{樊冬进 and 孙冰 and 封举富 and others\},
  year=\{2008\}
\}"
直接使用会报错,我们需要将”@article\{樊冬进2008基于方差及方差梯度的指纹图像自适应分割算法“改为@article\{anyEnglishName

}


在正文中标注参考文献时,在需要标注的地方输入~\verb|\cite{}|~指令,花括号内输入参考文献引用信息中的第一行信息即可(常常为文献的缩略信息),此时~\verb|[]|~符号在标注处的右上角显示。

\section{注意事项}

\begin{enumerate}
  \item 由于模板使用~UTF-8~编码,所以源文件应该保存成~UTF-8~格式,否则可能出现中文字符无法识别的错误。
  本模板中每一个~.tex~文件的文件的开头已经加上一行:\\
  \verb|% !Mode:: "TeX:UTF-8"|\\
     这样可以确保~.tex~文件默认使用~UTF-8~的格式打开。读者如果删去此行,很有可能会导致中文字符显示乱码。
     在~WinEdt~编辑器中可以使用以下两种方式保存成~UTF-8~格式:
      \begin{enumerate}
        \item 先建立~.tex~文件,另存为~.tex~文件时,选择用~UTF-8~ 格式保存。
        \item
            在~WinEdt~编辑器中,选择\\
            \mbox{~Document$\rightarrow$Document Settings$\rightarrow$Document Mode $\rightarrow$TeX:UTF-8} 同时在~WinEdt~最下面的状态栏中,可以看到该文档是~TeX~ 格式还是~TeX:UTF-8~格式。
            当文档为~TeX:UTF-8~格式时,状态栏一般显示:
            \makebox[\textwidth][l]{Wrap | Indent | INS | LINE |Spell | TeX:UTF-8 | -src~ 等。}
      \end{enumerate}
  \item 如果在~pdf~书签中,中文显示乱码的话,则注意以下说明:
    \begin{VerbWithBreak}
        \usepackage{CJKutf8}
        % 1. 如果使用CJKutf8
        %    Hyperref中应使用unicode参数
        % 2. 如果使用CJK
        %    Hyperref则使用CJKbookmarks参数
        %    可惜得到的PDF书签是乱码,建议弃用
        % 3. Unicode选项和CJKbookmarks不能同时使用
        \usepackage[
        %CJKbookmarks=true,
        unicode=true
        ]{hyperref}
     \end{VerbWithBreak}
  \item 建议采用以下两种编译方式:
  \begin{enumerate}
    \item latex + bibtex + latex + latex + dvi2pdf. 在这种编译情况下,对应的~hnumain.tex~文件的第一行是\verb|\def\usewhat{dvipdfmx}|~ (缺省设置)。 此时,所有图片文件应该保存为~.eps~格式,如~figures~文件夹里~.eps~图片。
          如果您选择在命令行中操作,可以在编译的时候依次输入~latex hnumain, bibtex hnumain, latex hnumain, latex hnumain~和~dvipdfmx hnumain, 编译完成之后,需要手动打开~pdf~文件。需要说明的是,为了是操作简便,以上命令已经作为~pdfmake.bat~批处理文件放在目录中。在编译无误的前提下,双击此文件,可以一键生成~pdf~。

    \item pdflatex + pdflatex. 在这种编译情况下,对应的~hnumain.tex~文件的第一行应该改为\verb|\def\usewhat{pdflatex}|~。 此时, 编译不支持~.eps~图片格式,此时需要在命令行下使用~epstopdf~指令将~figures~文件夹下 的~.eps~文件转化成~.pdf~ 文件格式,命令行中操作格式为~epstopdf a.eps~。
          在命令行编译的时候,依次输入~pdflatex hnumain~和~pdflatex hnumain, 编译完成之后,需要手动打开~pdf~文件。
  \end{enumerate}
    \item  当参考文献在编辑的时候,第一行为标签行,为该文献的缩略信息。当复制中文参考文献的~BibTeX~页面到~reference.bib~文件中时,需要把原来含有中文的标签行改成英文书写,否则会报错。
\end{enumerate}

\section{系统要求}

     CTeX 2.8, MiKTeX 2.8, TeX Live 2009~或以上版本。我们推荐您使用最新的~CTeX~中文套装,~CTeX 2.9.2.164~Full~版本,内含~WinEdt 7.0~编辑器,可以完成文件的编辑和编译工作。本模板目前只在~Windows~操作系统下测试通过,尚未在~Mac~ 系统和~Linux~系统下测试。

\section{Why~\LaTeX~?}

选择使用~\TeX/\LaTeX~的理由包括:
\begin{itemize}
\item 免费软件;
\item 专业的排版效果;
\item 是事实上的专业数学排版标准;
\item 广泛的西文期刊接收甚或只接收~\LaTeX~格式的投稿;
\item[] ……
\end{itemize}
不选择使用~\TeX/\LaTeX~的理由包括:
\begin{itemize}
\item 需要相当精力学习;
\item 图文混合排版能力不够强;
\item 对于表格的支持较差;
\item 仅在数学、物理、计算机等领域流行;
\item 中文期刊的支持较差;
\item[] ……
\end{itemize}

如果想知道~\LaTeX~与~Word~的详细区别,请参见:~\url{http://zzg34b.w3.c361.com/homepage/compareWord.htm}。

\subsection{我应该看什么~\LaTeX~读物~?}

我觉得最好在使用中学会~\LaTeX~,至于参考资料百度、Google上很多。如果实在需要读物,那么可以在以下几本书上进行选择。

\begin{enumerate}
\item 我能阅读英文
\begin{enumerate}
\item 迅速入门:lshort.pdf (中文版名为:一份不太简短的~\LaTeX{}~介绍)
\item 系统学习:A Guide to LaTeX, 4th Edition, Addison-Wesley
                机械工业出版社的有影印版,名曰~《\LaTeX{}~ 实用教程》
\item 深入学习:Knuth 《TeXbook》:必读。 《LATEX Companion》:如果说 高老头的~TeXbook~是论语,那么这本
               书算是一本史记,全面而精妙,是所有~\LaTeX~书中的精品。
\end{enumerate}

\item 我更愿意阅读中文
\begin{enumerate}
\item 迅速入门:lnotes2.pdf (\LaTeX~Notes~雷太赫排版系统简介, v2.0, 包太雷)
\item 系统学习:《\LaTeXe{}~科技排版指南》,邓建松(电子版)
      如果不好找,可以阅读陈志杰等《\LaTeXe~入门与提高》第二版,或者胡伟《\LaTeXe~完全学习手册》
\item 深入学习:~TeXbook0.pdf~(特可爱原本,TeXbook~的中译,xianxian)
\item 具体问题释疑:~CTeX-FAQ.pdf~,\\
        吴凌云,~\url{http://www.ctex.org/CTeXFAQ}~\\
      ~ChinaTeXMathFAQ~$V1.1$~~China\TeX~ 数学排版常见问题集
\end{enumerate}
\end{enumerate}

遇见问题和解决问题的过程可以快速提高自己的技能,建议此时:
\begin{itemize}
 \item 清楚,扼要地提出你的问题。
 \item 使用~Google~搜索。
\end{itemize}

\section{免责声明}

本模板依据《本科生毕业论文(设计)封面》和《毕业论文(设计)排版打印格式》编写,作者希望能给使用者写作论文带来方便。然而,作者不保证本模板完全符合学校要求,也不对由此带来的风险和损失承担任何责任。
