% !Mode:: "TeX:UTF-8"

\chapter{图片的插入方法}

\section{海南大学对于插图的要求}

 \par{插图要求,所有插图按分章编号,如第1章的第1张插图为“图1-1 ****图”,字体可以用宋体5号字。所有插图均需有图注(图的说明),不能只有图编号。图号及图注应在图的下方居中标出;一幅图如有若干幅分图,均应编分图号,用(a),(b),(c)......按顺序编排;插图须紧跟文述,在正文中,一般应先见图号及图的内容后再见图(即:正文中先见“本系统功能架构如图3-2所示。”字样,再在段后见图3-2),一般情况下不能提前见图,特殊情况需延后的插图不应跨节。}
\par{图形符号及各种线型画法须按照现行的国家标准;坐标图中坐标上须注明标度值,并标明坐标轴所表示的物理量名称及量纲,应均按国际标准(SD)标注,例如:kW,  m/s, N, m....等,但对一些示意图例外;图应具有“自明性”(即只看图、图题和图例,不阅读正文,就可理解图意);图中用字最小为宋体小五号字;插图必须同内容密切联系,切忌与文字和表重复。}
\par{图的绘制一定要紧凑美观并且保证清晰,避免使用彩色和带底色的图,以免影响论文印刷。}
\par{切忌直接从他人文章、其他文献、书籍扫描和网上直接拷贝图形,全文不能出现非本人绘制的图形(系统实现部分的系统界面截图除外),必须使用他人插图时,须在图题正下方注明出处。
}

\section{\LaTeX~中推荐使用的图片格式}

在~\LaTeX~中应用最多的图片格式是~EPS(Encapsulated PostScript)格式,它是一种专用的打印机描述语言,常用于印刷或打印输出。
EPS~格式图片可通过多种方式生成在这里可以采用Adobe Acrobat 软件进行转换。选中图片右击转换为pdf然后,选择保存,在保存格式下面选择文件的格式为eps即可。,对于用visio写的流程图,其转换格式和普通图片转换成eps格式完全相同。

\section{单张图片的插入方法}
单张图片独自占一行的插入形式如图~\ref{fig:xml}~所示。
\begin{figure}[htbp]
\centering
\includegraphics[width=0.4\textwidth]{XML}
\caption{树状结构}\label{fig:xml}
\vspace{\baselineskip}
\end{figure}


其插入图片的代码及其说明如下。
\vspace{1em}\noindent\hrule
\begin{VerbWithBreak}
\begin{figure}[htbp]
\centering
\includegraphics[width=0.4\textwidth]{文件名(.eps)}
\caption{标题}\label{标签名(通常为 fig:labelname)}
\vspace{\baselineskip} % 表示图与正文空一行
\end{figure}
\end{VerbWithBreak}

\noindent\hrule

\begin{VerbWithBreak}
figure环境的可选参数[htbp]表示浮动图形所放置的位置,h (here)表示当前位置,t (top)表示页芯顶部,b (bottom)表示页芯底部,p (page)表示单独一页。在Word等软件中,图片通常插入到当前位置,如果当前页的剩余空间不够,图片将被移动到下一页,当前页就会出现很大的空白,其人工调整工作非常不便。由LaTeX提供的浮动图片功能,总是会按h->t->b->p的次序处理选项中的字母,自动调整图片的位置,大大减轻了工作量。
\centering命令将后续内容转换成每行皆居中的格式。
"\includegraphics"的可选参数用来设置图片插入文中的水平宽度,一般表示为正文宽度(\textwidth)的倍数。
\caption命令可选参数“标签名”为英文形式,一般不以图片或表格的数字顺序作为标签,而应包含一定的图片或表格信息,以便于文中引用(若图片、表格、公式、章节和参考文献等在文中出现的先后顺序发生了变化,其标注序号及其文中引用序号也会跟着发生变化,这一点是Word等软件所不能做到的)。另外,图题或表题并不会因为分页而与图片或表格体分置于两页,章节等各级标题也不会置于某页的最底部,LaTeX系统会自动调整它们在正文中的位置,这也是Word等软件所无法匹敌的。
\vspace将产生一定高度的竖直空白,必选参数为负值表示将后续文字位置向上提升,参数值可自行调整。em为长度单位,相当于大写字母M的宽度。\vspace{\baselineskip} 表示图与正文空一行。
引用方法:“见图~\ref{fig:figname}”、“如图~\ref{fig:figname}~所示”等。
\end{VerbWithBreak}

\noindent\hrule\vspace{1em}

若需要将~2~张及以上的图片并排插入到一行中,则需要采用\verb|minipage|环境,如图~\ref{fig:dd}~和图~\ref{fig:ds}~所示。
\begin{figure}[htbp]
\centering
\begin{minipage}{0.4\textwidth}
\centering
\includegraphics[width=\textwidth]{dataDimensions}
\caption{数据维数的变化}\label{fig:dd}
\end{minipage}
\begin{minipage}{0.4\textwidth}
\centering
\includegraphics[width=\textwidth]{dataSize}
\caption{数据规模的变化}\label{fig:ds}
\end{minipage}
\vspace{\baselineskip}
\end{figure}

其代码如下所示。
\vspace{1em}\noindent\hrule
\begin{VerbWithBreak}
\begin{figure}[htbp]
\centering
\begin{minipage}{0.4\textwidth}
\centering
\includegraphics[width=\textwidth]{文件名}
\caption{标题}\label{fig:f1}
\end{minipage}
\begin{minipage}{0.4\textwidth}
\centering
\includegraphics[width=\textwidth]{文件名}
\caption{标题}\label{fig:f2}
\end{minipage}\vspace{\baselineskip}
\end{figure}
\end{VerbWithBreak}

\noindent\hrule

\begin{VerbWithBreak}
minipage环境的必选参数用来设置小页的宽度,若需要在一行中插入n个等宽图片,则每个小页的宽度应略小于(1/n)\textwidth。
\end{VerbWithBreak}

\noindent\hrule

\section{具有子图的图片插入方法}

图中若含有子图时,需要调用~subfigure~宏包, 如图~\ref{fig:subfig}~所示。
\begin{figure}[htbp]
  \centering
  \subfigure[Data Dimensions]{\label{fig:subfig:datadim}
                \includegraphics[width=0.4\textwidth]{dataDimensions}}
  \subfigure[Data Size]{\label{fig:subfig:datasize}
                \includegraphics[width=0.4\textwidth]{dataSize}}
  \caption{Scalability of data}\label{fig:subfig}
\vspace{\baselineskip}
\end{figure}

其代码及其说明如下。
\vspace{1em}\noindent\hrule

\begin{VerbWithBreak}
\begin{figure}[htbp]
  \centering
  \subfigure[第1个子图标题]{
            \label{第1个子图标签(通常为 fig:subfig1:subsubfig1)}
            \includegraphics[width=0.4\textwidth]{文件名}}
  \subfigure[第2个子图标题]{
            \label{第2个子图标签(通常为 fig:subfig1:subsubfig2)}
            \includegraphics[width=0.4\textwidth]{文件名}}
  \caption{总标题}\label{总标签(通常为 fig:subfig1)}
\vspace{\baselineskip}
\end{figure}
\end{VerbWithBreak}

\noindent\hrule

\begin{VerbWithBreak}
子图的标签实际上可以随意设定,只要不重复就行。但为了更好的可读性,我们建议fig:subfig:subsubfig格式命名,这样我们从标签名就可以知道这是一个子图引用。
引用方法:总图的引用方法同本章第1节,子图的引用方法用\ref{fig:subfig:subsubfig}来代替。
\end{VerbWithBreak}

\noindent\hrule\vspace{1em}

子图的引用示例:如图~\ref{fig:subfig:datadim}~和图~\ref{fig:subfig:datasize}~所示。

若想获得插图方法的更多信息,参见网络上的~\href{ftp://ftp.tex.ac.uk/tex-archive/info/epslatex.pdf}{Using Imported Graphics in \LaTeX and pdf\LaTeX}~文档。
