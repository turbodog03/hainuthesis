% !Mode:: "TeX:UTF-8"

%%  可通过增加或减少 setup/format.tex中的
%%  第274行 \setlength{\@title@width}{8cm}中 8cm 这个参数来 控制封面中下划线的长度。

\fancypagestyle{plain}{
\fancyhf{}
\renewcommand{\headrulewidth}{0 pt}
\fancyfoot[C]{\song\xiaowu~\thepage~}
}
% 目录具体项目需要在setup/format.tex第236行处添加(如果增加新项需要在188行处定义)
\cheading{海南大学}      % 设置正文的页眉
\ctitle{\LaTeX~模板操作说明书}    % 封面用论文标题,自己可手动断行
\caffil{机电工程学院} % 学院
\csubject{软件工程}   % 专业
\cgrade{2014~级}            % 年级
\cauthor{孙浩然}            % 姓名
\cnumber{20140501310092}        % 学号
\csupervisor{XXX}        % 导师
\xibie{机械系}                          %系别
\nianji{2014~级}                            %年级
%\cdate{\CTEXdigits{\the\year} 年\CJKnumber{\the\month} 月 \CJKnumber{\the\day} 日}  % 显示为中文日期格式
\cdate{~~\the\year~~年~~~~\the\month~~~~月~~~~\the\day~~~~日}
\newif\ifShowCabs
\ShowCabstrue           %注释则不显示中文摘要

\ifShowCabs
\cabstract{
中文摘要:论文第一页为中文摘要,一般为300字左右(约A4纸张1/2页)为好。中文摘要应以最简洁的语言对自己的论文及工作进行全面的介绍,是全文的一个高度概括。用句应精炼概括。
为了便于文献检索,应在本页下方另起一行注明论文的关键词(3-5个),之间用分号相隔。关键词应体现论文的主要内容,词组符合学术规范。中文“摘要”标题字用小三号加重黑体字,摘要内容用小四号宋体字。
}
\ckeywords{关键字1;关键字2;关键字3;关键字4}

\fi

\newif\ifShowEabs
\ShowEabstrue           %注释则不显示英文摘要

\ifShowEabs
\eabstract{
外文摘要:外文摘要另起一页。外文摘要内容应与中文摘要基本相一致,要语句通顺,语法正确,准确地反映论文的内容,并在其后列出与中文相对应的外文关键词。“摘要”标题字用加重小三号字体Times New Roman字体大写字母,摘要内容用小四号Times New Roman字体小写字母。
}

\ekeywords{kwords1 ; kwords2 ;kwords3}
\fi

\makecover

\clearpage
